\documentclass[parskip=half,a4paper,titlepage,bibliography=totoc]{scrreprt}

%==========================Bindekorrektur Option, oben in Documenclass hinzufügen===========
%,BCOR=15mm

%===========================PRÄAMBEL===============================
% Packages
%\usepackage[left=3cm,right=4cm,top=3cm,bottom=6cm]{geometry}
\usepackage[ngerman]{babel}
\usepackage[utf8]{inputenc}
\usepackage[T1]{fontenc}
\usepackage{setspace}
\setstretch{1,3}
\usepackage{array}
\usepackage{tabularx}
\usepackage{hyperref}
\usepackage{nameref}
\usepackage{graphicx}
\usepackage{epstopdf}
\usepackage{tikz}
\usepackage{color}
\usepackage{algorithmic}
\usepackage[absolute,overlay]{textpos}
\usepackage{babelbib} 
\usepackage{titleref}
\usepackage[colorinlistoftodos, textwidth=2cm, shadow, textsize = tiny]{todonotes}

%\setlength{\marginparwidth}{3cm}
%\usepackage{amsmath,amssymb,amsthm,amsfonts}
%\newcommand{\changefont}[3]{\fontfamily{#1} \fontseries{#2} \fontshape{#3} \selectfont}
%\usepackage{array}
%\usepackage{calc}
%\usepackage{scrlayer-scrpage}
%\usepackage{tabularx}
%\addtokomafont{caption}{\scriptsize} 

%===>Trennungsregeln für das gesamte Dokument.

%
%%% Trennungsregeln für das gesamte Dokument.
%
\hyphenation{da-bei}
\hyphenation{ge-schlos-sen-er}
\hyphenation{ent-spre-chen}
\hyphenation{lie-gen}
\hyphenation{Ge-sichts-ver-fol-gung}
\hyphenation{Re-prä-sen-tat-ion}
\hyphenation{pe-ri-phe-ren}
\hyphenation{an-ge-nom-me-ne}
\hyphenation{ge-speich-ert}
\hyphenation{ü-ber-nimmt}
\hyphenation{spei-chert}
\hyphenation{Schwan-kun-gen}
\hyphenation{Be-we-gungs-lo-sig-keit}
\hyphenation{be-reits}
\hyphenation{eu-kli-di-sche}
\hyphenation{ver-gleichs-wei-se}
\hyphenation{Im-ple-men-tier-un-gen}
\hyphenation{Haupt-auf-ga-be}
\hyphenation{dy-na-misch-es}
\hyphenation{ge-ne-rie-ren}
\hyphenation{aus-schließ-lich}
\hyphenation{auf-wei-sen}
\hyphenation{ge-ne-rie-ren}
\hyphenation{Zu-sam-men-set-zung}
\hyphenation{di-gi-ta-len}
\hyphenation{Mensch}
\hyphenation{Farb-ver-ständ-nis}
\hyphenation{Farb-wer-te}
\hyphenation{HSV-Farbraum}
\hyphenation{Schät-zung}
\hyphenation{kon-kur-rier-en-de}
\hyphenation{Ite-ra-ti-on}
\hyphenation{ge-samt-en}
\hyphenation{wis-sen-schaft-lich-en}
\hyphenation{di-gi-tal-en}
\hyphenation{Ma-schi-ne}
\hyphenation{Mo-dell}
\hyphenation{Be-wer-tungs-funk-tion}
\hyphenation{wel-ches}
\hyphenation{ganz-heit-liche}
\hyphenation{Be-ob-ach-tungs-me-cha-nis-mus}
\hyphenation{rgb-Farb-raum}
\hyphenation{Greif-be-we-gung-en}
\hyphenation{gauß-verteilt}
\hyphenation{auf-zu-füh-ren}
\hyphenation{hu-ma-no-i-den}
\hyphenation{O-ber-kör-per-Ver-fol-gung}
\hyphenation{ge-tes-te-ten}
\hyphenation{re-prä-sen-tiert}
\hyphenation{schlech-te-res}
\hyphenation{be-kann-ten}
\hyphenation{hoch-wer-ti-ge}
\hyphenation{ent-sprech-en-den}
\hyphenation{pro-blem-spe-zi-fisch-en}
\hyphenation{ein-ge-gang-en}
\hyphenation{ver-glei-che}
\hyphenation{Farb-zu-sam-men-setz-ung}
\hyphenation{Farb-raum}
\hyphenation{be-wer-ten}
\hyphenation{glo-ba-le}
\hyphenation{Be-wer-tung-en}
\hyphenation{Kon-fi-gu-ra-ti-ons-räu-men}
\hyphenation{Film-in-dus-trie}
\hyphenation{durch-zu-füh-ren}
\hyphenation{aus-ge-stat-et}
\hyphenation{igno-rie-ren}
\hyphenation{be-zeich-net}
\hyphenation{Ver-bin-dung}
\hyphenation{aus-zu-füh-ren}
\hyphenation{In-for-ma-ti-on-en}
\hyphenation{ent-wi-ckelt}
\hyphenation{Brenn-weite}
\hyphenation{Hin-ter-grund-ele-mente}
\hyphenation{ei-ner}
\hyphenation{be-stimmt}
\hyphenation{geo-me-trisch-e}
\hyphenation{be-schrie-ben}
\hyphenation{be-schrei-ben}
\hyphenation{be-stimmt}
\hyphenation{be-schreibt}
\hyphenation{Wahr-schein-lich-keits-funk-ti-on-en}
\hyphenation{Wahr-schin-lich-keits-funk-ti-on}
\hyphenation{mar-ker-lo-sen}
\hyphenation{Re-chen-leis-tung}
\hyphenation{Ste-re-o-ka-me-ra}
\hyphenation{In-te-res-se}
\hyphenation{Schwell-wer-tes}
\hyphenation{Kopf-po-si-ti-on-en}
\hyphenation{un-ter-such-ten}
\hyphenation{hie-rar-chi-sche}
\hyphenation{ge-gen-sei-tig}
\hyphenation{fin-det}
\hyphenation{be-trach-tet}
\hyphenation{Be-ob-ach-tungs-sche-ma}
\hyphenation{Ro-bo-ters}
\hyphenation{er-mög-lich-en}
\hyphenation{grei-fen-de}
\hyphenation{lin-ker}
\hyphenation{ge-spei-chert}
\hyphenation{drei-di-men-sio-na-len}
\hyphenation{Be-fin-den}
\hyphenation{dar-ge-stellt}
\hyphenation{müs-sen}
\hyphenation{be-stimmt}
\hyphenation{Si-tu-a-tio-nen}
\hyphenation{des-halb}
\hyphenation{letz-ten}
\hyphenation{Zu-fals-wer-tes}
\hyphenation{mö-gli-che}
\hyphenation{er-wei-tert}
\hyphenation{durch-schnitt-lich}
\hyphenation{Ro-bo-ter-kopf}
\hyphenation{schluss-end-lich}
\hyphenation{Farb-wahr-schein-lich-keits-ver-tei-lung} 


%===> Config
\graphicspath{{./Bilder/}}

\newcommand{\myname}{Peter Michael Bolch}
\newcommand{\mymanr}{1345211}
\newcommand{\mytitle}{Analyse internationaler Nachrichtenflüsse
im Twitter-Netzwerk}
\newcommand{\mysubmissiondate}{2014}
\newcommand{\myadvisor}{Matthias Keller}
\newcommand{\myprofessor}{Lehrstuhl \\ Prof. Dr. Hannes Hartenstein}
\newcommand{\myinstitute}{Dezentrale Systeme und Netzdienste\\ Institut für Telematik }
\newcommand{\myfaculty}{Fakultät für Informatik}
\newcommand{\myseminar}{Diplomarbeit}
\newcommand{\myseminardate}{2014}

%==============================================================================

% Macros
\newcommand{\name}[1]{\textit{#1}}
\newcolumntype{C}{>{\centering\arraybackslash}p{2em}}
% \renewcommand{\floatpagefraction}{.7}

%==============================================================================

\begin{document}

%=======================TITLEPAGES=============================================

%\newgeometry{left=2cm,top=2cm}
\pagestyle{empty}
% coordinates for the background shape on the titlepage
\newcommand{\diameter}{15} % Runde Ecken
\newcommand{\xone}{-60} % abstand von links, größer -> weiter nach rechts
%\newcommand{\xone}{-20}
\newcommand{\xtwo}{460} % abstand von rechts, kleiner -> weiter nach links
\newcommand{\yone}{70} % abstand von oben, kleiner -> weiter nach unten
\newcommand{\ytwo}{-710} % abstand von untern, größer -> weiter nach unten

\renewcommand{\titlefont}{\sffamily}
\sffamily

\begin{titlepage}

% [ Rahmen ]
\begin{tikzpicture}[overlay]
\draw[thick,color=black]  
 		 (\xone pt, \yone pt)
  -- (\xtwo pt, \yone pt)
 arc (90:0:\diameter pt) 
  -- (\xtwo + \diameter pt , \ytwo pt) 
	-- (\xone + \diameter pt , \ytwo pt)
 arc (270:180:\diameter pt)
	-- (\xone pt, \yone pt);
\end{tikzpicture}

% [ Logo ]
%\begin{textblock}{10}[0,0](1.7,1.125)
\begin{textblock}{10}[0,0](1.7,0.9)
	\includegraphics[width=.3\textwidth]{Bilder/KITLogo_RGB.pdf}
\end{textblock}

\begin{center}

%\hbox{}
%\vskip 4.0cm
\vfill
{\huge\bfseries\titlefont \textbf{Hier steht der Titel der\\
                Diplom- oder Studienarbeit} \par}

\vskip 2.0cm

{\Large Diplomarbeit}\\
{\Large von}\\

\vskip 1.0cm

{\LARGE{\textbf{Vorname~Nachname}}}\\

\vskip 1.0cm

{\large an der Fakultät für Informatik, Institut für Telematik}\\
{\large Forschungsbereich Dezentrale Systeme und Netzdienste}\\

\vskip 4.0cm

\begin{tabular}{l<{\hspace{15pt}}l}
Erstgutachter:      & Prof.~Dr.~Hannes~Hartenstein \\
Zweitgutachter:     & ?? \\
Betreuer:           & Dipl.-Inform.~Vorname~Nachname \\
Bearbeitungszeit:   & 01.~Januar~2010 -- 30.~Juni~2010 \\
\end{tabular}
\end{center}

\begin{textblock}{10}[0,0](1.75,15.4)
\tiny{ 
KIT -- Universit\"at des Landes Baden-W\"urttemberg und nationales Forschungszentrum der Helmholz-Gesellschaft
}
\end{textblock}

\begin{textblock}{10}[0,0](12,15.36)
\normalsize{
	\textbf{www.kit.edu} 
}
\end{textblock}

  
\end{titlepage}
\rmfamily
\clearpage


		% Titelseiten Datei einbinden
%\restoregeometry
\vspace*{32\baselineskip}
\hbox to \textwidth{\hrulefill}
\par
\hfill \newline
Ich erkläre hiermit, dass ich die vorliegende Diplomarbeit selbständig verfasst und
keine anderen als die angegebenen Quellen und Hilfsmittel verwendet habe.
\hfill \newline \newline
Karlsruhe, \mysubmissiondate
\hfill
\myname

\clearpage

\pagestyle{plain} 

%=======================INHALTSVERZEICHNIS======================================
\pagenumbering{roman}	% römische Ziffern für das Inhaltsverzeichnis
\setcounter{page}{1}
%%%%%%%%%%%%%%%%%%%%%%%%%%%%%%%%%%%%%%%%%%%%%%%%%%%%%%%%%%%%%%%%%%%%%%%%%%%%%%%
%\begin{center}
%\begin{minipage}{10cm}
%\em
%{\bf Kurzfassung:} 
%\end{minipage}
%\end{center}
%\vspace{3cm}

\tableofcontents
%\listoftodos

\clearpage
%\pagestyle{empty}
\pagenumbering{arabic} % ab hier: Seitennummerierung in arabischen Ziffern
\clearpage

%!TEX root = ../document.tex
\chapter{Einleitung}\label{chp:Einleitung}

	\section{Motivation und Hintergründe}

		Über den Mikroblogging-Dienst Twitter lassen sich in Echtzeit 140 Zeichen lange Textnachrichten veröffentlichen.
		Seit dem Start des Mikroblogging-Dienstes im Jahr 2006 sind die Nutzerzahlen kontinuierlich angestiegen.
		2010 konnte Twitter 75 Millionen aktive Nutzer verzeichnen \cite{Cheng2010}.
		Im Jahr 2013 wird Twitter täglich von zirka 100 Millionen Menschen weltweit aktiv genutzt.
		Dies berichtete Twitter 2013 in seinem Prospekt zum Börsengang \cite{twitterinc2013}.  
		Zur Gesamtanzahl der Nutzer-Konten gibt es von Twitter keine Informationen. 
		Dies kann mitunter damit begründet werden, dass die Gesamtanzahl der Nutzer-Konten auch inaktive Nutzer einschliesst und somit keine Informationen über die tatsächliche Aktivität im Netzwerk liefert. 
		Auch andere soziale Netzwerke ziehen die aktiven Nutzer als Metrik heran, des weiteren wird die Metrik vom Interactive Advertising Bureau (IAB) empfohlen. \cite{IAB}
		
		Die Twitter-Nutzer verfassen täglich mehr als 500 Millionen Nachrichten, sogenannte Tweets \cite{twitterinc2013}.
		Die meisten dieser Tweets sind öffentlich zugänglich und können von allen Twitter-Nutzern uneingeschränkt betrachtet werden. 
		Twitter bietet zusätzlich eine sogenannte Streaming-API an, welche es ermöglicht Tweets programmatisch zu empfangen. \footnote{API: Application Programming Interface oder auch Programmierschnittstelle}
		Die Streaming-API stellt ein Echtzeit-Sample der aktuell versendeten Tweets bereit und liefert laut Twitter maximal 1\% aller Tweets die zum aktuellen Zeitpunkt verfasst wurden.
		Über die sogenannte Filter-API lassen sich die Tweets nach bestimmten Kriterien wie Nutzer-ID, geografischer Region oder Schlüsselwörtern Filtern.
		 \footnote{https://dev.twitter.com/docs/streaming-apis} 

		Ein Tweet besteht aus einer Reihe von Informationen, neben dem Verfasser ist der Tweet-Text die wichtigste Information die in einem Tweet enthalten ist.  
		Der Tweet-Text wird vom Nutzer verfasst und abgesendet, er bildet die zentrale Information. 
		In den 140 Zeichen teilen Twitter-Nutzer Nachrichten von unterschiedlicher Ausprägung mit.
		Unter anderem wird über privates, Sportergebnisse, Großereignisse, persönliche Erfahrungen und Meinungen berichtet. 
		Auch Bilder und Links können in einem Tweet-Text versendet werden. 

		Mit Hilfe der Streaming-API ist es erstmals möglich große Mengen nutzergenerierte Informationen unterschiedlichster Ausprägung direkt zu erhalten. 
		Durch die Möglichkeiten die Twitter bietet kann theoretisch jeder Mensch Nachrichten und Informationen über das Twitter-Netzwerk verbreiten und weitergeben. 
		Diese Masse an nutzergenerierten Informationen bietet Wissenschaftlern in verschiedenen Bereichen zahlreiche neue Möglichkeiten.

		Sakaki et al interpretieren die Twitter-Kurznachrichten beispielsweise als Sensor-Daten \cite{Sakaki2010}.
		Der Twitter-Nutzer fungiert dabei als Sensor, der ein beliebiges Ereignis erfährt oder erlebt.
		Möglicherweise berichtet der Twitter-Nutzer im Text der Twitter-Kurznachricht über dieses Ereignis. 
		Damit kann der Text als Sensor-Datum interpretiert werden, wenn auch mit erheblichm Rauschen.    
		Sakaki et al zeigen aber, dass mit diesem Vorgehen, Erdbebenzentren lokalisiert oder die Trajektorie eines Typhoons vorhergesagt werden können.  
		
		Auch die Sozialwissenschaften und die Meinungsforschung profitieren von dem enormen Informationsfundus der durch Twitter geboten wird.  
		Tumasjan et al. untersuchen in \cite{Tumasjan2011} wie sich die politische Landschaft im Twitter-Netzwerk wiederspiegelt. 
		Die Wissenschaftler haben zur Bundestagswahl 2009 100.000 Tweets analysiert und stellten fest, dass die Erwähnungen von Parteien und Politikern in Twitter, den Wahlausgang sehr genau wiederspiegelten.  
		Die Kommunikation innerhalb des Twitter-Netzwerks kann aber auch neue Einsichten über die globale Kommunikation oder die Ausbreitung von Nachrichten liefern.
		Garcia-Gavilanes et al. erforschen in \cite{Garcia-Gavilanes2014} die Kommunikation zwischen Ländern. 
		Es wird gezeigt, dass die globale Kommunikation innerhalb des Twitter-Netzwerks nicht nur von der geografischen Distanz abhängig ist, sondern auch von sozialen, ökonomischen und kulturellen Attributen eines Landes.   
		Selbst die Epidemieforschung kann von den Daten des Twitter-Netzwerks profitieren. 
		So zeigten Szomsor et al. in \cite{Szomszor2011}, dass die Vorhersage der Schweingrippe im Jahr 2009 durch die Analyse von Twitter Daten eine Woche früher möglich gewesen wäre als dies mit konventionellen Frühwarnsystemen der Fall war. 

		Diese Erkenntnisse und Informationen sind allerdings nur gewinnbringend einzusetzen, wenn der Standort des Twitter-Nutzers bekannt ist. 
		Die Information, dass eine Krankheit ausgebrochen ist, ist mit einer exakten Georeferenz wertvoller als ohne diese. 
		Auch die Arbeit von Sakaki et al. ist auf eine Georeferenz angewiesen, wobei die Wissenschaftler ausführen, dass die ungefähre Position für ihre Anwendung ausreichend ist.
		Bei der Untersuchung internationaler Kommunikation wiederum, ist es wichtig zu Wissen aus welchem Land eine Twitter-Kurznachricht abgesetzt wurde.
		In diesem Fall kann die Georefrenz einen weiteren Raum umfassen und muss nicht GPS-Genauigkeit aufweisen.  
		Wohingegen eine detaillierte Untersuchung des politischen Klimas innerhalb Deutschlands eine Auflösung auf Bundesländer-Ebene erforderlich machen würde. 

 		
 		Twitter bietet seinen Nutzern die Möglichkeit ihren Standort im Nutzerprofil anzugeben. 
 		Hecht et al. stellen in \cite{Hecht2011} eine erste ausführliche Analyse der eingegebenen Standort-Daten bereit.  
 		Ab 2009 ermöglichte Twitter ein "'per-tweet geo-tagging"' \cite{Cheng2010}.
 		Dadurch können Anwendungen, auf Endgeräten mit GPS, Längen- und Breitengrad des aktuellen Standorts als Georeferenz an die Twitter-Kurznachricht anhängen.    
		Nur ca. 1,7\% der Twitter-Kurznachrichten enthalten allerdings eine konkrete Georeferenz in dieser Form.


	\section{Problembeschreibung} 
		Um das volle Potenzial der Informationen in Twitter-Kurznachrichten auszuschöpfen ist es wichtig die Position des Twitter-Nutzers, beziehungsweise den Ort wo eine Twitter-Kurznachricht abgeschickt wurde, bestimmen zu können. 
		Die Anzahl der Twitter-Kurznachrichten die unmittelbar einem geografischen Ort zugeordnet werden können ist sehr gering. 
		
		Es ist also wichtig ein Verfahren zu finden um Twitter-Nutzer oder Twitter-Kurznachrichten eine Georeferenz zuzuordnen. 
		Mit Hilfe, der in einem Tweet vorhandenen Daten sollte eine möglichst genaue Position bestimmt werden. 
		Dies soll auch möglich sein, wenn keine konkrete geografische Angabe in Form von Längen- und Breitengrad vorliegt. 

	\section{Fragestellungen und Anforderungen}\label{sec:fragestellung}
		Folgende Fragestellungen sollen beantwortet werden: 
		\begin{enumerate}
			\item[Q1] Wie kann Twitter-Nutzern, unter zuhilfenahme des Standort-Feldes und der Zeitzone, eine Georeferenz zugeordnet werden?
			\item[Q2] Kann die Lokalisierung von Twitter-Nutzern durch die Anwendung von probabilistischen Sprach-Modellen auf das Standort-Feld im Vergleich zum nachschlagen in einer Geodatenbank verbessert werden? 
		\end{enumerate}
		  
	\section{Anforderungen}\label{sec:Anforderungen}
		\begin{enumerate}
			\item[R1] Möglichst exakte Zuordnung einer Georefrenz zu einem Twitter-Nutzer. (R1) 
			\item[R2] Unabhängig von kommerziellen Anbietern geografischer Informationen, oder sonstiger benötigter Daten. (R2)
			\item[R3] Das Ergebniss ist eine Georeferenz, dabei soll vom Anwender individuell zwischen folgenden Hierarchieebenen gewählt werden können (R3) : 
			\begin{enumerate}
			 	\item Land
			 	\item Verwaltungsebene erster Ordnung \footnote{in D bspsw. Länder, Baden-Württemberg, Bayern usw. }
			 	\item Verwaltungsebene zweiter Ordnung \footnote{in D bspsw. Regierungsbezirke, Regierungsbezirk Stuttgart, Regierungsbezirk Karlsruhe usw.}
			 	\item Stadt
			 \end{enumerate} 
			\item[R4] Es soll möglich sein eine Mindestanforderung für die Konfidenz, mit welcher die Georeferenz bestimmt wurde anzugeben.
			\item[R5] Für jedes Ergebnis wird die Sicherheit, mit der  
			\item[R6] Verfahren unabhängig von Sprache und Schriftzeichen weltweit einsetzbar.
		\end{enumerate}
		
	\section{Gliederung der Arbeit}

		\subsection*{Abschnitt 2: Grundlagen}
			In diesem Abschnitt sollen die Grundlagen für die entwickelte Methode vermittelt werden. 
			Es wird auf den Mikroblogging-Dienst Twitter eingegangen und es werden grundsätzliche Methoden und Verfahren vorgestellt welche zum Verständniss der entwickelten Methode benötigt werden. Ebenso werden geografische Grundbegriffe vermittelt, welche in der vorliegenden Arbeit häufig genutzt werden.

		\subsection*{Abschnitt 3: Stand der Technik}
			Es werden aktuelle Ansätze betrachtet, eingeordnet und in Bezug auf die angegebenen Anforderungen untersucht.
			Es werden sowohl die Verfahren zur '"Analyse'" und Zuordnung als auch die Verfahren zum abbilden der geografischen Einheiten untersucht und eingeordnet. 

		\subsection*{Abschnitt 4: Lösungsansatz}
			In diesem Abschnitt wird die erarbeitete Methode erläutert und im Detail erklärt. 
			Hier werde ich entweder einen Top-Down Ansatz oder einen Bottom Up Ansatz wählen.

			Top-Down:
			\begin{enumerate}
				\item Genereller Aufbau der Wissensbasis \footnote{Datenbankschema oder Informationsschema} 
				\item Lokalisierung von Social Media Daten (Lokalisierungsprozess) 
				\item Geografische Hierarchiebenen \footnote{In Grundlagen und Stand der Technik behandelt bei Geografie, hier nur erklären wie verwednet wirdHier bin ich mir unsicher ob dies Sinn macht. 
			Theoretisch könnte man hier die geografischen Standards und Grundbegriffe definieren sowie die genutzten Komponenten der Implemnetierung.}
				\item Sicherheit anhand der Verteilungswahrscheinlichkeiten
				\item Einsatz der geografischen Hierarchiebenen zur Justierung der Sicherheit    
				\item NGramme zur Repräsentation der Indikatoren
			\end{enumerate}

			Bottom-Up:

			\begin{enumerate}
				\item NGramme aus Indikatoren erzeugen
				\item Geomapping
				\item Datenstruktur
				\item Treffer zählen (NGramm + Geoid gleich usw.)
				\item Geografische Hierarchiebene
				\item Unsicherheit bei Lokalisierung messen (neuer Daten) 
				\item Justierung der Lokalisierungsunsicherheit auf geografischen Hierarchiebenen
			\end{enumerate}

		\subsection*{Abschnitt 5: Referenzimplementierung der entwickelten Methode}
			Es werden ausgewählte Auszüge, Probleme und Fallstricke der Referenzimplementierung erläutert und erklärt. 

		\subsection*{Abschnitt 6: Leistungsbewertung der entwickelten Methode}
			In diesem Kapitel werden die Ergebnisse der Refernzimplementierung bewertet und, soweit sinnvoll, gegenüber bestehenden Ansätze einer kritischen Betrachtung unterzogen. 


		\subsection*{Abschnitt 7: Schlussfolgerungen}
			Unter besonderer Berücksichtigung der Ergebnisse des letzten Kapitels werden Schlussfolgerungen gezogen. 
			Der Beitrag und nutzen der entwickelten Methode soll kritisch hinterfragt werden.

		\subsection*{Abschnitt 8: Zusammenfassung und Ausblick}
			Zusammenfassung der Arbeit und kritischer Rückblick. Im Ausblick werden mögliche Verbesserungen und Ideen zur Weiterentwicklung gegeben.  

%!TEX root = ../document.tex
\chapter{Grundlagen und Stand der Technik} 


	\section{Social Media}

		\subsection{Geoinformationen in Social Media Daten}
	 		\begin{enumerate}
	 			\item \todo{gesichert und ungesicherte Geoniformationen definieren!} gesicherte Geoinformationen vs. ungesicherte Geoinformationen
	 			\item konkrete Geolocations (bsp. Städte <--> Länder) \cite{Hecht:2011:TJB:1978942.1978976}
	 			\item unmittelbar geografische Indikatoren
				\item mittelbar geografische Daten bspsw. Hashtags, Inhaltsanalysen ohne spezielle geografische Hinweise
	 			\item Lokalisierung von Social-Media Elementen (Videos, User, Nachrichten, Bilder) kleine Übersicht
	 			\item Hinleitung zu Twitter  
	 		\end{enumerate}

		\subsection{Twitter}
			Allgemeine Informationen zu Twitter. 
			\begin{enumerate}
				\item Was ist Twitter -> Tweets/Mechanismen/"'Wie wird Twitter genutzt"''
				\item \todo{Paper raussuchen -> Einfluss von Twitter auf Weltbild/Meinung/ } Einfluss von Twitter auf Weltbild/Meinung/ usw.
				\item Twitter als Nachrichtenmedium (Can Twitter Replace Newswire (Petrovic et. al ))
				\item Anatomie eines Tweets 
					\begin{enumerate}
						\item Welche Informationen sind in einem Tweet enthalten? 
						\item Konzentration auf Daten die Hinweise zur räumlichen Lage geben könnten aber auch allgemein auf die Daten eingehen.
					\end{enumerate}
			\end{enumerate}
	
	
	\section{Geografische Grundbegriffe und Geografiedaten}

		\subsection{Geografische Grundbegriffe}

		\subsection{Geonames.org}
			Allgemeines zu geonames.org, was ist geonames.org. 
			\begin{enumerate}
				\item Woher stammen die Daten?
				\item Umfang und Informationen
				\item Aktualität
				\item Hierarchiebeziehungen im geonames.org Datensatz
			\end{enumerate}	

		\subsection{}	

	\section{???} 
		\subsection{N-Gramme}
			\begin{enumerate}
				\item NGramme allgemein, Verwendung, Beispiele. 
				\item \todo{NGramme -> Nochmal genau prüfen, Zusammenhang zu Markov Modell und NGram Statistik herausstellen} Zusammenhang zwischen Länge/Grad eines N-Grammes und Wahrscheinlichkeiten. -> mathematische Herleitung?!
			\end{enumerate}

	\section{Stand der Technik}
		Zweistufiger Prozess bei den meisten, mir bekannten Ansätzen.
		Untersuchung auf Häufungen von Informationen bzw. Indikatoren anhand der konkreten geografischen Angaben. Meistens Cluster Verfahren auf geografischen Daten in Verbindung mit Indikatoren/Informationen die vorverarbeitet wurden.

		\begin{enumerate}
			\item Naiver Ansatz -> Geocoding mit Google Maps API V3, nur Indikatoren die geografische Namen enthalten. 
					Prinzipiell einfache Datenbankabfrage mit ein wenig semantik. 
					Keine Jargon Namen wie Big Apple etc.
				\begin{enumerate}
					\item Funktion der GMaps Api V3
					\item Einschränkungen der GMaps Api V3
					\item zurückgelieferte Daten der GMaps Api V3
					\item Kurze Beschreibung wie ich die API genutzt habe
				\end{enumerate}
			\item aktuelle Ansätze
				\begin{enumerate}
					\item{\todo{Ist das grundsätzliche Verfahren, analysieren Inhalt/Indikatoren -> Zuordnen auf geografische Angaben und danach Clustern tatsächlich immer gleich bei allen Arbeiten? kontinuierlich vs diskrete geografische Daten} 
					allgemeiner Ansatz : Geotagged Tweets analysieren (Inhalt/andere Indikatoren usw. ), zuordnen zu geografischen \"Bereichen\" und daraus lernen.}
					\item Verfahren mit Inhaltsanalysen
					\item Verfahren mit Indikatoren einzelne oder mehrere
					\item Welche Verfahren kommen beim mapping auf \todo{geografische Entität definieren} geografische Entitäten zum Einsatz
				\end{enumerate}
		\end{enumerate}

		\subsection{Probleme früherer Ansätze}
			\begin{enumerate}
				\item{Genutzte API's und Indikatoren nur in bestimmten Sprachen verfügbar}
				\item{keine Schätzung für Genauigkeit auf verschiedenen geografischen Hierarchieebenen verfügbar}  
			\end{enumerate}


\chapter{Entwurf} 

\section{Indikatoren zur Bestimmung der geografischen Lokation}

	\subsection{Geografische Indikatoren}
		\begin{enumerate}
			\item Mögliche Alternativen
			\item Begründung warum Userlocation und Timezone
			\item Beispiele und Auswertungen (manuell getaggter Datensatz)
			\item Verweis auf \"in justin biebers heart\"
		\end{enumerate}

	\todo{Wie detailliert hier auf Framework eingehen? Präprozessor-Konzept zur universellen Vorverarbeitung, oder eher in Implementierung}
	\subsection{Vorverarbeitung der Indikatoren (Präprozessor-Konzept)}
		\begin{enumerate}
			\item geonames matching (geonames tree) für geografische Namen bestehend aus mehreren Wörtern
			\item Eliminierung von Sonderzeichen
			\item Tokenizing
			\item Ngram Erzeugung
			\item \todo{Checken wie oft das vorkommt und wie groß der Nutzen ist} Zeitzone als \"schärfenden Indikator für doppeldeutige Namen\"
		\end{enumerate}

	\subsection{Encoding}
		Problematik unterschiedlicher Sprachen, 
		url-encoding sinnvoll als Vorbereitung auf Webservice. 

\section{Geolocation Mapping}

	\subsection{nearest neighbour mapping}
		\begin{enumerate}
			\item \todo{Welches Fehlermaß kann ich hier anwenden(Recherche)} Wie genau kann gemappt werden? Fehler Durchschnitt
			\item Mapping auf cities 1000/1000/15000 mit Daten zu durchschnitll. Abstand
			\item Hier ist noch Verbesserungspotenzial -> wenn Mapping Distanz zu weit entfernt -> verwerfen! 
		\end{enumerate} 



\section{Wissensgenerierung}
	
	\subsection{Generierung eines Wissendatensatzes}

	\subsection{Counter}

	\subsection{Verknüpfung mit Geodaten}

	\subsection{Auflösen auf Administartionsebenen, Länder}

\section{Lokalisieren von Tweets ohne konkrete geografische Daten}







%!TEX root = ../document.tex
\chapter{Implementierung} 

%!TEX root = ../document.tex
\chapter{Leistungsbewertung} 

\chapter{Schlussfolgerungen, Ausblick und Fragen} 

%!TEX root = ../document.tex
\chapter{Zusammenfassung} 

%!TEX root = ../document.tex
\chapter{Ideen und Notizen}

	\section{Stakeholder analyse}
	Welche potenziellen Stakeholder profitieren von der Arbeit? 
	Was benötigt jeder dieser Stakeholder? Bedürfnisse analysieren und Begründen.  

	\begin{enumerate}
		\item Marketing Professionals
		\item Statistiker allgemein
		\item Sozialwissenschaftler -> Analyse von Informationsströmen
	\end{enumerate}



	\section{Ideen}


	\begin{enumerate}
		\item \todo{In Einleitung} Voraussetzungen zur Anwendung des Verfahrens
		\begin{enumerate}
			\item Lerndaten mit konkreten geografischen Angaben
			\item Indikatoren in Lerndaten, welche auch in Datensätzen ohne konkrete geografische Angaben vorkommen (hier eventuelle Diskrepanzen zwischen geogetaggten und nicht geogetaggten tweets + Mentalität in bestimmten Ländern)
			\item Indikatoren mit geografischem Bezug, oder hinreichendem geografischen Bezug, Mittelbar oder unmittelbar
		 \end{enumerate}
		 \item Auf Jargon Namen für Städte eingehen, wie bspsw. the big apple -> New York City 
		 \item Landesgrenzen-Problematik wird durch meine Lösung obsolet -> auf stakeholder eingehen
		 \item \todo{Korrelation zwischen Lokalisierungungssicherheit und tatsächlichem Match berechnen} Wahrscheinlichkeiten für korrekte Lokalisierung kann angegeben und justiert werden 
		 \item Wenn Wahrscheinlichkeiten auf best. Ebene nicht hoch genug dann verschieben auf Admin2 -> Admin1 -> Länderebene
		 \item mit vorherigem werden Unsicherheiten bei Lokalisierung abgebildet (Wichtig für Informationsflüsse) 
		 \item  
	\end{enumerate}
	  


	  \section{Formulierungen}
	  	\subsection{unmittelbare ungesicherte geografische Indikatoren}
	  		Das "'userlocation"' Feld in einem Tweet kann durchaus eine konkrete Lokation beinhalten, jedoch wird auch oft irgendetwas eingetragen. \cite{Hecht2011}
	  		Es kann sich dabei um beliebige Wörter oder Sätze handeln, die einzige Limitierung ist die Anzahl zur Verfügung stehender Zeichen. Nichtsdestotrotz ist es das Ziel dieses Feldes seinen eigene Standort anzugeben.  Dabei kann allerdings nicht davon ausgegangen werde, das der eingetragene Wert nicht doch in einem Zusammenhang mit einer geografischen Lokation steht. 
	  		Bezeichnungen von Städten in Umgangssprache wie besipielsweise "The Big Apple" für New York City oder Motown für Detroit, sind für einige Personen nicht unmittelbar zuzuordenen, geben allerdiings eine konkrete Lokation an. Da die Masse an Bei bzw. Spitznamen für Städte nicht überschaubar ist und auch sprachliche Probleme bestehen ist es sinnvoll alle userlocation Einträge gleich zu behandeln und diese in erster Linie als Lokationsangaben zu behandeln. Durch die Einschränkung auf eine Geolocation werden einzelne gleich lautende Einträge, welche aber nicht auf einen konkreten Ort hinweisen in einzelnen Datensätzen abgelegt. 
	
\nocite{*}								% Alle Refenezen werden aufgelistet
\bibliographystyle{alpha} 				% Layout Stil der Bibliographie
 \bibliography{bibtex/dipl}                     % Literatur Datei einbinden
			
\end{document}
