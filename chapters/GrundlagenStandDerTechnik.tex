\chapter{Grundlagen und Stand der Technik} 


	\section{Social Media}

		\subsection{Geoinformationen in Social Media Daten}
	 		\begin{enumerate}
	 			\item \todo{definieren!} gesicherte Geoinformationen vs. ungesicherte Geoinformationen
	 			\item konkrete Geolocations (bsp. Städte <--> Länder) Referenz: From Justin Biebers Heart 
	 			\item unmittelbar geografische Indikatoren
				\item mittelbar geografische Daten bspsw. Hashtags, Inhaltsanalysen ohne spezielle geografische Hinweise
	 			\item Lokalisierung von Social-Media Elementen (Videos, User, Nachrichten, Bilder) kleine Übersicht
	 			\item Hinleitung zu Twitter  
	 		\end{enumerate}

		\subsection{Twitter}
			Allgemeine Informationen zu Twitter. 
			\begin{enumerate}
				\item Was ist Twitter -> Tweets/Mechanismen/"Wie wird Twitter genutzt"
				\item \todo{Paper raussuchen} Einfluss von Twitter auf Weltbild/Meinung/ usw.
				\item Twitter als Nachrichtenmedium (Can Twitter Replace Newswire (Petrovic et. al ))
				\item Anatomie eines Tweets 
					\begin{enumerate}
						\item Welche Informationen sind in einem Tweet enthalten? 
						\item Konzentration auf Daten die Hinweise zur räumlichen Lage geben könnten aber auch allgemein auf die Daten eingehen.
					\end{enumerate}
			\end{enumerate}
	
	\todo{schlechte Überschrift!} 
	\section{Geografische Grundbegriffe}

		\subsection{Geonames.org}
			Allgemeines zu geonames.org, was ist geonames.org. 
			\begin{enumerate}
				\item Woher stammen die Daten?
				\item Umfang und Informationen
				\item Aktualität
				\item Hierarchiebeziehungen im geonames.org Datensatz
			\end{enumerate}	

		\subsection{}	

	\section{} 
		\subsection{N-Gramme}
			\begin{enumerate}
				\item NGramme allgemein, Verwendung, Beispiele. 
				\item \todo{Nochmal genau prüfen, Zusammenhang zu Markov Modell und NGram Statistik herausstellen} Zusammenhang zwischen Länge/Grad eines N-Grammes und Wahrscheinlichkeiten. -> mathematische Herleitung?!
			\end{enumerate}

	\section{Stand der Technik}
		Zwei vorgehen bei aktuellen Ansätzen
		Zum ersten, dass zusammenfassen zu Gruppen von Hinweisen auf einen bestimmten Standort und das abbilden dieser Gruppen auf geografische Entitäten oder direktes abbilden der Indikatoren auf den Globus und das darauffolgende Gruppieren nach Indikatoren um Häufungen festzustellen.   
		\begin{enumerate}
			\item Naiver Ansatz -> Geotagging mit Google Maps API V3, nur Indikatoren die geografische Namen enthalten. 
					Prinzipiell einfache Datenbankabfrage mit ein wenig semantik. 
					Keine Jargon Namen wie Big Apple etc.
				\begin{enumerate}
					\item Funktion der GMaps Api V3
					\item Einschränkungen der GMaps Api V3
					\item zurückgelieferte Daten der GMaps Api V3
					\item Kurze Beschreibung wie ich die API genutzt habe
				\end{enumerate}
			\item aktuelle Ansätze
				\begin{enumerate}
					\item{\todo{in allen anderen Arbeiten gleiches Prinzip?} 
					allgemeiner Ansatz : Geotagged Tweets analysieren (Inhalt/andere Indikatoren usw. ), zuordnen zu geografischen \"Bereichen\" und daraus lernen.}
					\item Verfahren mit Inhaltsanalysen
					\item Verfahren mit Indikatoren einzelne oder mehrere
					\item Welche Verfahren kommen beim mapping auf \todo{geografische Entität definieren} geografische Entitäten zum Einsatz
				\end{enumerate}
		\end{enumerate}

		\subsection{Probleme früherer Ansätze}
			\begin{enumerate}
				\item{Genutzte API's und Indikatoren nur in bestimmten Sprachen verfügbar}
				\item{keine Schätzung für Genauigkeit auf verschiedenen geografischen Hierarchieebenen verfügbar}  
			\end{enumerate}
