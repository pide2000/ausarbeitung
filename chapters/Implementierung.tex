%!TEX root = ../document.tex
\chapter{Implementierung} 
Im Rahmen dieser Diplomarbeit ist eine Referenzimplementierung des vorgestellten Verfahrens entstanden.
In Auszügen soll die Referenzimplementierung hier vorgestellt werden. 
Hierbei sollen insbesondere Probleme bei der Umsetzung betrachtet werden, und wie diese gelöst wurden. 
Damit soll die Möglichekit gegeben werden, in eigenen Implementierungungen die Probleme frühzeitig zu erkennen und zu vermeiden. 
Des weiteren soll ein Überblick über die genutzten Datensätze und API's gegeben werden. \todo{Datensätze in Grundlagen?} 

\section{Komponenten der Referenzimplementierung}

	\subsection{Architektur}
	Allgemeine Architektur der Refernzimplementierung 

	\subsection{Präprozessorverarbeitung - Erzeugung der N-Gramme}
	Warum Präprozessoren -> schnelleres ändern der Vorverarbeitung.



\section{Datenbank}

\subsection{}

\todo{Eventuell was über die Geo Indexe in der Datenbank und die Nearest Neighbour Berechnungen.} 



\section{Geografie Daten} \todo{in Implementierung verschieben} 

\section{Data Sample}
	Beschreibung wie Daten erzeugt wurden, Zeiträume, Analysen

\section{geonames.org}	
Allgemeines zu geonames.org, was ist geonames.org. 
			\begin{enumerate}
				\item Woher stammen die Daten?
				\item Umfang und Informationen
				\item Aktualität
				\item Hierarchiebeziehungen im geonames.org Datensatz
			\end{enumerate}	