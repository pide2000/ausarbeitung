%!TEX root = ../document.tex
\chapter{Zusammenfassung} 

	Über das Twitter-Netzwerk werden täglich mehr als 500 Millionen Tweets und damit Informationen verbreitet und weitergegeben. 
	Die Tweets sind zum Größtenteil öffentlich zugänglich. 
	Aus dieser enormen Informationsmenge lassen sich Erkenntnisse ableiten, mit denen politische und wirtschaftliche Entscheidungsprozesse unterstützt und verbessert werden können. 
	Im Katastrophenfall können schnelle Informationen den Entscheidungsträgern wichtige Hinweise liefern, die notwendigen, präventiven Schutzmaßnahmen einzuleiten.
	Um die Informationen effizient nutzen zu können, müssen den Tweets geografische Koordinaten zugeordnet werden.
	Allerdings weisen nur ca. 1\% der Twitter-Nachrichten geografische Koordinaten auf.

	Mit dem in dieser Arbeit vorgestellten Verfahren ist es möglich, Tweets über Angaben im Nutzer-Standort und der Nutzer-Zeitzone einer geografischen Position zuzuordnen.
	Die Geolokalisierung von Tweets erfolgt über eine Wissensdatenbank (Georeferenz-Basis), die mit einem Lerndatensatz bestehend aus Tweets mit den benötigten Informationen eingelernt wird. 
	Aus diesen Lerndatensätzen werden Referenzwerte extrahiert, denen eine Georeferenz zugeordnet wird.
	Damit kann das Vorkommen der Referenzwert an einer geografischen Postion oder in einer geografischen Region erfasst und in der Georeferenz-Basis gespeichert werden.
	Die Georeferenz-Basis stellt eine Art probabilistisches Sprachmodell für geografische Positionen oder Regionen dar.
	Soll ein Tweet lokalisiert werden, werden aus dessen Nutzer-Standortfeld und Nutzer-Zeitzone potenzielle geografische Indiaktoren erzeugt.
	Durch eine Abfrage dieser potenziellen geografischer Indikatoren an diese Georeferenz-Basis werden diejenigen Datensätze zurückgegeben, für die der Referenzwert mit einem der potenziellen geografischen Indikatoren korrespondiert.
	Es liegt dann eine Menge an möglichen Georeferenzen zu dem Tweet vor.
	Durch eine Analyse der Verteilung der Referenzwerte wird dann die wahrscheinlichste Georeferenz gewählt.
	Durch die Angabe von Schwellwerten für die Verteilung kann das Verfahren justiert werden, für eine gegeben Anforderung an die Precision können entsprechende Schwellwerte gewählt werden.
	Der Einfluss auf den Recallwert ist vorhersehbar.
	Dies macht das Verfahren beherrschbar und die Ergebnisse vorhersagbar.
	Des weiteren kann durch die Wahl der Schwellwerte eine Optimierung des Trade-Off durchgeführt werden um die bestmöglichen Ergebnisse zu erzielen.
	Es kann zudem bestimmt werden, ob das Ergebnis eine Stadt(geografische Koordinaten), Verwaltungseinheit erster Ordnung(beispielsweise ein Bundesland), Verwaltungseinheit zweiter Ordnung(beispielsweise Regierungsbezirke) oder ein Land sein soll. 
	Das Wissen über den gewünschten Rückgabewert wird einbezogen, um die Ergebnisse zu optimieren.
	Es werden keine Einschränkungen bezüglich der verwendeten Sprache oder des verwendeten Alphabets gemacht.
 	
	Es konnten folgende Ergebnisse erzielt werden. 
	Für den Rückgabewert Stadt wurde eine maximale Ergebnismenge von 79,1\% bei einem Median der Fehlerdistanzen von 20,43km erzielt.
	Der beste Wert für den Median der Fehlerdistanzen liegt bei 3,24km mit einer Ergebnismenge von 0,99\%.
	Die Optimierung des Median zieht eine Absenkung der Ergebnismenge nach sich. 
	Durch die Evaluierung wurden Schwellwerte für den besten möglichen Trade-Off zwischen Median und Ergebnismenge ermittelt.
	Dadurch konnte eine Ergebnismenge von 53,21\% und ein Median von 9,17km erreicht werden. 
	Im Vergleich zum besten Wert des Median der Fehlerdistanzen (3,24km) konnte durch eine moderate Änderung des Median die Ergebnismenge um ca. das 50-fache (von ) gesteigert werden.
	Auf den restlichen geografischen Hierarchieebenen wurden als Kennzahlen Precision und Recall verwendet.
	Für die Verwaltungseinheit zweiter Ordnung konnte für den besten Trade-Off eine Precision von 77\% bei einem Recall von 12\%, für die Verwaltungseinheit zweiter Ordnung lag der beste Trade-Off eine Precision von 80\% bei einem Recall von 54\%.
	Auf Länderebene lag der beste Wert für die Precision bei 99\% bei einem Recall von 23\%, dies bedeutet nahezu alle Tweets, denen durch die Geolokalisierung ein Land zugeordnet werden konnte, wurden dem korrekten Land zugeordnet.
	Der beste Trade-Off lag hier bei einer Precision von 92\% und einem Recall von 84\%.



