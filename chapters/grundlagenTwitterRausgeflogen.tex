\paragraph{Antworten und direktes ansprechen eines Nutzers}
					Twitter bietet die Möglichkeit einzelne Nutzer direkt anzusprechen. 
					Mit Hilfe des @-Symbols kann ein Nutzer referenziert werden. 
					Der referenzierte Nutzer, besipielsweise @alfred, wird dann benachrichtigt, dass er in einem Tweet erwähnt wurde. 
					Der erwähnte Nutzer muss dabei nicht Follower des Verfassers sein. \todo{siehe Bild ref1} 
					Eine weitere Funktion im Twitter-Netzwerk ist das Antworten auf einen Tweet.
					Über eine Schaltfläche wird es ermöglicht auf einen Tweet zu Antworten. 
					Das @-Symbol und der Nutzername des Verfassers werden automatisch eingetragen, womit eine Benachrichtigung an den Verfasser des Ursprungstweets erfolgt. \todo{siehe Bild ref2}
					Es ist möglich, das auf einen Antwort-Tweet wiederum geantwortet wird, wodurch ein sogenannter Thread oder Konversation entsteht. \todo{siehe Bild ref3}
					Auch ist es möglich, dass an einer solchen Konversation mehrere Twitter-Nutzer beteiligt sind. 
					Dies ist dann der Fall, wenn im ursprünglichen Tweet, auf weitere User referenziert wurde. 
					Aber auch wenn ein Nutzer auf eine bestehende Konversation antwortet, werden alle beteiligtien Nutzer referenziert. \todo{siehe Bild ref4}
					\todo{Diagramm Antwort, Antwort Thread, Bild Antworten Button, Referenzieren}     


				\paragraph{Favorisieren}
					Mit dieser Funktion lässt sich ausdrücken, dass man einen Tweet interessant oder gut findet.
					Auch Zustimmung wird durch favorisieren ausgedrückt.  
					Einen Tweet zu favorisieren kann aber auch bedeuten '"ich habe deine Reaktion registriert"', oft um einen Antwort-Thread nicht abrupt abzubrechen sondern eine zustimmende Rückmeldung zu geben ohne extra einen Tweet zu verfassen.