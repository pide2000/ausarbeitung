\chapter{Einleitung}

\section{Motivation und Hintergründe}
Motivation aus Proposal (abändern).

\section{Problembeschreibung} 
Allgemein die Problematik der Lokalisierung von Social Media Daten betrachten und erläutern. 
Danach insbesondere auf Twitter und die Problematik der Informationsflüsse eingehen.

\section{Fragestellungen und Zielsetzungen}
Wie können Interaktionen, Benutzer, oder Daten aus Sozialen Netzwerken lokalisiert werden, wenn keine geografischen Koordinaten gemacht wurden? 
Wie können diese auf konkrete geografische Entitäten abgebildet werden. 

\section{Gliederung der Arbeit}

\subsection{Grundlagen und Stand der Technik}
In diesem Kapitel sollen die Grundlagen für die entwickelte \todo{Universelle Methode ist unter Umständen zu weit gefasst}  "Universelle Methode zur Ortsbestimmung von Social Media Daten in Abwesenheit geografischer Koordinaten" vermittelt werden. 
\todo{Hier muss aufgelistet werden was alles an Grundlagen vermittelt wird}
Des weiteren werden aktulle Ansätze bezüglich der Lokalisierung von Social Media Daten untersucht, die verschiedenen Verfahren untersucht und die Probleme der aktuellen Lösungen diskutiert.

\subsection{Technologien und Standards}
Hier bin ich mir unsicher ob dies Sinn macht. 
Theoretisch könnte man hier die geografischen Standards und Grundbegriffe definieren sowie die genutzten Komponenten der Implemnetierung.

\todo{Universelle Methode ist unter Umständen zu weit gefasst}
\subsection{Entwicklung einer universellen Methode zur Ortsbestimmung von Social Media Daten in Abwesenheit geografischer Koordinaten oder anderer konkreter Ortsangaben}
In diesem Kapitel wird die erarbeitete Methode erläutert und im Detail erklärt. 

Top Down
\begin{enumerate}
	\item Genereller Aufbau der Wissensbasis
	\item Lokalisierung von Social Media Daten (Lokalisierungsprozess) 
	\item Geografische Hierarchiebenen
	\item Sicherheit anhand der Verteilungswahrscheinlichkeiten
	\item Einsatz der geografischen Hierarchiebenen zur Justierung der Sicherheit    
	\item NGramme zur Repräsentation der Indikatoren
\end{enumerate}

\subsection{Referenzimplementierung der entwickelten Methode}
Ausgewählte Auszüge, Probleme und Fallstricke der Referenzimplementierung erläutert und erklärt. 


\subsection{Leistungsbewertung der entwickelten Methode}
In diesem Kapitel werden die Ergebnisse der Refernzimplementierung bewertet und, soweit sinnvoll, gegenüber bestehenden Ansätze einer kritischen Betrachtung unterzogen. 


\subsection{Schlussfolgerungen}
Unter besonderer Berücksichtigung der Ergebnisse des letzten Kapitels werden Schlussfolgerungen gezogen. 
Der Beitrag und nutzen der entwickelten Methode soll kritisch hinterfragt werden.

\subsection{Zusammenfassung und Ausblick}
Zusammenfassung der Arbeit und kritischer Rückblick. Im Ausblick werden mögliche Verbesserungen und Ideen zur Weiterentwicklung gegeben.  