%!TEX root = ../document.tex
\chapter{Einleitung}

	\section{Motivation und Hintergründe}
	\todo{Motivation aus Proposal einfügen und dahingehend abändern, dass Analyse der Tweet-Retweet Paare fehlt} 

	\section{Problembeschreibung} 
	Allgemein die Problematik der Lokalisierung von Social Media Daten betrachten und erläutern.
	Danach insbesondere auf Twitter und die Problematik der Informationsflüsse eingehen.

	\section{Fragestellungen und Zielsetzungen}
	Wie können Interaktionen, Benutzer, oder Daten aus Sozialen Netzwerken lokalisiert werden, auch wenn keine geografischen Koordinaten angegeben sind? 
	Lokalisierung anhand von Indikatoren bzw. Sekundärinformationen.
	\footnote{hier konkrete, mittelbare und unmittelbare geografische Indikatoren umschreiben um diese später zu definieren-> keine Vorwärtsverweise} 
	Wie können diese auf konkrete geografische Entitäten \footnote{geografische Entität noch nicht definiert, allgemeine geografische Begriffe verwenden}  abgebildet werden. 

	\section{Gliederung der Arbeit}

		\subsection*{KAPITEL1: Grundlagen und Stand der Technik}
			In diesem Kapitel sollen die Grundlagen für die entwickelte "'Methode zur Ortsbestimmung von Social Media Daten in Abwesenheit geografischer Koordinaten"' \footnote{Eventuell als "'Universelle Methode zur ..."} vermittelt werden. 
			Des weiteren werden aktuelle Ansätze bezüglich der Lokalisierung von Social Media Daten untersucht, die verschiedenen Verfahren untersucht und die Probleme der aktuellen Lösungen diskutiert.

		\subsection*{KAPITEL2: Technologien und Standards}
		Fussnote beachten!
			\footnote{Hier bin ich mir unsicher ob dies Sinn macht. 
			Theoretisch könnte man hier die geografischen Standards und Grundbegriffe definieren sowie die genutzten Komponenten der Implemnetierung.} 

		\subsection*{KAPITEL3: Entwicklung einer Methode zur konkreten Ortsbestimmung von Social Media Daten in Abwesenheit geografischer Koordinaten oder anderer konkreter Ortsangaben}
			In diesem Kapitel wird die erarbeitete Methode erläutert und im Detail erklärt. 
			Hier werde ich entweder einen Top-Down Ansatz oder einen Bottom Up Ansatz wählen.

			Top-Down:
			\begin{enumerate}
				\item Genereller Aufbau der Wissensbasis \footnote{Datenbankschema oder Informationsschema} 
				\item Lokalisierung von Social Media Daten (Lokalisierungsprozess) 
				\item Geografische Hierarchiebenen \footnote{In Grundlagen und Stand der Technik behandelt bei Geografie, hier nur erklären wie verwednet wirdHier bin ich mir unsicher ob dies Sinn macht. 
			Theoretisch könnte man hier die geografischen Standards und Grundbegriffe definieren sowie die genutzten Komponenten der Implemnetierung.}
				\item Sicherheit anhand der Verteilungswahrscheinlichkeiten
				\item Einsatz der geografischen Hierarchiebenen zur Justierung der Sicherheit    
				\item NGramme zur Repräsentation der Indikatoren
			\end{enumerate}

			Bottom-Up:

		\begin{enumerate}
			\item NGramme aus Indikatoren erzeugen
			\item Geomapping
			\item Datenstruktur
			\item Treffer zählen (NGramm + Geoid gleich usw.)
			\item Geografische Hierarchiebene
			\item Unsicherheit bei Lokalisierung messen (neuer Daten) 
			\item Justierung der Lokalisierungsunsicherheit auf geografischen Hierarchiebenen
		\end{enumerate}

		\subsection*{KAPITEL4: Referenzimplementierung der entwickelten Methode}
			Es werden ausgewählte Auszüge, Probleme und Fallstricke der Referenzimplementierung erläutert und erklärt. 


		\subsection*{KAPITEL5: Leistungsbewertung der entwickelten Methode}
			In diesem Kapitel werden die Ergebnisse der Refernzimplementierung bewertet und, soweit sinnvoll, gegenüber bestehenden Ansätze einer kritischen Betrachtung unterzogen. 


		\subsection*{KAPITEL6: Schlussfolgerungen}
			Unter besonderer Berücksichtigung der Ergebnisse des letzten Kapitels werden Schlussfolgerungen gezogen. 
			Der Beitrag und nutzen der entwickelten Methode soll kritisch hinterfragt werden.

		\subsection*{KAPITEL7: Zusammenfassung und Ausblick}
			Zusammenfassung der Arbeit und kritischer Rückblick. Im Ausblick werden mögliche Verbesserungen und Ideen zur Weiterentwicklung gegeben.  