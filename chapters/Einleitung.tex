%!TEX root = ../document.tex
\chapter{Einleitung}\label{chp:Einleitung}

	\section{Motivation}

		Die Verbreitung von Nachrichten und Informationen erfolgt immer stärker durch nutzergenerierte Inhalte.
		Eine Plattform hierfür bietet der Microblogging-Dienst Twitter. 
		Die Nutzer können 140 Zeichen lange Nachrichten, sogenannte "'Tweets"', erstellen und veröffentlichen können.
		Längst ist Twitter zu einem Massenphänomen geworden und übernimmt die Rolle eines Nachrichtenmediums \cite{Petrovic2013}.
		Die Twitter-Nutzer verfassen täglich mehr als 500 Millionen Tweets \cite{twitterinc2013}. 
		Durch die Möglichkeiten die Twitter bietet, kann theoretisch jeder Mensch Nachrichten und Informationen über das Twitter-Netzwerk verbreiten und weitergeben. 
		In den Tweets wird unter anderem über Großereignisse, persönliche Erfahrungen oder Erlebnisse berichtet. 
		Die Tweets sind zum Größtenteil öffentlich zugänglich.
		Dieses enorme Potenzial an Informationen sollte genutzt und verarbeitet werden.
		
		Es lassen sich daraus Erkenntnisse ableiten, mit denen politische und wirtschaftliche Entscheidungsprozesse unterstützt und verbessert werden können. 
		Aber auch Gefahrenpotentiale können rechtzeitig erkannt werden, um gezielte Gegenmaßnahmen einzuleiten.
		Dies setzt aber voraus, dass die Informationen aus den Tweets möglichst exakten geografischen Koordinaten zugeordnet und dadurch die Ereignisse lokalisiert werden können.

		Sakaki et al zeigen, dass mit Hilfe von Tweets mit geografischen Koordinaten Erdbebenzentren lokalisiert oder die Trajektorie eines Typhoons vorhergesagt werden können \cite{Sakaki2010}.  
		Tumasjan et al. untersuchen in \cite{Tumasjan2011} wie sich die politische Landschaft im Twitter-Netzwerk widerspiegelt. 
		Die Wissenschaftler haben zur Bundestagswahl 2009 100.000 Tweets analysiert und stellten fest, dass die Anzahl der Erwähnungen von Parteien und Politikern in Twitter, den Wahlausgang sehr genau abbildeten.  
		
		Die Kommunikation innerhalb des Twitter-Netzwerks kann aber auch neue Einsichten über die globale Kommunikation oder die Ausbreitung von Nachrichten liefern.
		Garcia-Gavilanes et al. erforschen in \cite{Garcia-Gavilanes2014} die Kommunikation zwischen Ländern. 
		Es wird gezeigt, dass die globale Kommunikation innerhalb des Twitter-Netzwerks nicht nur von der geografischen Distanz abhängig ist, sondern auch von sozialen, ökonomischen und kulturellen Attributen eines Landes.   

		Selbst die Epidemieforschung kann von den Daten des Twitter-Netzwerks profitieren. 
		So zeigten Szomsor et al. in \cite{Szomszor2011}, dass die Vorhersage der Schweinegrippe im Jahr 2009 durch die Analyse von Tweets eine Woche früher möglich gewesen wäre, als dies mit konventionellen Frühwarnsystemen der Fall war. 

 		Diese wichtigen Erkenntnisse und Vorhersagen konnten nur aufgrund von Tweets mit geografischen Koordinaten ermittelt werden.
 		Allerdings weisen nur ca. 1\% der Twitter-Kurznachrichten geografische Koordinaten auf \cite{Schulz2013}.
 		Dies ist ein sehr geringer Wert, wenn mehr Tweets mit einer zugeordneten geografischen Position zur Verfügung stehen würden, könnten diese Verfahren effizienter genutzt werden.
 		
	\section{Problembeschreibung} 

 		Wie kann es ermöglicht werden, dass Tweets ohne geografische Koordinaten eine geografische Position zugeordnet werden kann?

 		Twitter bietet seinen Nutzern diverse Möglichkeiten persönliche Angaben zu machen.
		Unter anderem kann im Nutzerprofil ein Standort angegeben werden. 
		Bei der Eingabe des Nutzer-Standortes wird vom Twitter-Nutzer abgefragt, wo dieser sich befindet. 
		Die Intention der Abfrage zielt also darauf ab, dass der Nutzer einen Wert eingibt, der auf ein geografisches Objekt verweist. 
		Dieses Feld eignet sich deshalb gut, um daraus eine geografische Position abzuleiten.
		Betrachtet man den Nutzer-Standort jedoch genauer, fällt auf, dass dieser nicht ohne weiteres zur Bestimmung einer geografischen Position verwendet werden kann.
		
		Naheliegend ist, dass der Nutzer im Nutzer-Standort einen Ortsnamen (Toponyme) verwendet, um seinen Standort anzugeben.
		Der Nutzer-Standort wird jedoch über ein Freitext-Feld eingegeben und direkt abgespeichert.
		Durch die freie Eingabe werden unter anderem Abkürzungen, größere geografische Regionen oder spezielle Bei- und Spitznamen im Nutzer-Standort eingegeben. 
		Wenn der Nutzer-Standort ein Toponym darstellt, können zudem Probleme wie Mehr- und Doppeldeutigkeiten auftreten.

		Der Nutzer-Standort muss aber nicht zwangsweise Werte mit geografischem Bezug enthalten. 

		Zudem gibt der Nutzer-Standort nicht unbedingt den Ort an, von dem der Tweet versendet wurde. 
		Tweets mit geografischen Koordinaten werden zumeist von mobilen Endgeräten versendet. 
		Der Nutzer muss sich also zum Zeitpunkt des Absendens eines Tweets nicht an dem im Nutzer-Standort angegeben Ort aufhalten.

		Der Nutzer-Standort ist eine freiwillige Angabe des Twitter-Nutzers im Nutzer-Profil. 
		Von Hecht et al. \cite{Hecht2011} wird der Inhalt der Nutzer-Standorte von 100.000 Nutzer-Profilen manuell analysiert.
	    Ca. 66\% aller analysierten Nutzer-Standorte enthalten demnach einen Wert mit geografischem Bezug.
	    Es könnten somit zusätzlich 66\% der Twitter-Nutzer eine geografische Position zugeordnet. 

	    Der Nutzer-Standort bietet somit ein großes Potenzial, um weiteren Tweets eine geografische Position zuzuordnen.
	    Um eine Geolokalisierung von Tweets mit Hilfe des Nutzer-Standortes realisieren zu können, müssen die oben genannten Probleme bezüglich der angegebenen Werte im Nutzer-Standort gelöst werden.
	    Zusätzlich muss verifiziert werden, dass der Nutzer-Standort den Ort des Absendens eines Tweets hinreichend genau beschreibt.

	    Abhängig von der Anwendung sind zudem die Anforderungen bezüglich der Genauigkeit, Trefferquote und der gewünschten Zuordnung des Ergebnisses zu einer geografischen Region unterschiedlich. 

	    \newpage

	  \section{Fragestellungen}

	  	Aus der Problembeschreibung ergeben sich folgende Fragestellungen:  

	  	\begin{itemize}
	  		\item Wie können die oben genannten Probleme, der eingegebenen Werte im Nutzer-Standort, weitestgehend eliminiert werden?
	  		\item Wie genau kann aus dem Nutzer-Standort die Position, von der ein Tweet abgesendet wurde, bestimmt werden? 
			\item Ist es möglich, die Ergebnisse bezüglich der Qulität der Ergebnisse zu justieren?
			\item Ausnutzen einer geografischen Hierarchie zur besseren Bestimmung des geografischen Standortes? 
	  	\end{itemize}

	\section{Zielsetzung} \label{sec:Zielsetzung} 

		Das übergeordnete Ziel dieser Arbeit besteht darin, Tweets mit Hilfe des angegebenen Nutzer-Standortes einer geografischen Position zuzuordnen.
		Dadurch soll die Position, von der ein Tweet versendet wurde, möglichst genau bestimmt werden.

		Es soll dazu ein Verfahren zur Geolokalisierung von Tweets entwickelt werden, welches die Probleme der angegebenen Werte im Nutzer-Standort so weit wie möglich eliminiert.
		Es sollen Vorgaben bezüglich der Güte und der Trefferquote der Ergebnisse gemacht werden können.
		Das Verfahren soll abhängig von dieser Vorgabe justiert werden können.

		Die Zuordnung von geografischen Positionen erfolgt in der Regel durch geografische Koordinaten.
		Exakte geografische Positionen in Form von Lägen- und Breitengrad sind aber nicht immer erforderlich. 
		Für einige Anwendungen sind zwar exakte geografische Zuordnungen notwendig, vielfach ist jedoch eine flächige Zuordnung ausreichend. 
		Es soll dazu eine geeignete Einteilung des Globus in geografische Regionen mit unterschiedlichen Hierarchieebenen vorgenommen werden.
		Für jede Hierarchieebene soll eine separate Evaluierung der Ergebnisse möglich sein.

		Twitter wird weltweit genutzt, das Verfahren soll deshalb auch unabhängig von unterschiedlichen Sprachen und Schriftzeichen funktionieren.
	
	\section{Gliederung der Arbeit}

			In Kapitel 2 sollen die Grundlagen für die entwickelte Methode vermittelt werden. 
			Es wird auf den Mikroblogging-Dienst Twitter eingegangen und es werden grundsätzliche Methoden und Verfahren vorgestellt welche zum Verständnis des entwickelten Verfahrens benötigt werden. 
			Ebenso werden häufig genutzte geografische Grundbegriffe eingeführt.
			In Kapitel 3 werden verwandte Arbeiten betrachtet, eingeordnet und in Bezug auf die angegebenen Anforderungen untersucht.. 
			Kapitel 4  stellt ein Verfahren zur Lösung der Fragestellungen vor. 
			Um einen Überblick zu gewährleisten, wird das Verfahren zunächst allgemein betrachtet, danach wird jeder Verfahrensschritt dargelegt.
			Das Verfahren läuft in zwei Schritten, einer Lernphase und der eigentlichen Geolokalisierung. 
			Im Kapitel 4, Evaluierung, werden die Ergebnisse der Referenzimplementierung bewertet.
			Es werden die besten erreichten Ergebnisse dargestellt und interpretiert. 
			Des weiteren werden verschiedene Einstellungen für Schwellwerte zur Justierung des Verfahrens bezüglich der Qualität getestete.
			In Kapitel 5 wird das Verfahren und die Ergebnisse zusammengefasst. 
			Im Ausblick werden mögliche Verbesserungen und Ideen zur Weiterentwicklung gegeben.  