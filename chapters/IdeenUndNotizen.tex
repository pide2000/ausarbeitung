\chapter{Ideen und Notizen}

	\section{Stakeholder analyse}
	Welche potenziellen Stakeholder profitieren von der Arbeit? 
	Was benötigt jeder dieser Stakeholder? Bedürfnisse analysieren und Begründen.  

	\begin{enumerate}
		\item Marketing Professionals
		\item Statistiker allgemein
		\item Sozialwissenschaftler -> Analyse von Informationsströmen
	\end{enumerate}



	\section{Ideen}


	\begin{enumerate}
		\item \todo{In Einleitung} Voraussetzungen zur Anwendung des Verfahrens
		\begin{enumerate}
			\item Lerndaten mit konkreten geografischen Angaben
			\item Indikatoren in Lerndaten, welche auch in Datensätzen ohne konkrete geografische Angaben vorkommen (hier eventuelle Diskrepanzen zwischen geogetaggten und nicht geogetaggten tweets + Mentalität in bestimmten Ländern)
			\item Indikatoren mit geografischem Bezug, oder hinreichendem geografischen Bezug, Mittelbar oder unmittelbar
		 \end{enumerate}
		 \item Auf Jargon Namen für Städte eingehen, wie bspsw. the big apple -> New York City 
		 \item Landesgrenzen-Problematik wird durch meine Lösung obsolet -> auf stakeholder eingehen
		 \item \todo{Korrelation zwischen Lokalisierungungssicherheit und tatsächlichem Match berechnen} Wahrscheinlichkeiten für korrekte Lokalisierung kann angegeben und justiert werden 
		 \item Wenn Wahrscheinlichkeiten auf best. Ebene nicht hoch genug dann verschieben auf Admin2 -> Admin1 -> Länderebene
		 \item mit vorherigem werden Unsicherheiten bei Lokalisierung abgebildet (Wichtig für Informationsflüsse) 
		 \item  
	\end{enumerate}
	  