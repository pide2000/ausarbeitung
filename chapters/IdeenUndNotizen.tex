%!TEX root = ../document.tex
\chapter{Ideen und Notizen}

	\section{Stakeholder analyse}
	Welche potenziellen Stakeholder profitieren von der Arbeit? 
	Was benötigt jeder dieser Stakeholder? Bedürfnisse analysieren und Begründen.  

	\begin{enumerate}
		\item Marketing Professionals
		\item Statistiker allgemein
		\item Sozialwissenschaftler -> Analyse von Informationsströmen
	\end{enumerate}



	\section{Ideen}


	\begin{enumerate}
		\item \todo{In Einleitung} Voraussetzungen zur Anwendung des Verfahrens
		\begin{enumerate}
			\item Lerndaten mit konkreten geografischen Angaben
			\item Indikatoren in Lerndaten, welche auch in Datensätzen ohne konkrete geografische Angaben vorkommen (hier eventuelle Diskrepanzen zwischen geogetaggten und nicht geogetaggten tweets + Mentalität in bestimmten Ländern)
			\item Indikatoren mit geografischem Bezug, oder hinreichendem geografischen Bezug, Mittelbar oder unmittelbar
		 \end{enumerate}
		 \item Auf Jargon Namen für Städte eingehen, wie bspsw. the big apple -> New York City 
		 \item Landesgrenzen-Problematik wird durch meine Lösung obsolet -> auf stakeholder eingehen
		 \item \todo{Korrelation zwischen Lokalisierungungssicherheit und tatsächlichem Match berechnen} Wahrscheinlichkeiten für korrekte Lokalisierung kann angegeben und justiert werden 
		 \item Wenn Wahrscheinlichkeiten auf best. Ebene nicht hoch genug dann verschieben auf Admin2 -> Admin1 -> Länderebene
		 \item mit vorherigem werden Unsicherheiten bei Lokalisierung abgebildet (Wichtig für Informationsflüsse) 
		 \item  
	\end{enumerate}
	  


	  \section{Formulierungen}
	  	\subsection{unmittelbare ungesicherte geografische Indikatoren}
	  		Das "'userlocation"' Feld in einem Tweet kann durchaus eine konkrete Lokation beinhalten, jedoch wird auch oft irgendetwas eingetragen. \cite{Hecht2011}
	  		Es kann sich dabei um beliebige Wörter oder Sätze handeln, die einzige Limitierung ist die Anzahl zur Verfügung stehender Zeichen. Nichtsdestotrotz ist es das Ziel dieses Feldes seinen eigene Standort anzugeben.  Dabei kann allerdings nicht davon ausgegangen werde, das der eingetragene Wert nicht doch in einem Zusammenhang mit einer geografischen Lokation steht. 
	  		Bezeichnungen von Städten in Umgangssprache wie besipielsweise "The Big Apple" für New York City oder Motown für Detroit, sind für einige Personen nicht unmittelbar zuzuordenen, geben allerdiings eine konkrete Lokation an. Da die Masse an Bei bzw. Spitznamen für Städte nicht überschaubar ist und auch sprachliche Probleme bestehen ist es sinnvoll alle userlocation Einträge gleich zu behandeln und diese in erster Linie als Lokationsangaben zu behandeln. Durch die Einschränkung auf eine Geolocation werden einzelne gleich lautende Einträge, welche aber nicht auf einen konkreten Ort hinweisen in einzelnen Datensätzen abgelegt. 