%!TEX root = ../document.tex
\chapter{Verwandte Arbeiten} \label{chap:VerwandteArbeiten} 
	Durch das aufkommen der nutzergenerierte 
	Die Geolokalisierung von Tweets ist ein Feld an dem nach wie vor aktiv geforscht wird.



	An die Geolokalsierung werden dabei gewisse Mindestanforderungen bezüglich der Genauigkeit und der Trefferquote gestellt.
	Die entwickelten Verfahren unterscheiden sich sowohl durch die verwendete Methode als auch durch den Untersuchungsgegenstand.  
	
	In diesem Abschnitt sollen bestehende Ansätze zur Geolokalisierung im Twitter-Umfeld untersucht werden. 
	Es werden Kriterien zur Einordnung der bestehenden Ansätze erarbeitet und erläutert.   
	Die Arbeiten werden mit Hilfe der Kriterien schematisch eingeordnet um einen Überblick zu erhalten. 

		\section{Kategorisierung bestehender Ansätze}

		In früheren Arbeiten wurde bereits versucht, eine Einordnung der bestehenden Verfahren vorzunehmen. 
		Es ist interessant die Kategorisierungsansätze und die verwandten Arbeiten einiger Autoren zu studieren.
		Es lässt sich dadurch die Entwicklung zum Thema Lokalisierung im Twitter-Umfeld beobachten. 
		Einige Kategorisierungsansätze werden im folgenden aufgelistet und erläutert.

		Sowohl in \cite{Hecht2011} als in \cite{Cheng2010} beschränken sich die verwandten Arbeiten nicht auf die Lokalisierung im Twitter-Umfeld, es werden Arbeiten zur Lokalisierung von Web-Inhalten im Allgemeinen aufgelistet. 
		Dies lässt darauf schliessen, dass sich vor den Jahren 2010/2011 nur wenige Arbeiten mit der Lokalisierung im Twitter-Umfeld beschäftigt haben.  
		
		\subsubsection{Kategorisierung über die untersuchte Ressource}
		
			\cite{Hecht2011} nimmt deshalb eine Kategorisierung anhand der untersuchten Ressource vor. 
			Es wird unterschieden zwischen Forschungen zur "'Lokalisierung von Microblogging-Seiten und deren Inhalten"' und der "'Lokalisierung von Nutzern, welche Inhalte zu Web 2.0 Seiten beisteuern"'. 
			Zusätzlich wird in dieser Arbeit das "'Verhalten der Nutzer im Umgang mit der Veröffentlichung ihres aktuellen Standorts"' und die "'Vorhersage privater Informationen"' betrachtet. Darauf soll hier allerdings nicht weiter eingegangen werden.      

		\subsubsection{Kategorisierung über die verwendete Methode}

			\cite{Cheng2010} klassifiziert die vorgestellten Arbeiten anhand der verwendeteten Methodik. 
			Es wird auf Arbeiten zur Lokalisierung von Webseiten, Web-Logs, Suchanfragen und Web-Nutzern verwiesen. 
			Diese werden in die folgenden drei Kategorien eingeteilt.

		\paragraph*{"'Inhaltsanalyse mit Begriffen in einem geografischen Verzeichnis (Content analysis with terms in a gazetteer)"'}  
	
			Es wird darunter eine einfache Datenbanksuche verstanden. 
			Es werden einzelne Wörter in einer Datenbank nachgeschlagen um diese einem konkreten geografischen Ort zuweisen zu können.
			Dabei kann sowohl lokal auf eine Geo-Datenbank als auch auf Internet Ressourcen zurückgegriffen werden.  
			In der Regel durchläuft der untersuchte Text eine manuelle oder automatische Vorverarbeitung um potenziell geografische Begriffe, sogenannte Toponyme, herauszufiltern. 

		\paragraph*{"'Inhaltsanalyse mit probabilistischen Sprachmodellen (Content analysis with probabilistic language models)"'}
		
			Dabei werden Texte oder Textteile einer Twitter-Kurznachricht zu vordefinierten geografischen Regionen wie Ländern oder Städten zugeordnet. 
			Nach einer Vorverarbeitung des Textes erfolgt eine statistische Auswertung, um danach den Text oder einzelne Textteile, wie beispielsweise Wörter, einer geografischen Region zuzuordnen. 
			Eine unbekannter Text kann dann mit Hilfe der zuvor gelernten Zuordnung einer geografischen Region zugeordnet werden.

		\paragraph*{"'Schlussfolgerungen durch soziale Verbindungen (Inference via social relations)"'} 

			Es werden soziale Verbindungen, die in Netzwerken abgebildet sind, herangezogen um Rückschlüsse auf den geografischen Ort des untersuchten Inhaltes oder einer Person ziehen zu können.

		Preidhorsky et al. schlagen in \cite{Priedhorsky2013} eine weitere Einteilung anhand der Methodik vor. 
		Allerdings werden hier ausschließlich Arbeiten im Twitter-Umfeld betrachtet. 

		\paragraph*{"'Geocoding"'} Im wesentlichen entspricht dies der "'Inhaltsanalyse mit Begriffen in einem geografischen Verzeichnis"' aus \cite{Cheng2010}. 
		"'Geocoding"' wird als Begriff in vielen Fachrichtungen unterschiedlich definiert, was zu Missverständnissen führen kann. 
		In \cite{bibsmaniaaa:Goldberg2008} wird genauer auf den Begriff des Geocoding und die Poblematik eingegangen und eine Definition  des Begriffs vorgeschlagen.
		Im vorliegenden Kontext ist es präziser und weniger missverständlich die Methodik als "'Inhaltsanalyse mit Begriffen in einem geografischen Verzeichnis"' zu bezeichnen, anstatt den Begriff "'Geocoding"' einzusetzen. 
		
		\paragraph*{"'Geografische Themenmodelle (geografic Topic Modeling)"'} wird definiert als die Verbindung von "'Themenmodellierung"' und "'Standorterkennung (Location Awareness)"'. 
		Durch klassisches "'Themenmodellierung"' lässt sich aus aus Texten eine Menge von Themen extrahieren. 
		Durch eine Lernphase werden Wörterbücher zu den Themen erstellt.
		Mit Hilfe dieser Themen-Wörterbücher kann später das Thema eines Textes bestimmt werden. \cite{Blei2012} 
		Unter "'Standorterkennung"' wird hier verstanden, dass nicht nur das Thema sondern auch eine bestimmte Region extrahiert werden kann. 
		Dies kann durch geografischen Koordinaten in Twitter-Kurznachrichten realisiert werden. 
		Im Unterschied zur Kategorie "'Inhaltsanalyse mit probabilsitischen Sprachmodellen"' aus \cite{Cheng2010} wird hier jedoch keine vorgegebene geografische Region gefordert. 
		Vielmehr ergeben sich die geografischen Regionen aus den Themenmodellen und den zugehörigen geografischen Koordinaten.
		Es wird damit eine kontinuierliche Region beschrieben, welche nicht zwangsweise durch Stadt-, Staaten- oder Ländergrenzen beschränkt ist.  

		\paragraph*{"'Statistische Klassifizierung (Statistical classifiers)"'} Diese Kategorie entspricht der "'Inhaltsanalyse mit probabilsitischen Sprachmodellen"' wobei in \cite{Cheng2010} nur eine Arbeit in dieser Kategorie betrachtet wird. \cite{Priedhorsky2013} listet mehrere Arbeiten auf, die sich in diese Kategorie einordnen lassen.   

		\paragraph*{"'Informationen aus sozialen Verbindungen (Social Network Information)"'} analog zu "'Schlussfolgerungen durch soziale Verbindungen"' aus \cite{Cheng2010} werden soziale Verbindungen herangezogen um den Standort zu bestimmen. 

		Priedhorsky et al. wählen eine ähnliche Einteilung wie vormals Cheng et al. in 2010, die verwandten Arbeiten stammen allerdings aus dem Twitter-Umfeld. 
		Dabei ist zu bemerken, dass sich die verwendeten Methoden zur Lokalisierung im Twitter-Umfeld nicht wesentlich von denen in anderen Bereichen unterscheiden. 
		Um die Arbeiten im Twitter-Umfeld sinnvoll voneinander abgrenzen zu können muss die Kategorisierung mehr Dimensionen umfassen. 
		Es müssen mehr Kriterien zur Kategorisierung herangezogen werden als die reine Methodik.   

		Mahmud et al. betrachten in \cite{Mahmud2012} hauptsächlich Arbeiten im Twitter-Umfeld. 
		Diese werden in die folgenden Kategorien unterteilt. 

		\begin{enumerate}
			\item  "'Inhaltsbasierte Standortschätzung von Tweets (Content-based Location Estimation from Tweets)"'
			\item "'Inhaltsbasierte Standortextrahierung von Tweets (Conetnt-based Location Extraction from Tweets"'
			\item "'Standortschätzung ohne den Tweet Inhalt zu nutzen (Location Estimation without using Tweets Content)"'
		\end{enumerate}

		\paragraph*{"'Inhaltsbasierte Standort-Schätzung von Tweets (Content-based Location Estimation from Tweets)"'} hier wird die geografische Position durch eine Inhaltsanalyse der Twitter-Kurznachricht geschätzt. 
		Die Schätzung erfolgt dabei durch probabilistische Modelle.
		Diese Kategorie vereint damit "'Geografische Themenmodelle"', "'Statistische Klassifizierung"' aus \cite{Priedhorsky2013} mit "'Inhaltsanalyse mit probabilsitischen Sprachmodellen"' aus \cite{Cheng2010} und ist damit als genereller anzusehen, als die vorgenannten Kategorien. 

		\paragraph*{"'Inhaltsbasierte Standort-Extrahierung von Tweets (Content-based Location Extraction from Tweets"'} die verwandten Arbeiten in dieser Kategorie versuchen direkte Hinweise auf einen geografischen Ort aus einer Twitter-Kurznachricht zu extrahieren. 
		Diese Kategorie ähnelt dem "'Geocoding"' beziehungsweise der "'Inhaltsanalyse mit Begriffen in einem geografischen Verzeichnis"'. 

		\paragraph*{"'Standortschätzung ohne den Tweet Inhalt zu nutzen (Location Estimation without using Tweets Content)"'} hierunter versteht der Autor alle Informationen die nicht unmittelbar im Tweet-Text enthalten sind. Dazu zählen Informationen aus dem Nutzerprofil oder Informationen über die sozialen Verbindungen des Nutzers.


		\cite{Mahmud2012} nutzt ebenfalls die Methodik um die Arbeiten zu kategorisieren. 
		Allerdings wird hier eine generellere Einteilung vorgenommen. 
		So wird unterteilt, ob der Standort geschätzt oder extrahiert wurde.  
		Mahmud et al. bringen aber auch eine weitere Dimension ein. 
		Es wird hier zusätzlich unterschieden ob das angewendete Verfahren den Tweet-Inhalt nutzt oder andere Informationen. 

		Dies ist sinnvoll, denn die genannten Methoden lassen sich sowohl auf den Tweet-Inhalt als auch auf andere Informationen, beispielsweise aus dem Nutzerprofil, anwenden. 
		
		Frühere Arbeiten verweisen auf ein weiteres Spektrum an Arbeiten aus anderen Bereichen, wie Lokalisierung von Flickr Bildern oder Web-Log Einträgen.
		Arbeiten zur Lokalisierung im Twitter-Umfeld werden hier seltener erwähnt. 
		In späteren Arbeiten, wie in \cite{Priedhorsky2013}, wird hingegen fast ausschließlich auf Arbeiten aus dem Twitter-Umfeld verwiesen. 
		Dies spiegelt die steigende Anzahl der Arbeiten zur Lokalisierung im Twitter-Umfeld wieder.
		Betrachtet man die Ausarbeitungen zur Lokalisierung im Twitter-Umfeld genauer, wird allerdings schnell klar, dass die Kategorisierung der Arbeiten anhand der verwendeten Methodik, dem Umfang nicht mehr gerecht wird. 
		
		Bei genauerer Betrachtung der Arbeiten stellt man allerdings fest, dass diese Klassifizierungen dem Umfang der Arbeiten nicht gerecht wird. 
		\cite{Hecht2011} verweist auf ähnliche Ansätze mit einem anderen Untersuchungsgegenstand.
		\cite{Cheng2010} kategorisiert die Arbeiten anhand der Methodik, und verweist ebenso auf andere Untersuchungsgegenstände. 
		\cite{Priedhorsky2013} verweist ausschliesslich auf Arbeiten im Twitter-Umfeld und kategorisiert diese anhand der verwendeten Methodik. 
		Die Methodeneinteilung ist aufgrund der Begriffswahl missverständlich und kann somit zu Problemen führen. 

		\subsection{Erweiterter Kategorisierungsansatz}
		In \cite{Schulz2013} werden die folgenden Dimensionen zur Abgrenzung herangezogen.

		\todo{Indikatoren aus \cite{Schulz2013}}

		Allerdings lassen sich noch andere Dimensionen zur Klassifizierung der Arbeiten heranziehen. 
		Wird besipielsweise der Text einer Twitter-Kurznachricht durch eine einfache Geokodierung untersucht wird dies andere Ergebnisse liefern als eine Untersuchung auf Basis eines geografischen Themenmodells.  
		

		\cite{Hecht2011} nutzen diese Methode um eine Ground-Truth zu bestimmen indem das Userlocation-Feld in Wikipedia nachschlagen wird. Wikipedia bietet zu vielen Artikeln eine grografische Position in Form von Längen- und Breitengrad an, diese werden dann der untersuchten Twitter-Kurznachricht zugeordnet. 
		\cite{Hale2012} nutzen die Yahoo und die Google Geocoding Api um das Userlocation-Feld eingehender zu untersuchen.  
		

		
		Eine weitere zu betrachtende Dimension stellt daher der konkrete Untersuchungsgegenstand in Form des Indikators dar.


		Betrachtet man die Gesamtheit an arbeiten im Bereich der Lokalisierung im Twitter Netzwerk drängen sich noch mehr Dimensionen zur Klassifizeirung der arbeiten auf.

		\begin{enumerate}
		 	\item Räumliche Indikatoren
		 	\item Techniken
		 	\item Fokus der Lokalisierung
		 \end{enumerate} 	


		\todo{Tabelle einfügen, bereits fertig, nur noch Format anpassen (Lesbarkeit)}

	

	\section{}

		Durch die Kategorisierung wird deutlich, dass auf dem Nutzer-Standort in Kombination mit der Zeitzone bisher keine NLP ausgeführt wurde. 
		Es ist sinnvoll NLP auf den vorgenannten feldern auszuführen, da die Inhalte ähnlich dynamisch sein können wie im Text Feld.
		Der Nutzer-Standort bietet den Vorteil, dass die Intention ist, dass ein Toponym eingegeben. 


		Es wird selten eine geografische Hierarchie zum abbilden der Positionen verwendet.




	\section{GAZZETEER UND NACHTEILE}

		Bei einer Abfrage des Toponyms "'Hamburg"' auf ein Ortsverzeichnis, wie geonames.org, würden somit mindestens 31 Georeferenzen als Ergebnis erscheinen.
		Tatsächlich liefert geonames.org folgendes Ergebnis:

		\begin{table}[htpb]
			\caption{Abfrage an geonames.org für Hamburg} 
			\centering
			\begin{tabular}{|c|c|}
				\hline
				Land & Anzahl\\
				\hline\hline
				USA & 29 \\
				\hline
				Süd Afrika & 2 \\
				\hline
				Deutschland & 1 \\
				\hline
				Kanada & 1 \\
				\hline
				Surinam & 1 \\
				\hline
			\end{tabular}
			\label{tab:usCitiesGermanNames} 
		\end{table} 

		Dies basiert auf Mehrdeutigkeiten aus dem Ergebnis muss eine Auswahl getroffen werden ohne zusätzliches Wissen über den Ort zu haben getroffen werden.


		Tatsächlich beinhaltet die geonames.org Datenbank keinen dieser Bei- und Spitznamen.
		Durch eine Abfrage an dieses Ortsverzeichnis könnte somit keine Georeferenz bestimmt werden.
		Eine Abfrage in Google Maps hingegen bietet bei Eingabe der oben genannten Bei- und Spitznamen Detroit als Vorschlag an. \footnote{http://de.wikipedia.org/wiki/Detroit (abgerufen Juli 2014)} 


		\textit{Probleme früherer Ansätze auf Gazetteer Basis, es wird für ein Land kein Polygon zurückgegeben sondern ein konkreter Punkt. 
		Messung der Genauigkeit bei Evaluierung auf Länderebene nicht möglich.  
		Ich kann das tun da ich alle test tweets auf eine Stadt abbilde, somit zusätzliche Information über Stadt, Adm1 usw bekomme und somit evaluieren kann ob sich ein Tweet in einem Land befindet, denn auch die ergebnisse haben implizit auch alle anderen geografsichen Hierarchieebenen.
		}




	