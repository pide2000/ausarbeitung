%!TEX root = ../document.tex
\chapter{Lösungsansatz} \label{chp:Loesungsansatz}
In diesem Kapitel wird ein Verfahren zur Lokalisierung von Twitter-Nutzern vorgestellt, welches die Fragestellungen aus \ref{sec:fragestellung} beantwortete.
Es werden dabei die Anforderungen aus Kapitel \ref{sec:Anforderungen} berücksichtigt.
Zunächst soll das Verfahren und die Vorgehensweise allgemein erklärt um dem Leser ein Überblick zu bieten. 
Daraufhin wird die Lösung systematisch von grundauf erarbeitet.    

\section{Verfahen zur Georeferenzierung von Twitter-Nutzern}

	\subsection{Informationen und benötigte Daten}

	\subsection{Verfahrensablauf}


\section{Analyse der Tweet Daten}
	
	

	\subsection{Detailanalyse Standort und Zeitzone}



\section{Indikatoren zur Ortsbestimmung}

	\begin{enumerate}
		\item{Training und Validierungsdaten erklären} 
	\end{enumerate}


	\todo{u.U. "'unmittelbar und mittelbar geografische Indikatoren"' "'Entwurf"' -> "'Grundlagen und Stand der Technik verschieben"'} 
	\subsection{unmittelbar geografische Indikatoren}
		\begin{enumerate}
			\item Mögliche Alternativen
			\item Begründung warum Userlocation und Timezone
			\item Beispiele und Auswertungen (manuell getaggter Datensatz)
			\item \cite{Hecht2011} 
		\end{enumerate}

	\subsection{mittelbar geografische Indikatoren}
		\begin{enumerate}
			\item bspsw. Hashtags, Inhaltsanalysen ohne spezielle geografische Hinweise, 
		\end{enumerate}

	
	\subsection{Vorverarbeitung der Indikatoren (Präprozessor-Konzept)}
		\begin{enumerate}
			\item geonames matching (geonames tree) für geografische Namen bestehend aus mehreren Wörtern
			\item Eliminierung von Sonderzeichen
			\item Tokenizing
			\item Ngram Erzeugung allgemein
			\item \todo{Was bringt die Zeitzone als zusätzlicher Indikator? Verbesserun messen} Zeitzone als "'schärfenden Indikator für doppeldeutige Namen"'
		\end{enumerate}

	\subsection{Encoding}
		Problematik unterschiedlicher Sprachen, 
		url-encoding sinnvoll als Vorbereitung auf Webservice. 

\section{Geolocation Mapping}

	\subsection{nearest neighbour mapping}
		\begin{enumerate}
			\item \todo{Welches Fehlermaß kann für das mapping angewandt werden? auf Städteebene gut möglich mit geografischen Distanzen, admin2,amdin1, Land schlecht möglich mit Distanzen} Wie genau kann gemappt werden? Fehler Durchschnitt. 
			\item Mapping auf cities 1000/1000/15000 mit Daten zu durchschnitll. Abstand
			\item Hier ist noch Verbesserungspotenzial -> wenn Mapping Distanz zu weit entfernt -> verwerfen! 
		\end{enumerate} 

\section{Verknüpfung von Indikatoren und geografischen Lokationen zur wiedergewinnung des erlernten Wissens}
	
	\subsection{Generierung eines Wissendatensatzes}

	\subsection{Verknüpfung mit Geodaten}

	\subsection{Auflösen auf Administartionsebenen, Länder}

\section{Lokalisieren von Tweets ohne konkrete geografische Daten}
	
	\subsection{Ablauf der Lokalisierung}
	
	\subsection{Lokalisierungssicherheit durch Ausnutzung der geografischen Hierarchiebeziehungen}


\section*{einbauen!!!}
\subsection{Geografische Grundbegriffe und Geografiedaten}

		\subsubsection{Geografische Grundbegriffe}

		\subsubsection{Geonames.org} \footnote{eventuell erst in Implemetierung darauf eingehen} 
			

		\subsubsection{}	

	\subsection{???} 
		\subsubsection{N-Gramme}
			\begin{enumerate}
				\item NGramme allgemein, Verwendung, Beispiele. 
				\item \todo{NGramme -> Nochmal genau prüfen, Zusammenhang zu Markov Modell und NGram Statistik herausstellen} Zusammenhang zwischen Länge/Grad eines N-Grammes und Wahrscheinlichkeiten. -> mathematische Herleitung?!
			\end{enumerate}





