%!TEX root = ../document.tex
\chapter{Lösungsansatz} \label{chp:Loesungsansatz}
In diesem Kapitel wird ein Verfahren zur Lokalisierung von Twitter-Nutzern vorgestellt.
Die Fragestellungen aus Kapitel \ref{sec:fragestellung} werden, unter Berücksichtigung der Anforderungen aus Kapitel \ref{sec:Anforderungen}, beantwortet.  

Zunächst wird ein Überblick über die Funktion und den Ablauf des Verfahrens gegeben ohne detailliert auf die einzelnen Verfahrensschritte einzugehen.
Im darauffolgenden Kapitel wird das Verfahen dann von Grundauf betrachtet und die einzelnen Verfahrensschritte eingehender erklärt. 


	\section{Überblick}
	Das erarbeitete Verfahren soll es ermöglichen Twitter-Nutzern, deren geografische Position unbekannt ist, eine Georeferenz zuzuordnen.
	Um diese Georeferenzierung zu ermöglichen sollen Indikatoren aus den Twitter-Profilen verwendet werden. 
	Als geografische Indikatoren werden der Nutzer-Standort und die Nutzer-Zeitzone herangezogen.

	Das Verfahren zur Georeferenzierung kann dabei in zwei Phasen eingeteilt werden.
	In einer ersten Phase wird eine Datenbasis generiert, diese Phase wird im folgenden Lernphase genannt. 
	Die zweite Phase ist die eigentliche Georeferenzierung, welche die zuvor erzeugte Datenbasis nutzt um Nutzern, deren geografsiche Position unbekannt ist, eine Georeferenz zuzuordnen. 

	\subsection{Lernphase}
	In dieser Phase wird eine Datenbasis erzeugt die eine spätere Georeferenzierung ermöglicht. 
	Um diese Datenbasis erstellen zu können muss eine umfangreiche Sammlung an Tweets vorliegen.
	Diese Tweet-Sammlung wurde über die Twitter-Streaming-API generiert und wird im folgenden als Lerndaten bezeichnet. \footnot{Siehe Kapitel \ref{chp:Einleitung} genutzte Daten} 
	 
	Des weiteren wurden in den Datensatz nur Tweets aufgenommen, welche über eine Georeferenz verfügen. 
	In der Lernphase wird der Nutzer-Standort, die Nutzer-Zeitzone sowie die Georeferenz jedes einzelnen Tweets verarbeitet um eine Datenbasis zur Georeferenzierung zu generieren. 
	Der Nutzer-Standort wird dabei einer mehrstufigen Vorverarbeitung unterzogen.
	In den einzelnen Stufen der Vorverarbeitung werden Sonderzeichen entfernt, alle Zeichen in Kleinbuchstaben umgewandelt, in der Wortfolge werden nach geografischen Begriffen gesucht die zwei oder mehr Worte beinhalten, die Zeichen werden url-encoded und die einzlenen Wörter werden in Tokens zerlegt welche alphanumerisch sortiert werden.
	Im nächsten Schritt werden aus den Tokens Uni-Gramme, Bi-Gramme und Tri-Gramme erstellt. 
	Jedes dieser so entstandenen N-Gramme wird die Zeitzone angehängt.  

	 

	Das Ergebnis dieser Phase ist eine Datenbasis welche eine Reihe von Zeichenketten aus der Nutzer-Standort und der Nutzer-Zeitzone generiert haben. 
	Jeder dieser Zeicheneketten wird die Häufigkeit ihres Vorkommens und die nächste Stadt mit über 15.000 Einwohnern zugeordnet. 


	\subsection{Georeferenzierung}  

	In der zweiten Phase kann mit Hilfe der Datenbasis die eigentliche Georeferenzierung von Twitter-Nutzern, mit unbekannter geografischer Position, durchgeführt werden. 

	Es wird dem Anwedner ermöglicht die Genauigkeit über die geografischen Hierarchien anzugeben und einen Konfidenzschwellwert zu bestimmen.


	Bei der Georeferenzierung kann bestimmt werden wie genau die geografische Position oder die geografische Region bestimmt werden soll. 
	Dabei kann zwischen den bereits in Kapitel \ref{par: geografische Hierarchie} erläuterten Hierarchieebenen gewählt werden.
	Dies hat den Vorteil, dass die Genauigkeit der Georeferenzierung den jeweiligen Anforderungen angepasst werden kann.
	Des weiteren kann eine Konfidenz angegeben werden, desto höher die Konfidenzschwelle gewählt wird desto sicherer ist die Georeferenzierung. 







	\subsection{Informationen und benötigte Daten}

	\subsection{Verfahrensablauf}


\section{Analyse der Tweet Daten}

\section{Indikatoren zur Ortsbestimmung}

	\begin{enumerate}
		\item{Training und Validierungsdaten erklären} 
	\end{enumerate}


	\todo{u.U. "'unmittelbar und mittelbar geografische Indikatoren"' "'Entwurf"' -> "'Grundlagen und Stand der Technik verschieben"'} 
	\subsection{unmittelbar geografische Indikatoren}
		\begin{enumerate}
			\item Mögliche Alternativen
			\item Begründung warum Userlocation und Timezone
			\item Beispiele und Auswertungen (manuell getaggter Datensatz)
			\item \cite{Hecht2011} 
		\end{enumerate}

	\subsection{mittelbar geografische Indikatoren}
		\begin{enumerate}
			\item bspsw. Hashtags, Inhaltsanalysen ohne spezielle geografische Hinweise, 
		\end{enumerate}

	
	\subsection{Vorverarbeitung der Indikatoren (Präprozessor-Konzept)}
		\begin{enumerate}
			\item geonames matching (geonames tree) für geografische Namen bestehend aus mehreren Wörtern
			\item Eliminierung von Sonderzeichen
			\item Tokenizing
			\item Ngram Erzeugung allgemein
			\item \todo{Was bringt die Zeitzone als zusätzlicher Indikator? Verbesserun messen} Zeitzone als "'schärfenden Indikator für doppeldeutige Namen"'
		\end{enumerate}

	\subsection{Encoding}
		Problematik unterschiedlicher Sprachen, 
		url-encoding sinnvoll als Vorbereitung auf Webservice. 

\section{Geolocation Mapping}

	\subsection{nearest neighbour mapping}
		\begin{enumerate}
			\item \todo{Welches Fehlermaß kann für das mapping angewandt werden? auf Städteebene gut möglich mit geografischen Distanzen, admin2,amdin1, Land schlecht möglich mit Distanzen} Wie genau kann gemappt werden? Fehler Durchschnitt. 
			\item Mapping auf cities 1000/1000/15000 mit Daten zu durchschnitll. Abstand
			\item Hier ist noch Verbesserungspotenzial -> wenn Mapping Distanz zu weit entfernt -> verwerfen! 
		\end{enumerate} 

\section{Verknüpfung von Indikatoren und geografischen Lokationen zur wiedergewinnung des erlernten Wissens}
	
	\subsection{Generierung eines Wissendatensatzes}

	\subsection{Verknüpfung mit Geodaten}

	\subsection{Auflösen auf Administartionsebenen, Länder}

\section{Lokalisieren von Tweets ohne konkrete geografische Daten}
	
	\subsection{Ablauf der Lokalisierung}
	
	\subsection{Lokalisierungssicherheit durch Ausnutzung der geografischen Hierarchiebeziehungen}


\section*{einbauen!!!}
\subsection{Geografische Grundbegriffe und Geografiedaten}

		\subsubsection{Geografische Grundbegriffe}

		\subsubsection{Geonames.org} \footnote{eventuell erst in Implemetierung darauf eingehen} 
			

		\subsubsection{}	

	\subsection{???} 
		\subsubsection{N-Gramme}
			\begin{enumerate}
				\item NGramme allgemein, Verwendung, Beispiele. 
				\item \todo{NGramme -> Nochmal genau prüfen, Zusammenhang zu Markov Modell und NGram Statistik herausstellen} Zusammenhang zwischen Länge/Grad eines N-Grammes und Wahrscheinlichkeiten. -> mathematische Herleitung?!
			\end{enumerate}




\section{Einbauen in dieses Kapitel}

Die folgenden Vorverarbeitungsschritte werden durchgeführt.

	\begin{enumerate}
		\item Eliminierung von Sonderzeichen
		\item ausschließlich Kleinschreibung
		\item geografische Zuordnung von Teilworten \todo{Bessere Umschreibung finden für Teilworte, es ist eig. eine Menge von Worten aus Nutzer-Standort} 	
		\item Tokenisierung
		\item html Encoding
	\end{enumerate}

	Diese Vorverarbeitung dient der Bereiningung und Vereinheitlichung des Nutzer-Standorts.  
	Die gegografische Zuordnung dient der  
