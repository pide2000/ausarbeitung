%!TEX root = ../document.tex
\chapter{Lösungsansatz} \label{chp:Loesungsansatz}
	
	\section{Überblick} 

		Die Bestimmung der geografischen Position, von der ein Tweet abgesetzt wurde, soll durch die Auswertung des Nutzer-Standortes erfolgen. 
		Das Wissen zur Auflösung des Nutzer-Standortes soll dabei durch die Analyse von Tweets aufgebaut werden.
		
		Der Lösungsansatz besteht aus den folgenden zwei Teilen:

		\begin{enumerate}
			\item Training: Lernverfahren zur Erzeugung einer Wissensdatenbank 
			\item Geolokalisierung: Auflösen des Nutzer-Standortes durch die Verarbeitung der Informationen aus der Wissensdatenbank
		\end{enumerate}

		Beim Training soll aus einer Menge an Trainingsdaten eine Wissensdatenbank (im folgenden Georeferenz-Basis) generiert werden.
		Es soll dabei die quantitative Verteilung der Werte des Nutzer-Standortes erfasst werden.
		Um dies leisten zu können werden Trainingsdatensätze benötigt. 
		Jeder Trainingsdatensatz repräsentiert einen Tweet, der den Nutzer-Standort, die Nutzer-Zeitzone und geografische Koordinaten beinhaltet.
		Aus diesen Trainingsdatensätzen kann durch die Untersuchung der Nutzer-Standorte und der zugehörigen geografischen Koordinaten bestimmt werden wie oft ein Wert an einer bestimmten geografischen Position vorkommt.
		Die Nutzer-Zeitzone wird verwendet um Doppel- und Mehrdeutigkeiten bezüglich der Werte im Nutzer-Standort auflösen zu können.  

		Bei der Geolokalisierung soll ein gegebener Nutzer-Standort auf eine Georeferenz aufgelöst werden. 
		Dies geschieht durch eine Abfrage der Werte im Nutzer-Standort an die Georeferenz-Basis.
		Es werden daraufhin Ergebnisdatensätze zurückgeliefert, welche die geografische Verteilung des entsprechenden Wertes widerspiegeln.
		Durch eine Analyse der Verteilungen soll entschieden werden welcher der zurückgelieferten Ergebnisdatensätze am wahrscheinlichsten die korrekte geografische Position, von welcher der Tweet abgesetzt wurde, liefert.
		Die Güte und der Trefferquote der Ergebnisse kann dabei mit Hilfe von Schwellwerten justiert werden.
		Des weiteren kann die Hierarchieebene angegeben werden welcher das Ergebnis zuzuordnen ist.

		In Abbildung \ref{img:einteilungLoesungsansatz} sind die beiden Teile des Verfahrens und deren genereller Ablauf dargestellt.
		In den folgenden Abschnitten wird auf die einzelnen Teile und deren detaillierten Ablauf eingegangen. 
		
		Um die einzelnen Schritte für das Training und die Geolokalisierung spezifizieren zu können soll zunächst der Nutzer-Standort eingehend untersucht werden.
		Dazu wird in einem ersten Abschnitt die Datenbasis vorgestellt.
		Danach wird eine quantitative Bewertung des Nutzer-Standortes vorgenommen bevor die Herausforderungen erläutert werden die bei der Verwendung des Nutzer-Stadortes als geografsicher Indikator bestehen.
		Es wird in Abschnitt \ref{sec:VefrahrenZumEinlernen} basierend auf den Ergebnissen ein Verfahrensablauf zum Training entwickelt. 
		In Abschnitt \ref{sec:AufloesenDesNutzerStandortes} wird die Geolokalisierung, welche die eingelernte Georeferenz-Basis aus dem Training verwendet.
		Dabei werden Schwellwerte zur Justierung der Güte und Trefferquote eingeführt.
		Im letzten Abschnitt \ref{sec:ausnutzenDerGeografischenHierarchie} wird vorgestellt wie die geografische Hierarchie einbezogen wird.


	\section{Verwendete Datenbasis}

		In diesem Abschnitt wird die Datenerhebung und die daraus resultierenden Datensätze betrachtet.
		Zunächst wird aufgezeigt wie die Daten erhoben wurden, und daraus die Trainingsdatensätze sowie Testdatensätze generiert wurden.
		Zudem werden 1000 Tweets per Hand auf die grundsätzliche Eignung als geografischer Indikator untersucht.
		Es werden dabei quantitative Daten zu geografischem Bezug des Nutzer-Standortes erhoben.
		In einem weiteren Abschnitt werden die Herausforderungen betrachtet die bei der Verwendung des Nutzer-Standortes als geografischer Indikator entstehen. 
		Diese sind weitestgehend der freien unkontrollierten Eingabe des Nutzers geschuldet.

		\subsection{Datenerhebung} 

			Die Trainingsdatensätze und die Testdatensätze, die später zur Evaluierung verwendet werden, wurden als zufällige Samples aus einem größeren Datensatz erzeugt.
			Der Datensatz wurde mit Hilfe der Twitter Streaming API erzeugt.
			Die Twitter Streaming API bietet die Möglichkeit ausschließlich Tweets, welche einen Längen- und Breitengrad als Positionsangabe besitzen, abzufragen.
			Bei der näheren Betrachtung hat sich jedoch herausgestellt, dass viele der Tweets für den Längen- und den Breitengrad die Werte (0,0) enthalten. 
			Dies beschreibt eine Position im Golf von Guinea vor der Küste West-Afrikas. 
			Es ist davon auszugehen, dass die 65428 Tweets welche diese Angabe als geografische Koordinaten besitzen nicht von dort versendet wurden.
			Diese wurden aus dem Datensatz entfernt.
			Des weiteren wurden Tweets ohne Werte im Nutzer-Standort entfernt, da diese weder für das Training noch für die Evaluierung genutzt werden können.
			Daraus resultiert eine Basis an Datensätzen mit 383222 Tweets aus dem die Trainingsdatensätze und die Testdatensätze erzeugt werden.  
			Der Testdatensatz besteht aus 20000 zufällig gesampelten Tweets aus diesen Basisdatensätzen. 
			Die restlichen 383222 Tweets werden als Lerndatensätze verwendet. 

				\begin{table}[h]
				\centering
				\caption{Basisdatensätze}
				\label{my-label}
				\begin{tabular}{|l||l|}
				\hline
				Zeitraum & \begin{tabular}[c]{@{}l@{}}23.01.2014 12:00 Uhr\\ bis  04.02.2014 21:00 Uhr\end{tabular} \\ \hline
				Gesamtanzahl Tweets							 & 623645  \\ \hline
				Tweets ohne valide geografische Koordinaten  &	65428    \\ \hline
				Tweets ohne Nutzer-Standort 				 & 169320  \\ \hline
				\begin{tabular}[c]{@{}l@{}}Ohne Nutzer-Stadort und\\ geografische Koordinaten nicht valide\end{tabular} & 65428 \\ \hline
				Basis zur Erzeugung Lerndatensätze und Testdatensätze & 403222 \\ \hline
				Lerndatensätze & 383222 \\ \hline
				Testdatensätze & 20000  \\ \hline
				\end{tabular}
				\end{table}

	\section{Quantitative Bewertung des Nutzer-Standortes}

		In diesem Abschnitt soll untersucht werden in wie vielen Fällen dem Nutzer-Standort geografischer Bezug nachgewiesen werden kann.
		Dabei werden zum Nutzer-Standort quantitative Daten erhoben um die Eignung des Nutzer-Standortes zur Geolokalisierung zu überprüfen.

		\subsection{Methodik zur Untersuchung des Nutzer-Standortes}

			Aus den Basisdatensätzen wurden zufällig 1000 Tweets gewählt. 
			Es wurde eine Oberfläche zur Untersuchung der Nutzer-Standorte dieser 1000 Tweets erstellt.
			Die Oberfläche bietet die Möglichkeit in Google-Maps oder auf dem Ortsverzeichnis von geonames.org nach Toponymen zu suchen.
			Wurde ein Toponym per Suche gefunden wurde dieses dem Datensatz zugeordnet.

			In Abbildung \ref{img:oberflaecheQuantitative} ist die Oberfläche zur Untersuchung von Nutzer-Standorten dargestellt. 
			Der aktuelle Nutzer-Standort wird in der Leiste oben angezeigt, hier Istanbul. 
			Auf der linken Karte wird das Ergebnis von Google-Maps angezeigt.
			Rechts kann im geonames.org Ortsverzeichnis gesucht werden. 
			Die gefundenen Datensätze werden in der Liste rechts angezeigt.  
			In der rechten Karte kann man sich mit einem Klick auf "'in Karte anzeigen"'' den jeweiligen geonames.org Datensatz aus der Liste anzeigen lassen.

			\begin{figure}[!ht]
					\begin{center}
						\includegraphics[scale=0.5]{_oberflaecheQuantitative.pdf}
						\caption{Oberfläche zur Untersuchung des Nutzer-Standortes}
						\label{img:oberflaecheQuantitative}
					\end{center}  
			\end{figure}	

			Die mit dieser Oberfläche erhobenen Daten wurden untersucht. 
			Die Ergebnisse werden in den folgenden Abschnitten vorgestellt.
			Es wurde über Google-Maos und geonames.org versucht den Nutzer-Standort zuzuordnen. 
			Auch wenn die Sprache oder das verwendete Alphabet nicht bekannt waren wurde eine Georeferenz zugeordnet sofern die Suche auf den Ortsverzeichnissen einen Treffer ergab. 


	
			\todo{chap: Lösungsanatz sec: Untersuchung der Werte ---> die nächsten vier Kapitel: Hecht in Grundlagen, meine Untersuchungen hier lassen unter Methodik} 

		\subsection{Geografischer Bezug des Nutzer-Standortes} 
			
			In den eigenen Untersuchungen konnten 76\% der Nutzer-Standorte ein geografischer Bezug nachgewiesen werden. 
			Wurde geografischer Bezug nachgewiesen, so musste der Wert des Nutzer-Standortes im Ortsverzeichnis von geonames.org vorkommen. 
			Daraus kann gefolgert werden, dass in 76\% der Fälle ein Toponym im Nutzer-Standort vorhanden ist.
			
			In den restlichen 24\% der Fälle konnte kein geografischer Bezug mit Hilfe der Ortsverzeichnisse nachgewiesen werden. 
			Dies bedeutet nicht, dass grundsätzlich kein geografischer Bezug vorhanden ist. 
			Es konnte lediglich anhand der genutzten Quellen kein geografischer Bezug hergeleitet werden.
			Beispielsweise wurde "'Swag City"', ein Beiname für die Stadt "'Ann Arbor"' denn der Spitzname für die Stadt war in den Datenbanken nicht hinterlegt. 
			Dies zeigt, dass Werte denen über ein Ortsverzeichnis keine Georeferenz zugewiesen werden konnte, durchaus geografischen Bezug haben können.
			Das verwerfen solcher Werte durch unzureichendes Wissen über Toponyme kann zum Verlust wertvoller Informationen über den tatsächlichen Standort führen.

			Hecht et al. konnten in \cite{Hecht2011} den Datenwerten in den Nutzer-Standorten in 80\% der Fälle einen geografischen Bezug feststellen.
			In den restlichen 20\% der Fälle konnte im Nutzer-Standort kein geografischer Bezug festgestellt werden. 
			Diese Ergebnisse decken sich mit den eigenen Untersuchungen. 
			Es ist allerdings zu beachten, dass Hecht et al. ausschließlich Tweets aus den USA verwendet haben und die Nutzer-Standorte nur in englischer Sprache verfasst waren. 
					
	\section{Herausforderungen bei der Verwendung des Nutzer-Standortes} 

		Hier sollen nun die Werte des Nutzer-Standortes eingehender untersucht werden. 
		Neben den in Kapitel \ref{sec:zuordnungToponymeGeogObj} erwähnten Problemen bei der Zuordnung von Toponymen zu geografischen Objekten sind noch weitere Probleme zu erwarten.
		Diese sind hauptsächlich bedingt durch die freie Eingabe des Nutzer-Standortes.
		Es wurden dabei folgende Klassen identifiziert in die die Werte eingeteilt werden können:

			\begin{enumerate}	
			 	\item Partieller geografischer Bezug des Wertes
			 	\item Widersprüchliche Datenwerte Werte
			 	\item Geografische Hierarchien
			 	\item Domänenspezifische Toponyme im Nutzer-Standort (Neologismen)
			 	\item Toponyme unterschiedlicher Hierarchieebenen
			 \end{enumerate}	 

		\subsection{Partieller geografischer Bezug des Datenwertes im Nutzer-Standort} \label{subsub:partiellerGeografischerBezug} 

			Hierbei haben nur Teile des Wertes im Nutzer-Standort geografischen Bezug. 
			Es werden oft weitere Informationen angegeben die keinen geografischen Bezug haben. 
			Im folgenden einige Beispiele.

			\begin{table}[h]
				\centering
				\caption{Beipiele für Nutzer-Standorte mit partiellem geografischen bezug}
				\label{my-label}
				\begin{tabular}{|l|l|l|}
				\hline
				\multicolumn{1}{|c|}{Wert} & \multicolumn{1}{c|}{geografischer Bezug} & \multicolumn{1}{c|}{kein geografischer Bezug} \\ \hline
				East side of that London & East side London & of that \\ \hline
				11th Dimension California & California & 11th Dimension \\ \hline
				New Orleans Home of the goons & New Orleans & Home of the goons \\ \hline
				Between here and there Miami & Miami & Between here and there \\ \hline
				\end{tabular}
			\end{table}
			
			Es können also auch nur Teile des Nutzer-Standorts für eine Geolokalisierung von Nutzen sein.
			Diese Informationen müssen extrahiert werden um einen Nutzen aus ihnen ziehen zu können. 

		\subsection{Widersprüchliche Toponyme im Nutzer-Standort} \label{subsub:wiederspruechlicheBezuege} 

			Es existieren auch Datenwerte in denen mehrere widersprüchliche Angaben gemacht werden.
			Dies bedeutet es werden zwei oder mehr Datenwerte mit geografischem Bezug angegeben, die auf unterschiedliche geografische Objekte verweisen.
			Auch hier sollen einige Beispiele genannt werden:

			\begin{table}[h]
			\centering
			\caption{Beispiele für widersprüchliche Toponyme}
			\label{my-label}
			\begin{tabular}{|l|l|l|l|}
			\hline
			\multicolumn{1}{|c|}{Wert}      & \multicolumn{1}{c|}{Toponym 1} & \multicolumn{1}{c|}{Toponym 2} & \multicolumn{1}{c|}{Entfernung in km} \\ \hline
			Bolton\textbackslash/Leigh      & Bolton                         & Leigh                          & 14 									\\ \hline
			Liverpool\textbackslash/London  & Liverpool                      & London                         & 350 								\\ \hline
			Balikesir \textbackslash/ Izmir & Balikesir                      & Izmir                          & 180 								\\ \hline
			\end{tabular}
			\end{table}					
				
			In diesen Beispielen sind jeweils zwei Städte angegeben.

			Es kann nun spekuliert werden wieso der Nutzer zwei Städte angibt.
			Ist er in einer der Städte aufgewachsen und lebt momentan in der anderen?
			Pendelt er zwischen den Städten um zu Arbeiten?

			Es kann hier nicht eindeutig entschieden werden in welcher Stadt sich der Nutzer aufhält.

		\subsection{Geografische Hierarchien im Nutzer-Standort} \label{subsub:geografischeHierarchienImNutzerStandort} 

			Es ist auch möglich, dass im Nutzer-Standort teile einer geografischen Hierarchie angegeben sind.
			Beispielsweise die Angabe einer Stadt in Kombination mit einem Land.
			
			In den USA wird beispielsweise oft die Stadt und der zugehörige Bundesstaat angegeben.
			In Brasilien hingegen wird oft ein Bundesstaat und das Land angegeben.
			Hier einige Beispiele:

			\begin{table}[h]
			\centering
			\caption{Beispiele für Nutzer-Standorte mit geografischen Hierarchieebenen}
			\label{my-label}
			\begin{tabular}{|l|c|c|c|c|}
			\hline
			\multicolumn{1}{|c|}{Wert} & Stadt       & \multicolumn{1}{l|}{Adm2} & Adm1 & \multicolumn{1}{l|}{Land} \\ \hline
			Los Angeles, California    & Los Angeles & Los Angeles (County)                                  & California                      & -                         \\ \hline
			Mato Grosso, Brazil        & -           & -                                                     & Mato Rrosso                     & Brazil                    \\ \hline
			West Sussex, England       & -           & -                                                     & West Sussex                     & England                   \\ \hline
			\end{tabular}
			\end{table}


			Aus diesen Werten kann spezielleres Wissen abgeleitet werden, da der Standort detaillierter beschrieben wird.
			Ein Sonderfall bildet das erste Beispiel, es handelt sich entweder um die Stadt Los Angeles oder um das Los Angeles County.
			Dies ist ein gutes Beispiel für die Doppeldeutigkeit eines Toponyms.
			Trotzdem wird durch die Angabe California der Standort genau genug spezifiziert. 
			Wenn man die Anfrage "'Los Angeles"' an das geonames.org Ortsverzeichnis sendet erhält man 91 Treffer. 
			Filtert man die Ergebnisse und überprüft ob "'California"' als Verwaltungsebene erster Ordnung eingetragen ist so erhält man lediglich einen Datensatz.  
			Durch die Kombination mehrere Toponyme unterschiedlicher geografischer Hierarchieebenen wird die geografische Position näher beschrieben.
			Diese Information sollte bei einer Untersuchung erhalten bleiben.
			
		\subsection{Domänenspezifische Toponyme im Nutzer-Standort} 

			In sozialen Netzwerken können sich eigene Begriffe und Formulierungen etablieren. 
			Diese sind im allgemeinen nicht bekannt.

			Im Twitter-Umfeld haben sich in den letzten Jahren einige spezielle Begriffe und Formulierungen zur Verwendung in Tweet-Texten etabliert. 
			Das im Twitter-Umfeld auch spezielle Toponyme im Nutzer-Standort verwendet werden, kann nicht gänzlich ausgeschlossen werden. 

			Ein Beispiel hierfür ist "'Bieberville"', welches in den untersuchten Daten von Hecht et al. öfter vorkommt.
			"'Bieberville"' wird abgeleitet von dem Pop-Star Justin Bieber.	
			Twitter wird oft als "'Bieberville"' bezeichnet.
			Da der Pop-Star in Twitter sehr aktiv ist und deshalb viele seiner Fans auch in Twitter aktiv sind hat sich dieser Name etabliert.
			Unter diesem Gesichtspunkt hätte "'Bieberville"' keinen geografischen Bezug.
			Sucht man allerdings im Internet nach "'Bieberville"' stößt man auf einen Imbiss in Groß-Bieberau.
			"'Bieberville"' kann also durchaus einen geografischen Bezug haben, wenngleich es im Twitter-Umfeld nicht als solcher benutzt wird. 
			Ein Nutzer-Standort der in einem Land kein Toponym darstellt, kann in einem anderen durchaus ein Toponym sein.
			Ist in einem Ortsverzeichnis beispielsweise "'Bierberville"' als Bezeichnung für den Imbiss in Groß-Bieberau hinterlegt, würde dieser als Georeferenz zugeordnet werden, was im Umfeld von Twitter in einem Großteil der Fälle falsch wäre.

			Solche Begriffe und Formulierungen können nur sehr schlecht mit einem Ortsverzeichnis zu Georeferenzen zugeordnet werden. 

		\subsection{Qualitative } 
			
			Es ist zunächst nötig die genutzten geografischen Indikatoren einer Vorverarbeitung zu unterziehen. 
			Ziel ist es aus dem Nutzer-Standort möglichst viele Informationen zu extrahieren und etwaige Probleme zu beseitigen.
			In der folgenden Liste sind einige Nutzer-Standorte angegeben. 
			Anhand dieser Liste sollen die Vorverarbeitungsschritte demonstriert werden.

			\begin{multicols}{2}
			\begin{enumerate}
				\item Bélem-PA
				\item West Sussex, England
				\item South Florida
				\item Pitmedden,  Scotland, UK
				\item Mato Grosso \& Rio de Janeiro
				\item -****-
				\item USA \textbackslash/ Los Angeles
				\item Nottingham\textbackslash/London
				\item Los Angeles, USA
				\item I $\heartsuit$ New York 
				\item $\dagger$\textasciitilde Los Angeles\textasciitilde$\dagger$
				\item earth-sea
				\item In front of the computer
				\item 11th Dimension | California
				\item York
				\item York
			\end{enumerate}
			\end{multicols}

		\subsection{Zusammenfassung der quantitativen Erhebungen ----- BIS HIER --------} 

			In ca. 80\% der Fälle kann der Nutzer-Standort tatsächlich einen Hinweis auf eine Georeferenz liefern.
			Die Eingaben sind allerdings nicht standardisiert und können somit nicht ohne Vorverarbeitung verwendet werden.
			Wie oben dargestellt kann der Nutzer-Standort nur partiellen geografischen Bezug haben.
			Es können mehrere Informationen verschiedener geografischer Hierarchieebenen im Datenwert auftauchen. 
			Und die Datenwerte im Nutzer-Standort können keinerlei geografischen Bezug aufweisen.
			Dies macht es grundsätzlich schwierig die Datenwerte im Nutzer-Standort direkt als geografischen Indikator zu verwenden.
			
			Ist ein geografischer Bezug des Nutzer-Standortes nachzuweisen, handelt es sich bei dem Eintrag in den meisten Fällen um ein Toponym.
			Bei der durchgeführten Untersuchung wurden nur Ortsverzeichnisse zur Bestimmung eines geografischen Bezugs verwendet.
			In Ortsverzeichnissen sind nur Toponyme hinterlegt es kann deshalb davon ausgegangen werden, dass ca. 76\% der Nutzer-Standorte auf Toponyme zurückzuführen sind.

			Widersprüchliche Angaben, geografische Hierarchien und partieller geografischer Bezug können bei der Geolokalisierung zu Problemen führen.
			Diese entstehen meist dadurch, dass der Datenwert nicht nur eine konkrete, sondern mehrere Angaben beinhaltet.
			
			Um den Problemen zu begegnen sollen die Datenwerte im Nutzer-Standort einer Vorverarbeitung unterzogen werden. 
			Dadurch sollen möglichst viele einzelne Informationen aus dem Nutzer-Standort extrahiert werden.

			Es können Beispielsweise Datenwerte mit geografischem Bezug von Datenwerten ohne geografischen Bezug getrennt untersucht werden.

			Sollten Mehrdeutigkeiten im Nutzer-Standort auftauchen können diese teilweise durch die Hinzunahme der Nutzer-Zeitzone aufgelöst werden.

			\textit{Dieser Teil zu Begründung warum Ortsverzichnis Abfrage schlecht} 
			In Ortsverzeichnissen kann eine große Anzahl an Toponymen hinterlegt werden.
			Aufgrund der immensen Vielfalt an Toponymen ist es aber nahezu unmöglich alle existierenden Toponyme abzudecken. 

	\section{Verfahren zum Einlernen der Georeferenz-Basis} \label{sec:VefrahrenZumEinlernen} 

		Beim e


		\textit{EINLEITUNG ALT} 

			Die Probleme bei der Nutzung von Ortsverzeichnissen sind meist auf deren Unvollständigkeit zurückzuführen. 

			Der Vorteil besteht darin, dass eine domänenspezifische Georeferenz-Basis geschaffen wird.
			Diese kann potenziell mehr geografische Indikatoren zuordnen als ein normales Ortsverzeichnis.
			Denn es können dadurch domänenspezifische Eigenheiten bei der Verwendung von Toponymen berücksichtigt werden. 
			Auch domäneninterne Begriffe sollen hierdurch gelernt werden können.

			Im ersten Unterkapitel soll der Ablauf und die nötigen Daten zur Umsetzung eines solchen Lernverfahrens erläutert werden.

			Danach wird auf die Vorverarbeitung des Nutzer-Standortes eingegangen.
			Es werden basierend auf den erzeugten Referenzwerten und den zugeordneten Georeferenzen absolute Häufigkeiten bestimmt.
			Des weiteren wird die in Abschnitt \ref{sec:generelleStruktur} vorgestellte Struktur der Georeferenz-Basis angepasst und erweitert.
			Dabei wird am grundsätzlichen Prinzip nichts geändert, es wird nach wie vor einem Referenzwert eine Georeferenz zugewiesen. 

			Die eingelernte Georeferenz-Basis soll es ermöglichen durch die Datenwerte im Nutzer-Standort und der Nutzer-Zeitzone eine Georeferenz zu bestimmen.
			Nach der Struktur aus Kapitel \ref{sec:generelleStruktur} besteht eine Georeferenz-Basis zumindest aus Referenzwerten und einer zugeordneten Georeferenz.
			Die Referenzwerte sollen aus dem Nutzer-Standort und der Nutzer-Zeitzone erzeugt werden.
			Um eine Georeferenz zuordnen zu können muss zu jedem Tupel aus Nutzer-Standort und Nutzer-Zeitzone eine Georeferenz vorliegen.
			Als Basis für das Lernverfahren werden die Tweet-Lerndaten genutzt.
			Jeder Tweet enthält dabei neben dem Nutzer-Standort und der Nutzer-Zeitzone eine Georeferenz in Form von Längen- und Breitengrad. 
			Die Tweets werden mit Hilfe des Längen- und Breitengrades der nächstgelegenen Stadt zugeordnet werden. 
			Dadurch kann jedem Referenzwert in der Georeferenz-Basis später eine Stadt als Georeferenz zugeordnet werden.
			Es entstehen somit Tupel aus einem Referenzwert und einer zugehörigen Georeferenz.
			Diese sollen in der Georeferenz-Basis gespeichert werden.

			Zusätzlich soll ein neues Feld in der Georeferenz-Basis eingeführt werden.
			Dieses soll angeben, wie oft die Kombination aus einem Referenzwert und einer Stadt vorgekommen ist. 
			Das Feld gibt also die absolute Häufigkeit der Vorkommen an.
			Nachdem die Georeferenz-Basis eingelernt ist liegen somit für jeden Referenzwert eine Georeferenz und die absolute Häufigkeit vor.
			Basierend darauf können dann die relativen Häufigkeiten für jedes paar aus Referenzwert und Stadt berechnet werden.
			Bei der Geolokalisierung werden diese Werte genutzt um die wahrscheinlichste Georeferenz für einen gegebenen Nutzer-Standort und Nutzer-Zeitzone zu bestimmen.

			In Abbildung \ref{img:EinlernenAllg} ist das Verfahren zum einlernen einer Georeferenz-Basis an einem Beispiel dargestellt.

			\begin{figure}[!ht]
					\begin{center}
						\includegraphics[scale=0.5]{einlernenAllg.pdf}
						\caption{Verfahren zum einlernen einer Georeferenz-Basis}
						\label{img:EinlernenAllg}
					\end{center}
			\end{figure}	

			In den nächsten Unterkapiteln sollen nun die einzelnen Schritte des Verfahrens genauer betrachtet werden. 
			Zudem wird die Erweiterung der Datenbasis erläutert.
	           
			Als Referenzwerte sollen die Informationen aus dem Nutzer-Standort verwendet werden 
			Bei der Verwendung des Nutzer-Standort als Referenzwert können Probleme auftreten. 
			Die 
			Um die Informationen aus dem Nutzer-Standort als Referenzwert nutzen zu können 

			muss deshalb eine Vorverarbeitung stattfinden. 

			Der Nutzer-Standort soll zunächst eingehender betrachtet werden.

		\textit{SUMMARY ALT} 
			
			In Abbildung \ref{img:einlernenAblauf} ist der Gesamtablauf des Einlernens dargestellt.  
			Zunächst werden die einzelnen Vorverarbeitungsschritte für den Nutzer-Standort und die Nutzer-Zeitzone durchgeführt.
			Parallel kann der Längen- und Breitengrad auf eine Stadt aufgelöst werden.
			Die dadurch entstandenen potenziellen geografischen Indikatoren und die zugehörige Georeferenz wird nun in die Georeferenz-Basis gespeichert. 
			Dabei wird überprüft ob dieser Datensatz bereits vorhanden ist. 
			Ist dies der Fall, wird der Häufigkeitswert des entsprechenden Datensatzes inkrementiert.
			Ist der Datensatz noch nicht vorhanden wird er angelegt und die absolute Häufigkeiten mit 1 initialisiert.

			 \begin{figure}[!ht]
					\begin{center}
						\includegraphics[scale=0.6]{einlernenAblauf.pdf}
						\caption{Ablaufplan einlernen}
						\label{img:einlernenAblauf}
					\end{center}
				\end{figure}


			Die Vorverarbeitung extrahiert dabei zusätzliche Informationen aus dem Nutzer-Standort.
			Durch die Bereiningung der Werte im Nutzer-Standort, die alphanumerische Sortierung, das identifzieren von Toponymen mit mehreren Worten und die darauffolgende Erzeugung von N-Grammen werden zusätzliche Informationen aus jedem Nutzer-Standort gewonnen.
			Die Nutzer-Zeitzone wird einbezogen um Doppeldeutigkeiten auflösen zu können.

			Durch die neue Datenstruktur, mit den absoluten Häufigkeiten und der abgebildeten geografischen Hierarchie, lassen sich nun tiefergehende Analysen durchführen um eine robuste Geolokalisierung zu ermöglichen. 
			Die absoluten Häufigkeiten geben dabei an wie oft ein Referenzwert in einer bestimmten Region vorkommt.

			Durch dieses Verfahren lässt sich eine Datenbasis erzeugen die domänenspezifische Eigenheiten, in Bezug auf die Verwendung spezieller Begriffe oder Formulierungen, berücksichtigt.
			Des weiteren werden Toponyme, die in Ortsverzeichnissen unter Umständen nicht hinterlegt sind, berücksichtigt.
			Auch geografische Indikatoren mit mittelbarem geografischen Bezug, zum Beispiel die Verwendung spezieller Begriffe in einer geografischen Region, können einbezogen werden. 

		

		
		\subsection{Vorverarbeitungsschritte des Nutzer-Standortes}

			\subsubsection{Bereinigung des Nutzer-Standortes}

				\textit{Eliminierung von Sonder- und Satzzeichen} 
				Es werden oft Sonder- und Satzzeichen im Nutzer-Standort verwendet. 
				Beispielsweise als Trenner zwischen Toponymen unterschiedlicher geografischer Hierarchieebenen.
				Beispiele hierfür sind "'West Sussex, England"', "'USA \textbackslash/ Los Angeles"' oder "'Bélem-PA"'.
				Das Trennzeichen wird dabei nicht einheitlich verwendet.  
				Es kann deshalb nicht entschieden werden ob ein Satzzeichen als Trenner zweier Hierarchieebenen genutzt wird oder nicht.
				Bei "'USA \textbackslash/ Los Angeles"' wird \textbackslash/ als Trenner für Hierarchieebenen verwendet.
				Bei "'Nottingham\textbackslash/London"' hingegen werden zwei Städte angegeben.
				Es ist also insbesondere nicht klar welcher Zusammenhang zwischen den Datenwerten, die durch ein Sonder- oder Satzzeichen getrennt sind, besteht. 

				Bei "'I $\heartsuit$ New York "' werden Sonderzeichen zum ausdrücken von Emotionen genutzt.
				In "'$\dagger$\textasciitilde Los Angeles\textasciitilde$\dagger$"' werden Sonderzeichen als Dekoration genutzt.
				Einige Nutzer-Standorte bestehen ausschließlich aus Sonder- und Satzzeichen.
				
				Grundsätzlich ist es aufgrund schwierig zu entscheiden ob Satz- und Sonderzeichen einen Mehrwert bieten um einen Nutzer-Stadort genauer zu bestimmen.
				Dies ist immer dann der Fall, wenn die Sonder- und Satzzeichen Bestandteil eines Toponyms sind. 
				Es gibt tatsächlich Fälle in denen Sonder- oder Satzzeichen Bestandteil eines Toponyms sind.
				Ein Beispiel hierfür wäre "'3. Arrondisement Paris"'.
				Das weggelassen des Punktes hätte allerdings keinen Einfluss auf die gelieferte Information.

				In den oben genannten Fällen bringen Sonder- und Satzzeichen keine zusätzlichen Informationen.
				Es sollen deshalb in einem ersten Vorverarbeitungsschritt alle Sonder- und Satzzeichen entfernt werden. 

				Liste nach dem entfernen von Sonder- und Satzzeichen:

				\begin{multicols}{2}
					\begin{enumerate}
						\item Bélem PA
						\item West Sussex England
						\item South Florida
						\item Pitmedden Scotland UK
						\item Mato Grosso Rio de Janeiro
						\item 
						\item USA Los Angeles
						\item Nottingham London
						\item Los Angeles USA
						\item I New York 
						\item Los Angeles
						\item earth sea
						\item In front of the computer
						\item 11th Dimension California
						\item York
						\item York
					\end{enumerate}
				\end{multicols}

				Der Wert 6 existiert nun nicht mehr, der Wert ist leer und wird somit nicht weiter betrachtet.

				Unnötige Sonder- und Satzzeichen wurden entfernt und die entstandenen Werte können nun einfacher weiterverarbeitet werden.

			\subsubsection{Zusammenfassen von Toponymen mit Hilfe von a priori Wissen aus einem Ortsverzeichnis}

				\textit{Zusammenfassen von Toponymen}

				Oft bestehen Toponyme aus zwei oder mehr Worten.
				Diese sollen mit Hilfe eines Ortsverzeichnisses zusammengefasst werden. 
				Dies kann selbstverständlich nur für bekannte Toponyme durchgeführt werden.
				
				"'Los"' und "'Angeles"' bilden gemeinsam "'Los Angeles"' und sollen in der weiteren Verarbeitung gemeinsam betrachtet werden. 
				Dies soll zunächst mit einem + Zeichen gekennzeichnet werden.

				Daraus resultiert:

				\begin{multicols}{2}
					\begin{enumerate}
						\item Bélem PA
						\item West+Sussex England
						\item South Florida
						\item Pitmedden Scotland UK
						\item Mato+Grosso Rio+de+Janeiro
						\item USA Los+Angeles
						\item Nottingham London
						\item Los+Angeles USA
						\item I New+York 
						\item Los+Angeles
						\item earth sea
						\item In front of the computer
						\item 11th Dimension California
						\item York
						\item York
					\end{enumerate}
				\end{multicols}

				In diesem Schritt wird insbesondere keine Geolokalisierung vorgenommen. 
				Er dient dazu möglichst früh vorhandenes Wissen über Toponyme einzubeziehen und somit eine unnötige Fragmentierung zu vermeiden.
				Im wesentlichen sollen die Werte hiermit für die nächsten Verarbeitungsschritte vorbereitet werden.

			\subsubsection{Angleichung durch alphanumerische Sortierung}

				\textit{Alphanumerische Sortierung}

				In diesem Schritt sollen die Werte alphanumerisch sortiert werden. 

				Nach der Sortierung liegt folgende Tabelle vor:

				\begin{multicols}{2}
					\begin{enumerate}
						\item Bélem PA
						\item England West+Sussex
						\item Florida South
						\item Pitmedden Scotland UK
						\item Mato+Grosso Rio+de+Janeiro
						\item Los+Angeles USA
						\item London Nottingham
						\item Los+Angeles USA
						\item I New+York 
						\item Los+Angeles
						\item earth sea
						\item computer front In of the 
						\item 11th California Dimension 
						\item York
						\item York
					\end{enumerate}
				\end{multicols}

				Die Werte 6 und 8 sind nun gleich. 
				Durch die Sortierung werden Werte mit gleichem Inhalt aber unterschiedlicher Reihenfolge angeglichen.
				Durch das zusammenfassen der Toponyme im vorherigen Schritt werden bekannte Toponyme nicht getrennt.

				Dieser Schritt stellt allerdings einen Kompromiss dar.
				Es werden zwar Werte mit gleichem Inhalt und unterschiedlicher Reihenfolge angeglichen.
				Aber es werden auch potenzielle Toponyme, die aus mehreren Teilen bestehen, auseinandergezogen.
				Aus "'Motor City Michigan USA"' würde "'City Michigan Motor USA"' entstehen.
				Der Zusammenhang zwischen Motor und City wäre nicht mehr vorhanden und könnte auch nicht wiedergewonnen werden.

		\subsection{Fragmentierung zur Extraktion potenziell geografischer Indikatoren} 

				\textit{Erzeugung von N-Grammen}

				Es sollen nun N-Gramme bis zum Grad 3 erzeugt werden. 
				Dieses Vorgehen löst gleich mehrere Probleme.

				Zum ersten können sowohl geografische Indikatoren als auch Werte die kein geografischen Indikator darstellen in den Nutzer-Standorten vorhanden sein.
				Durch die Erzeugung von N-Grammen können diese getrennt voneinander betrachtet werden.
				Es löst das Problem des partiellen geografischen Bezugs des Nutzer-Standortes aus \ref{subsec:partiellerGeografischerBezug} in der Hinsicht, dass die Werte für die weitere Verarbeitung getrennt betrachtet werden können.

				Zum zweiten können in einem Nutzer-Standort mehrere geografische Indikatoren enthalten sein.  
				Der Wert "'Pitmedden Scotland UK"' enthält mit "'UK"' das Land, mit "'Scottland"' den Landesteil und mit "'Pitemedden"' eine Stadt. 
				Mit dem Längen- und Breitengrad aus dem zugehörigen Tweet kann nun allen drei Werten eine Georeferenz zugeordnet werden.

				Aber auch zwei verschiedene Städte wie "'Nottingham\textbackslash/London"' können in einem Wert vorkommen.
				Diese werden zu "'Nottingham\textbackslash/London"', "'Nottingham"' und "'London"'.
				Wird hier wiederum beiden Werten die geografische Koordinate zugeordnet können drei Fälle auftreten.
				Erstens: Der Nutzer war in keiner der beiden Städte als er den Tweet abgesetzt hat. 
				Damit sind beide Zuordnungen unbrauchbar.
				Zweitens: Er war in einer der beiden Städte, dann ist zumindest einer der entstandenen Datensätze brauchbar.
				Dies löst das Problem aus \ref{subsec:wiederspruechlicheBezuege} in der Hinsicht, dass die Werte für die weitere Verarbeitung getrennt betrachtet werden können.

				Besteht allerdings eine Beziehung der Werte zueinander, kann diese durch Bi- und Trigramme abgebildet werden.
				Aus "'Pitmedden Scotland UK"' wird "'Pitmedden Scotland"', "'Scotland UK"' und "'Pitmedden Scotland UK"' erzeugt.

				Dieser Schritt soll nur an einigen ausgewählten Beispielen aus der Liste erfolgen.
				Die Bestandteile der N-Gramme werden zur Verdeutlichung mit \textless \textgreater gekennzeichnet in späteren Beispielen wird dies weggelassen, da durch das zusammenfassen von Werten mit einem plus klar ist welches die Elemente des NGramms darstellen

				\begin{multicols}{2}
					\begin{enumerate}
						\item \textless England\textgreater   \textless West+Sussex\textgreater  
						\item \textless England\textgreater  
						\item \textless West+Sussex\textgreater  
						\item \textless Mato+Grosso\textgreater   \textless Rio+de+Janeiro\textgreater  
						\item \textless Rio+de+Janeiro\textgreater  
						\item \textless Mato+Grosso\textgreater  
						\item \textless 11th\textgreater   \textless California\textgreater   \textless Dimension\textgreater   
						\item \textless 11th\textgreater   \textless California\textgreater  
						\item \textless California\textgreater   \textless Dimension\textgreater   
						\item \textless 11th\textgreater  
						\item \textless California\textgreater  
						\item \textless Dimension\textgreater   
						\item \textless York\textgreater  
						\item \textless York\textgreater  
					\end{enumerate}
				\end{multicols}
				Der Wert 7 ist ein Trigramm.
				Bei den Werten 1,4,8 und 9 handelt es sich um Bigramme. 
				Die Werte von 2,3,5,6,10,11 und 12 sind Unigramme.				

				In diesem Schritt werden aus dem Nutzer-Standort mehrere potenzielle geografische Indikatoren erzeugt die als Referenzwerte genutzt werden können.
				Bevor diese als Referenzwerte genutzt werden soll aber noch die Nutzer-Zeitzone einbezogen werden.



		\subsection{Eliminierung von Mehr- und Doppeldeutigkeiten}

				In Kapitel \ref{sec:ToponymeInGeografischenIndikatoren} wurde auf die Probleme bei der Verwendung von Toponymen als geografische Indikatoren eingegangen.
				Dabei wurde die Doppel- und Mehrdeutigkeit von Toponymen betrachtet.
				Bei der Verwendung des Nutzer-Standortes als geografischen Indikator kann dieses Problem auch auftauchen. 

				Um diesem Problem zu begegnen soll die Nutzer-Zeitzone als weiterer geografischer Indikator hinzugezogen werden.
				Zunächst sollen die generellen Eigenschaften der Nutzer-Zeitzone erläutert werden bevor erklärt wird wie die Nutzer-Zeitzone das Problem der Mehrdeutigkeit von Toponymen beheben kann.

				\textit{Eigenschaften der Nutzer-Zeitzone}

					Die Nutzer-Zeitzone stellt einen geografischen Indikator dar.
					Sie beschreibt eine eindeutige geografische Region auf dem Globus und somit hat direkten geografischen Bezug.
					Dabei entsprechen die Grenzen der Region nicht unbedingt den Landesgrenzen oder den Grenzen sonstiger Verwaltungseinheiten. 
					In Abbildung \ref{img:timezones} sind die Zeitzonen der Erde dargestellt.

					\begin{figure}[!ht]
						\begin{center}
							\includegraphics[scale=0.5]{zeitzonen_weltkarte.jpg}
							\caption{Zeitzonen der Erde.}
							\label{img:timezones}
						\end{center}
					\end{figure}	
					%Quelle: http://www.zeitzonen.de/images/frontend/mod\_tz\_map/zeitzonen\_weltkarte.gif

					Die Nutzer-Zeitzone kann in Twitter über eine Liste gewählt werden.
					Der Wert der Nutzer-Zeitzone stellt deshalb garantiert eine Zeitzone dar.
					Es ist dem Nutzer nicht möglich einen Wert einzugeben der nicht einer Zeitzone entspricht.

					In Abbildung \ref{img:usTimezones} wurden Tweets anhand ihres Längen- und Breitengrades platziert.
					Jeder Punkt in der Abbildung entspricht einem Tweet.
					Es wurden nur Tweets aus den USA ausgewählt.
					Anhand der Nutzer-Zeitzone wurde jedem Tweet eine Farbe zugeordnet.
					In Tabelle \ref{tab:timezoneColors} sind die Farbzuordnungen aufgelistet. 

					\begin{table}[h]
					\centering
					\caption{In Abbildung \ref{img:usTimezones} werden folgende Farben verwendet}
					\label{tab:timezoneColors}
						\begin{tabular}{|l|l|}
							\hline
							Zeitzone      & Farbe     \\ \hline \hline
							Pacific Time  & Rot       \\ \hline
							Eastern Time  & Grün/Gelb \\ \hline
							Central Time  & Blau      \\ \hline
							Mountain Time & Pink      \\ \hline
						\end{tabular}
					\end{table}

					 \begin{figure}[!ht]
						\begin{center}
							\includegraphics[scale=0.5]{usTimezones.pdf}
							\caption{Tweets, abhängig der Zeitzone eingefärbt}
							\label{img:usTimezones}
						\end{center}
					\end{figure}	

					 Die Zeitzonen sind an den Farben gut zu erkennen. 
					 Lediglich die dünn besiedelte Region der Mountain Time kann nur an einigen Ballungszentren erkannt werden. 
					 Grundsätzlich scheint der Großteil der Angaben aber korrekt zu sein.

					Ist ein Toponym Doppel- oder Mehrdeutig kann nicht entschieden werden welches geografische Objekt zugeordnet werden soll.
					Liegen die beiden geografischen Objekte allerdings in zwei unterschiedlichen Zeitzonen und der Nutzer hat eine Nutzer-Zeitzone angegeben kann die Doppeldeutigkeit aufgelöst werden.
					Es muss lediglich die Nutzer-Zeitzone betrachtet werden.
					Somit kann dem Nutzer eine korrekte Georeferenz zugewiesen werden.
					Voraussetzung hierfür ist natürlich, dass die geografischen Objekte in zwei unterschiedlichen Zeitzonen liegen und die Nutzer-Zeitzone angegegeben ist.

				\textit{Hinzufügen der Nutzer-Zeitzone}

					Die Zeitzone stellt eine begrenzte Anzahl an Werten dar.
					Diese werden vom Nutzer nicht frei eingegeben. 
					Es wird hier hier deshalb keine weitere Vorverarbeitung vorgenommen.

					An die Referenzwerte, die aus dem Nutzer-Standort erzeugt wurden soll nun die Zeitzone angehängt werden.
					Jeder Referenzwert soll einmal mit und einmal ohne Zeitzone existieren. 
					Damit wird garantiert, dass eingelernte Referenzwerte die eine falsche Zeitzone aufweisen, trotzdem berücksichtigt werden können. 
					Beispielsweise ist die Nutzer-Zeitzone "'Pacific Time (US \& Canada)"' für den Nutzer-Standort "'Jakarta"' nicht korrekt.
					Aus dieser Kombination würden die Referenzwerte "'Jakarta Pacific Time (US \& Canada)"' und "'Jakarta"' entstehen. 
					Bei einer Abfrage von "'Jakarta"' kann dann trotzdem der Referenzwert "'Jakarta"' genutzt werden. 
					Würde allerdings nur die Kombination "'Jakarta Pacific Time (US \& Canada)"' als Referenzwert existieren würde keine Georeferenz zurückgeliefert werden können.

					Die Elemente der Zeitzonen werden wiederum mit einem Plus zusammengefasst um deutlich zu machen, dass es sich um ein Element eines NGramms handelt.
					Um die Nutzer-Zeitzone von den aus dem Nutzer-Standort generierten Elementen unterscheiden zu können, wird die Nutzer-Zeitzone kursiv geschrieben.
					Auch hier wird die Liste weiter eingeschränkt und es werden lediglich noch zwei Beispiele betrachtet.
					\begin{multicols}{2}
						\begin{enumerate}
							\item England West+Sussex
							\item  England  
							\item  West+Sussex  
							\item  England    West+Sussex   \textit{London}
							\item  England   \textit{London}
							\item  West+Sussex   \textit{London}
							\item  11th    California    Dimension    
							\item  11th    California  
							\item  California    Dimension   
							\item  11th  
							\item  California  
							\item  Dimension   
							\item  11th  California    Dimension   \textit{Pacific+Time+US+Canada}
							\item  11th    California   \textit{Pacific+Time+US+Canada}
							\item  California    Dimension   \textit{Pacific+Time+US+Canada}
							\item  11th   \textit{Pacific+Time+US+Canada}
							\item  California   \textit{Pacific+Time+US+Canada}
							\item  Dimension   \textit{Pacific+Time+US+Canada}
							\item York
							\item  York   \textit{London}
							\item York
							\item  York   \textit{Eastern+Time}
						\end{enumerate}	
					\end{multicols}
					Mit Hilfe der Zeitzone können nun auch die beiden letzten Einträge unterschieden werden. 
					Zum einen "'York"' in England zum anderen "'York"' in den USA. 
					Dies löst das Problem der Mehrdeutgkeiten von Toponymen. 
					Mit Hilfe der zusätzlichen Zeitzone können geografische Objekte, zumindest wenn sie in zwei verschiedenen Zeitzonen liegen, unterschieden werden.

		\subsection{Quantitative Betrachtung der Referenzwerte bezüglich der geografischen Position}

			\todo{chap: Lösungsansatz sec: einlernen subsec quantitative betrachtung ---- mergen und splitten komplett überarbeiten ---} 

			\subsubsection{Zählen}

				\textit{Absolute Häufigkeiten}

					Es sollen nun absolute Häufigkeiten in der Georeferenz-Basis eingeführt werden.
					Die absoluten Häufigkeiten sollen angeben wie oft ein Tupel aus Referenzwert und Georeferenz in der Georeferenz-Basis vorhanden ist.
					Dadurch werden Duplikate in der Georeferenz-Basis vermieden.
					Zusätzlich kann ermittelt werden ob ein Referenzwert an einer bestimmten Position gehäuft auftritt.

					Vor dem abspeichern eines neuen Tupels soll nun zunächst geprüft werden ob dieses bereits in der Georeferenz-Basis vorhanden ist. 
					Ist dies der Fall, so wird die absolute Häufigkeit des entsprechenden Eintrags um 1 erhöht.
					Ist das Tupel noch nicht gespeichert so wird ein neuer Datensatz angelegt und die absolute Häufigkeit mit 1 initialisiert.

					Die absolute Häufigkeit misst damit wie oft ein Referenzwert an einer geografischen Position vorkommt.
					Es ist zu erwarten, dass ein Referenzwert mit geografischem Bezug gehäuft an einer bestimmten geografischen Position oder in einer Region auftritt.
					Während Referenzwerte die keinen geografischen Bezug haben sehr verteilt oder nur sehr selten auftreten. 

			\subsubsection{Problem der Genauigkeit geografischen Koordinaten in Tweets ----- NACH UNTEN MERGEN -------}  

				Betrachtet man die Tweet-Lerndaten kann beim Vergleich der Georeferenzen ein Problem entstehen.
				Die Georeferenz aus den Tweet-Lerndaten besteht aus den geografischen Koordinaten des zugehörigen Tweets.
				Die geografische Position wird zumeist mit Hilfe von GPS-Modulen mobiler Endgeräte wie Smartphones bestimmt. 
				Diese können eine Position oft auf wenige Meter genau bestimmen.
				Das bedeutet, zwei Tweets die wenige Meter voneinander abgesetzt wurden haben unter Umständen unterschiedliche Werte für den Längen- und Breitengrad.
				Dies kann für die Bestimmung der Häufigkeiten problematisch sein, da die Werte in der Regel nicht exakt übereinstimmen.
				Um dies zu umgehen sollen die geografischen Koordinaten auf Städte abgebildet werden.

				\textit{Voronoi-Diagramme} 

					Sei eine Menge von Punkten $Z = {z_1,z_2,...,z_n}$ auf einer Ebene verteilt.
					Eine Voronoi-Region $V^i$ zu einem Punkt $z_i$ beinhaltet dann alle Punkte $P^i={p_1,p_2,...,p_n}$ die näher an $z_i$ liegen als an allen anderen Punkten $Z_j={z_j in Z|z_j!=z_i}$.
					Alle Voronoi-Regionen zu allen Punkten in Z bilden ein Voronoi-Diagramm.

					Dieses Konzept kann zur Bestimmung der nächstgelegenen Stadt verwendet werden. 
					Die geografischen Positionen der Städte bilden dabei die Punkte $Z$.
					Zu jeder Stadt wird nun die Voronoi-Region erzeugt. 
					Anhand der Voronoi-Region in der ein Punkt p liegt kann nun bestimmt werden welche Stadt am nächsten zu p liegt.

					Jeder Punkt auf dem Globus kann so einer Stadt zugeordnet werden.
					Insbesondere wird der Globus auf Städteebene in geografische Regionen eingeteilt.
					In Abbildung \ref{img:voronoi} ist ein Voronoi-Diagramm einiger deutscher Städte dargestellt.

					\begin{figure}[h!]
					\begin{center}
					\includegraphics[scale=0.5]{voronoi.png}
					\caption{Vornoi-Diagramm Deutschland}
					\label{img:voronoi}
					\end{center}
					\end{figure}	

					Durch die Teilmengenbeziehung ist es nun nicht mehr nötig die geografischen Regionen der anderen Hierarchieebenen zu bestimmen.
					Es ist zu beachten, dass durch die Erzeugung der Vornoi-Regionen Ländergrenzen nur approximiert werden können. 
					Voronoi-Regionen zu Städten die in der Nähe einer Landesgrenze liegen können über die Landesgrenzen hinausgehen. 
					Damit werden geografische Positionen unter Umständen dem falschen Land zugeordnet.
					Auch dies ist in Abbildung \ref{img:voronoi} an den Landesgrenzen zu erkennen. 
					Umso größer allerdings die Anzahl der Städte ist, umso genauer wird die Approximation. 

				\textit{Vorverarbeitung der geografischen Koordinaten --------- AUFSPLITTEN ---------}

					Die geografischen Koordinaten sollen auf Städte aufgelöst werden.
					Dies entspricht der untersten Ebene der geografischen Hierarchie.
					Bei der Auflösung auf eine Stadt werden durch die geografischen Hierarchieebenen implizit auch die Verwaltungseinheiten und das Land bestimmt.

					Jedem Tweet soll mit Hilfe von Voronoi-Regionen die am nächsten gelegene Stadt zugeordnet werden.
					Als Ergebnis liegt nun pro Referenzwert statt einer geografischen Koordinate eine Stadt vor.
					Dies kann als Übergang einer kontinuierlichen Darstellung durch geografische Koordinaten zu einer diskreten Darstellung durch Städte angesehen werden. 
					Wird ein Tweet innerhalb einer Voronoi-Region abgesetzt wird er der enstprechenden Stadt zugeordnet.

					Es bleibt zu definieren was unter einer Stadt zu verstehen ist.
					In dicht besiedelten Gebieten können viele kleine Städte vorhanden sein. 
					Damit wäre die Positionsangabe wiederum zu genau.
					Deshalb soll die Definition einer Stadt hier über die Einwohnerzahl stattfinden.
					Es werden nur Städte verwendet deren Einwohnerzahl 15000 überschreitet.
					Der Tweet wird also der nächstgelegenen Stadt mit mehr als 15000 Einwohnern zugeordnet.
					In Ballungsräumen sind damit mehr Städte zu erwarten, womit die Genauigkeit zunimmt.
					Es ist allerdings auch zu erwarten das dort tendenziell mehr Tweets abgesetzt werden als in ländlichen Gebieten.
					Womit in jeder Voronoi-Region zu einer Stadt ausreichend Tweets vorhanden sind.
					Damit können potenziell mehr Tweets zu einer Stadt zugeordnet werden.
					Wohingegen in ländlichen Gebieten ein größeres Einzugsgebiet pro Stadt zustande kommt.
					Es sind allerdings auch weniger Tweets zu erwarten. 
					Durch das größere Einzugsgebiet werden dennoch ausreichend viele Tweets auf eine Stadt zugeordnet. 

					Die Distanzen, welche zwischen der tatsächlichen Position des Tweets und der zugeordneten Stadt liegen wurden protokolliert. 
					Dies spiegelt den Fehler der bei einer solchen Zuordnung entsteht wieder.

					In Tabelle \ref{tab:distances} sind die Ergebnisse dargestellt.

					\begin{table}[h]
					\centering
					\caption{Fehlerdistanzen zwischen Tweet Ursprung und zugeordneter Stadt (in km)}
					\label{tab:distances}
					\begin{tabular}{|l|l|}
					Durchschnitt & 7      \\ \hline
					Median       & 3.5    \\ \hline
					0.25 Quantil & 1.7    \\ \hline
					0.75 Quantil & 6.9    \\ \hline
					0.85 Quantil & 10     \\ \hline
					0.95 Quantil & 24.1   \\ \hline
					0.98 Quantil & 44.2   \\ \hline
					Größte Distanz      & 3424.5 \\ \hline
					Kleinste Distanz     & 0     
					\end{tabular}
					\end{table}

					Im Median liegt die Fehlerdistanz zwischen der tatsächlichen Position und der zugeordneten Stadt bei 3,5 Kilometern.
					Über die Quantile können die Fehlerdistanzen noch genauer untersucht werden.
					Das 0.25 Quantil sagt aus, dass 25\% aller Fehlerdistanzen unter 1,7 Kilometern liegen.
					95\% der Fehlerdistanzen liegen unter 24,1 Kilometer. 
					Dies ist ausreichend genau. 

		\subsection{Vorverarbeitung der geografischen Koordinaten -------- von oben HIER REIN MERGEN --------}  

			\textit{Fazit ------ > fraglich wohin.} 

				\todo{chap: Lösungsansatz sec: einlernen subsec- vorver geog Koordinaten --- > Fazit aufteilen}

				Nach der Vorverarbeitung liegen eine Menge von potenziellen geografischen Indikatoren mit zugehörigen geografischen Koordinaten vor.
				Bei der Vorverarbeitung wurden einige Probleme des Nutzer-Standortes beseitigt.
				Insbesondere können Datenwerte separat voneinander betrachtet werden.

				Die so erzeugten potenziellen geografischen Indikatoren können nun in der Georeferenz-Basis als Referenzwerte gespeichert werden.
				Mit der so erzeugten Georeferenz-Basis ist es grundsätzlich möglich eine Geolokalisierung durchzuführen. 
				Allerdings wird jedem potenziellen geografischen Indikator eine Georeferenz zugeordnet.
				Das ist grundsätzlich problematisch, da Referenzwerten die keinen geografischen Bezug aufweisen eine Georeferenz zugeordnet wird. 
				Dies kann bei der Geolokalisierung zu Fehlern führen.
				Ein Referenzwert mit geografischem Bezug wird gleich behandelt wie ein Referenzwert ohne geografischen Bezug.
				Es sollte entschieden werden können ob der Referenzwert einen geografischen Bezug hat oder nicht.
				Bisher existiert noch kein Hinweis auf den geografischen Bezug.
				Es existiert lediglich eine Datenbasis die Referenzwerten eine Georeferenz zuweist. 

				Des weiteren können zu einem Referenzwert beliebig viele Datensätze existieren, selbst wenn diese auf denselben Ort verweisen. 
				Bei einer Abfrage an die Datenbank werden potenziell große Mengen an Datensätzen zurückgeliefert die alle auf dieselbe Georeferenz verweisen.

	\section{Auflösen des Nutzer-Standortes mit Hilfe der Georeferenz-Basis} \label{sec:AufloesenDesNutzerStandortes} 

		\textit{Geografischer Bezug der eingelernten Referenzwerte ALT} \label{sec:geografischerBezug} 
			
			Die eingelernten Referenzwerte beinhalten alle Werte aus den Nutzer-Standorten der Tweet-Lerndaten.
			Es sind also auch Referenzwerte vorhanden die keinen geografischen Bezug haben.
			Es ist die Frage zu beantworten: Wie kann bestimmt werden ob ein Referenzwert geografischen Bezug hat oder nicht?
			Oder: Wie kann vermieden werden, dass ein Referenzwert, der keinen geografischen Bezug hat, zur Geolokalisierung genutzt wird?
			
			Dies ist wichtig, denn durch die Referenzwerte wird in der eigentlichen Geolokalisierung einem geografischen Indikator eine Georeferenz zugewiesen. 
			Wird einem geografischen Indikator durch einen Referenzwert ohne geografischen Bezug eine Georeferenz zugewiesen ist diese mit hoher Wahrscheinlichkeit fehlerhaft.
			Dies wiederum führt zu schlechten und unzuverlässigen Ergebnissen.
			Es muss also ein Verfahren gefunden werden um zu entscheiden ob die Referenzwerte einen geografischen Bezug haben oder nicht.

			Es ist zu beachten, dass die Vorverarbeitung keine Aussage zum geografischen Bezug macht, sondern vielmehr die Referenzwerte aus den Nutzer-Standorten extrahiert. 
			Dies soll sicherstellen, das möglichst viele Informationen aus den Nutzer-Standorten gezogen werden können und insbesondere keine Informationen verloren gehen.  
			
			Um zu entscheiden ob ein Referenzwert geografischen Bezug hat oder nicht wird die absolute Häufigkeit verwendet.
			Die absoluten Häufigkeiten geben an wie oft ein Referenzwert in einer bestimmten Region, zunächst in der Voronoi-Region der entsprechenden Stadt, vorkommt.			 
			Daraus kann nun ein geografischer Bezug abgeleitet werden.

			In Kapitel \ref{sec:einlernen} wurde ein Verfahren zum einlernen der Georeferenz-Basis vorgestellt. 
			Im darauffolgenden Kapitel \ref{sec:geografischerBezug} wurde eine Möglichkeit vorgestellt wie der geografische Bezug der Referenzwerte untersucht werden kann.
			Dies soll hier genutzt werden um eine Georeferenz zu bestimmen. 

			Es soll nun das Verfahren zur Geolokalisierung eines Twitter-Nutzers vorgestellt werden.
			Dabei dient die Georeferenz-Basis und die in ihr abgelegten Werte als Basis für die Zuweisung einer Georeferenz.

			Zunächst werden aus dem Nutzer-Standort und der Nutzer-Zeitzone potenzielle geografische Indikatoren erzeugt.
			Dies geschieht analog zur Erzeugung von Referenzwerten beim einlernen der Georeferenz-Basis.
			Es werden also dieselben Vorverarbeitungsschritte für den Nutzer-Standort und die Nutzer-Zeitzone durchgeführt.
			Daraus resultiert eine Menge potenzieller geografischer Indikatoren.
			Die Werte der potenziellen geografischen Indikatoren werden nun in der Georeferenz-Basis nachgeschlagen. 
			Dabei werden alle Datensätze deren Referenzwerte mit den potenziellen geografischen Indikatoren korrespondieren zurückgegeben.
			Auf diesen Datensätzen erfolgt die weitere Verarbeitung und Bestimmung der wahrscheinlichsten Georeferenz.
			
			Es liegt nun eine Menge an Datensätzen aus der Georeferenz-Basis vor.
			Die Referenzwerte entsprechen dabei den potenziellen geografischen Indikatoren.
			Die absoluten und relativen Häufigkeiten werden für jeden Referenzwert separat analysiert.
			Das Ziel der Analyse ist es, die wahrscheinlichste Georeferenz zu ermitteln.

			In Abbildung \ref{img:ablaufGeolok} ist der gesamte Ablauf an einem Beispiel dargestellt.
			In den folgenden Abschnitten soll nun die Analyse genauer betrachtet werden.

				\begin{figure} 
				\begin{center}
							\includegraphics[scale=0.5]{geolokalisierungAllg.pdf}
							\caption{Ablauf der Geolokalisierung mit Beispiel}
							\label{img:ablaufGeolok}
						\end{center}
				\end{figure}	

			Grundsätzliche Idee des auflösens mit absoluten Häufigkeiten
			Hier Idee skizzieren. 
			Schwellwert für den absoluten Häufigkeitswert zur Justierung der Konfidenzen.

		\subsection{Auflösen des Nutzer-Standortes basierend auf absoluten Häufigkeiten} 

			\todo{chap: LsgAnsatz sec: Auflösen subsec: abs. Hauf. Komplette Einführung + Ablauf + Auswertung} 

			\subsubsection{Ablauf}

			\subsubsection{Auswertung}  

				\textit{Die absolute Häufigkeit als Hinweis auf geografischen Bezug zu Städten} 
				
				Eine hohe absolute Häufigkeit kann ein Hinweis auf den geografischen Bezug eines Referenzwertes darstellen. 
				Aufgrund der Eigenschaften des Nutzer-Standorts ist anzunehmen, dass in der Voronoi-Region einer Stadt der Name der zugehörigen Stadt häufig vorkommt.
				Dadurch kann eine Relevanz des Referenzwertes zu einer Stadt abgeleitet werden. 
				Tritt der Referenzwert nicht häufig auf, so ist er von nur wenigen Nutzern als Nutzer-Standort in einer Stadt angegeben worden und somit für die Stadt nicht relevant.
				In Abbildung \ref{img:ulIstanbulWalesZoom} sind die Tweets in denen "'Istanbul"' im Nutzer-Standort vorkommt aufgetragen.
				Es ist deutlich eine Häufung um die Stadt Istanbul zu erkennen. 
				Werden die Tweets rund um Istanbul nun auf die Stadt Istanbul abgebildet, wird die Kombination aus dem Referenzwert "'Istanbul"' und der Georeferenz Istanbul sehr häufig vorkommen.
				In den Nutzer-Standorten der Tweets rund um Istanbul taucht "'Istanbul"' tatsächlich 972 mal auf.
				Damit kann eine Gewisse Relevanz für den Referenzwert "'Istanbul"' zur Stadt Istanbul abgeleitet werden.

				\begin{figure}[!ht]
						\begin{center}
							\includegraphics[scale=0.5]{ulIstanbulWalesZoom.pdf}
							\caption{}
							\label{img:ulIstanbulWalesZoom}
						\end{center}
				\end{figure}	

				Es kann also ein Schwellwert für die Häufigkeit eingeführt werden um Referenzwerte mit geografischem Bezug zu identifizieren.
				Allerdings garantiert die absolute Häufigkeit noch nicht, dass ein Referenzwert einen geografischen Bezug hat. 
				Es können auch Werte an einem bestimmten Ort häufig vorkommen, die keinen geografischen Bezug haben.
				Um dies erkennen zu können muss ein weiterer Wert berechnet werden.

		\subsection{Probleme bei der ausschließlichen Betrachtung der absoluten Häufigkeiten} 

			Betrachtet man die absoluten Häufigkeiten isoliert voneinander wird die Verteilung des Referenzwertes außer acht gelassen. 
			Ein Referenzwert kann eine gleichmäßige Verteilung über mehrere Städte aufweisen.
			Das bedeutet, der Referenzwert wird in vielen unterschiedlichen Städten benutzt. 
			Es können trotzdem Häufungen in Städten auftreten.
			Diese können sogar über einem gewählten Schwellwert für die absolute Häufigkeit liegen.
			Die Häufung kann jedoch relativ gesehen sehr gering sein.
			Dies ist ein Hinweis darauf das der Referenzwert keinen geografischen Bezug hat.
			Die relative Häufung sagt hier aus, das ein Referenzwert sehr verteilt auftritt. 
			Es ist also wichtig nicht nur die absoluten Häufigkeiten, sondern auch die relativen Häufigkeiten der Referenzwerte zu berücksichtigen.

			Problem am Beispiel und mit Daten.

			Einbeziehung der relativen Häufigkeiten. 
			Aufgrund von Beispielbetrachtung The und La Plata. Absolut betrachtet.


			"'La Plata"' tritt rund um die Stadt La Plata in Argentien 91 mal auf.
			Der Referenzwert "'La Plata"' hat offensichtlich einen geografischen Bezug zu einer Stadt. 

			Obwohl "'the"' keinen offensichtlichen geografischen Bezug hat ist die absolute Häufigkeit von 91 Vorkommen in Jakarta hoch.
			Der Schwellwert könnte nun aufgrund der Erfahrung mit dem Referenzwert "'La Plata"' auf 90 angesetzt werden.
			Dann würde davon ausgegangen werden, dass der Referenzwert "'the"' einen geografischen Bezug hat.
			Betrachtet man allerdings die Abbildung \ref{img:ULThe} fällt auf, dass Tweets mit dem Wert "'the"' im Nutzer-Standort sehr verteilt auf dem Globus auftreten.
			Im Gegensatz dazu tritt "'La Plata"' in den Nutzer-Standorten sehr konzentriert auf.
			In Abbildung \ref{img:ULlaPlata} wird die Verteilung von Tweets deren Verfasser "'La Plata"' im Nutzer-Standort enthalten dargestellt. 
			Es ist deutlich eine Häufung um die Stadt La Plata in Argentien zu erkennen, weltweit tritt der Referenzwert aber sehr selten auf.
			\begin{figure} 
				\begin{center}
					\includegraphics[scale=0.6]{ulTheG.pdf}
					\caption{Tweets mit Nutzer-Standort "'The"'}
					\label{img:ULThe}
					\end{center}
				\end{figure}
			\begin{figure}
			\begin{center}
					\includegraphics[scale=0.5]{ulLaPlataG.pdf}
					\caption{Tweets mit Nutzer-Standort "'La Plata"'}
					\label{img:ULlaPlata}
				\end{center}
			\end{figure}		

		\subsection{Auflösen des Nutzer-Standortes basierend auf absoluten und relativen Häufigkeiten} 

			Idee skizzieren
			\todo{chap: LsgAnsatz sec: Auflösen subsec: abs. Hauf. Komplette Einführung + Beispiel überarbeiten} 

			\textit{Berechnung der relativen Häufigkeiten}  

				Um die relativen Häufigkeiten zu berechnen soll das Vorkommen eines Referenzwertes in einer Stadt, durch die Gesamtanzahl der Vorkommen des Referenzwertes geteilt werden.
				Damit erhält man den prozentualen Anteil der auf eine Stadt entfallenden Vorkommen eines Referenzwertes.
				Als Basis für diese Berechnung dienen die absoluten Häufigkeiten.

				Sei $(r_i,c_j)$ ein Datensatz der Georeferenz-Basis mit Referenzwert $r_i$ und Georeferenz $c_i$.
				Des weiteren liefert $H(r_{i},c_{j})$ die absolute Häufigkeit zu einem Referenzwert $r_i$ und einer Georefrenz $c_i$. 

				Damit kann die relative Häufigkeit $rel_{(r_i,c_j)}$ für jede Kombination $(r_i,c_j)$ durch die folgende Formel berechnet werden. 
				$n_c$ ist dabei die Anzahl aller Georeferenzen.

				\begin{equation}
					d_{r_i,c_j}=\frac{H(r_i,c_j)}{\sum^{n_c}_{j=0}{H(r_i,c_j)}}
				\end{equation}	

			\subsubsection{Ablauf}

			\subsubsection{Lösung des Beispielproblems von oben}

				Berechnet man nun die relativen Häufigkeiten kann dies abgebildet werden.
				In Tabelle \ref{tab:the} sind die zugeordneten Städte, die absoluten Häufigkeiten und die berechneten relativen Häufigkeiten für die Vorkommen des Wortes "'the"' aufgetragen. 
				Die Einträge sind absteigend nach dem Wert der relativen Häufigkeit sortiert.
				In Tabelle \ref{tab:the} werden nur die vier ersten Einträge dargestellt.  
				Insgesamt kam das Wort "'the"' in 2824 verschiedenen Städten vor.
				Dabei wurde es insgesamt 5764 mal verwendet. 
				Trotz der hohen absoluten Häufigkeit in Jakarta liegt die relative Häufigkeit bei lediglich 1,6\%.
				Aus den relativen Häufigkeiten lässt sich nun die, in Abbildung \ref{img:ULThe} vermutete, globale Verteilung ablesen.


				\begin{table}[h]
				\centering
				\caption{"'the"'}
				\label{tab:the}
				\begin{tabular}{|l|l|l|}
				\hline
				Stadt             & abs. Häufigkeit & rel. Häufigkeit in \% \\ \hline \hline
				Jakarta           & 91              & 1,6                       \\ \hline
				Singapore         & 27              & 0,5                       \\ \hline
				Bekasi            & 25              & 0,4                       \\ \hline
				Philadelphia      & 23              & 0,4                       \\ \hline
				... & ... & ... \\ \hline
				\end{tabular}
				\end{table}

				Die Verteilung erklärt sich dadurch, dass das Wort "'the"' im englischen sehr häufig auftritt.
				Englisch ist die internationale Verkehrssprache und wird dementsprechend global und sehr häufig verwendet.
				Dies spiegelt sich in der Verteilung der vorkommen auf dem Globus wieder.
				Die Häufung um Jakarta kann teilweise damit erklärt werden, dass sehr viele Tweets die in und um Jakarta abgesetzt werden eine Angabe von Längen- und Breitengrad aufweisen. 

				In Tabelle \ref{tab:laPlata} wird dieselbe Auswertung für "'La Plata"' dargestellt. 
				"'La Plata"' ist insgesamt 129 mal in 23 verschiedenen Städten aufgetaucht. 
				In der Stadt La Plata, in Argentien, kam es 91 mal vor.
				Dies entspricht einer relativen Häufigkeit von 70,5\%.

				\begin{table}[h]
				\centering
				\caption{"'La Plata"'}
				\label{tab:laPlata}
				\begin{tabular}{|l|l|l|}
				\hline
				Stadt            & abs. Häufigkeit & rel. Häufigkeit in \% \\ \hline \hline
				La Plata         & 91              & 70,5                      \\ \hline
				Villa Gesell     & 9               & 7,0                       \\ \hline
				Mar del Plata    & 5               & 3,9                       \\ \hline
				Quilmes          & 3               & 2,3                       \\ \hline
				... & ... & ... \\ \hline
				\end{tabular}
				\end{table}

				"'La Plata"' taucht erwartungsgemäß am häufigsten rund um die Stadt La Plata auf.  

				Es lässt sich also anhand der relativen Häufigkeiten ein geografischer Bezug des Referenzwertes nachweisen.
				Damit kann man zunächst bestimmen ob ein Referenzwert geografischen Bezug hat oder nicht.
				Die relativen Häufigkeiten können nach dem einlernen der Referenzwerte berechnet und in der Georeferenz-Basis hinterlegt werden.
				Mit Hilfe eines Schwellwertes für die relativen Häufigkeiten, kann nun bestimmt werden wann ein Referenzwert geografischen Bezug hat.

				Aber auch die ausschließliche Betrachtung der relativen Häufigkeiten reicht nicht aus um die geografische Relevanz nachzuweisen.
				Da die relativen Häufigkeiten basierend auf den Vorkommen des Referenzwertes berechnet werden sagen diese wiederum nichts über die absoluten Häufigkeit aus. 
				Kommt ein Referenzwert zwei mal in zwei verschiedenen Städten vor, liegen die jeweiligen relativen Häufigkeiten bei 50\%.
				Dies ist ein hoher Wert.
				Da der Referenzwert aber nur einmal vorkam, ist es sehr unwahrscheinlich das er einen geografischen Bezug hat.  

		\subsection{Einführung von Schwellwerten für die absolute und relative Häufigkeit zur Justierung der Ergebnisse}  

			\todo{chap: LsgAnsatz sec: Auflösen subsec: Schwellwerte für abs. und rel. einführen komplett schreiben} 

	\section{Ausnutzen der geografischen Hierarchie zur Verbesserung der Ergebnisse} \label{sec:ausnutzenDerGeografischenHierarchie}

		\todo{chap: LsgAnsatz sec: Ausnutzen Hierarch subsec: Idee skizzieren und Splitten -> mergen} 
		Idee skizzieren

		\textit{Geografischer Bezug zu Verwaltungseinheiten und Ländern} 

			Beim einlernen der Georeferenz-Basis werden durch die absoluten Häufigkeiten die Vorkommen pro Stadt gespeichert.
			Mit den daraus errechneten relativen Häufigkeiten kann nicht entschieden werden, ob für den Referenzwert eine globale Verteilung vorliegt oder ob der Referenzwert unter Umständen nur regional begrenzt, zum Beispiel in einem Land, auftritt.
			Für Referenzwerte die ein Land oder eine Verwaltungseinheit bezeichnen ist auf Städteebene eine geringe relative Häufigkeit zu erwarten.
			Diese Referenzwerte treten in einer größeren geografischen Region als der Voronoi-Region einer Stadt auf.
			Sie sind somit über mehrere Städte verteilt und werden auf Stadtebene einer geringe relative Häufigkeit aufweisen.
			Es ist beispielsweise zu erwarten das ein Ländername in den Nutzer-Standorten von Tweets aus dem gesamten Land auftritt.
			Durch die Unterteilung des Landes in Stadtgebiete wird der Wert sehr verteilt auf die Städte des Landes auftreten.

			Soll nun statt einer Stadt eine Verwaltungseinheit oder das Land als Georeferenz bestimmt werden, kann mit diesen absoluten Häufigkeiten auf Stadtebene keine Aussage über den geografischen Bezug gemacht werden. 
			
			Am folgenden Beispiel soll dieser Sachverhalt erläutert werden.

			In Abbildung \ref{img:ulIstanbulWalesZoom} sind in grün Tweets dargestellt, welche im Nutzer-Standort "'Wales"' enthalten.
			Wales entspricht einer Verwaltungseinheit erster Ordnung und gehört zu Großbritannien.
			Die Tweets sind in der gesamten geografischen Region, über die sich Wales erstreckt, verteilt.
			Außerhalb von Wales tritt "'Wales"' im Nutzer-Standort sehr selten auf.
			Die relativen Häufigkeiten auf Stadtebene werden in Tabelle \ref{tab:walesCity} dargestellt.
			Diese bestätigen eine Verteilung über eine größere geografische Region.
			Anhand der relativen Häufigkeiten kann aber nicht entschieden werden ob der Referenzwert global verteilt ist, oder in einer bestimmten geografischen Region, wie einem Land, auftritt. 
			"'Wales"' taucht in insgesamt 78 Städten 346 mal in Nutzer-Standorten auf.

			\begin{table}[h]
			\centering
			\caption{"'Wales"'}
			\label{tab:walesCity}
			\begin{tabular}{|l|l|l|}
			\hline
			Stadt      & abs. Häufigkeit & rel. Häufigkeit in \% \\ \hline \hline
			Cardiff    & 44 			 & 12,7 \\ \hline
			Newport    & 32 			 & 9,2  \\ \hline
			Carmarthen & 24 			 & 6,9  \\ \hline
			Swansea    & 18 			 & 5,2  \\ \hline
			...    & ... & ...  \\ \hline
			\end{tabular}
			\end{table}

			Die relative Häufigkeit von 12,7\% deutete eher darauf hin, dass "'Wales"' keinen geografischen Bezug aufweist.
			Auf Städteebene ist dies auch durchaus korrekt. 
			Allerdings kann aus diesen Ergebnissen kein geografischer Bezug zu einer der anderen geografischen Hierarchieebenen abgeleitet werden.
			Das Problem ist, dass Wales keinen geografischen Bezug zu einer Stadt aufweist, wohl aber zu einer Verwaltungseinheit erster Ordnung und daher zu einer geografischen Region. 

			Um dieses Problem lösen zu können müssen zu einem Referenzwert die relativen Häufigkeiten für die anderen geografischen Hierarchieebenen berechnet werden.
			Damit kann dann die Verteilung der Referenzwerte auf diese Hierarchieebenen betrachtet werden.
			
		\textit{Berechnung der relativen Häufigkeiten zu Verwaltungseinheiten und Ländern ----->>>> nach unten MERGEN } 

			Da zu jedem Referenzwert die zugehörigen Verwaltungseinheiten und Länder bekannt sind können die absoluten und relativen Häufigkeiten direkt aus der Georeferenz-Basis berechnet werden.

			Es müssen lediglich die absoluten und relativen Häufigkeiten aufsummiert werden, bei denen der Wert der jeweiligen geografischen Hierarchieebenen übereinstimmen.
			Im Beispiel aus Tabelle \ref{tab:walesCity} müssen alle absoluten und relativen Häufigkeiten derjenigen Städte aufsummiert werden, die in derselben Verwaltungseinheit erster Ordnung liegen.
			Betrachtet man die Verwaltungseinheiten zu allen Städten in denen "'Wales"' im Nutzer Standort vorkommt ergibt sich folgende Liste.

			\begin{enumerate}
				\item Wales 35
				\item England 30
				\item unterschiedliche Verwaltungseinheiten 13
			\end{enumerate}

			Aus dieser Betrachtung alleine lässt sich noch nicht entscheiden ob der Referenzwert "'Wales"' einen geografischen Bezug zu einer Verwaltungseinheit hat.
			Denn der Referenzwert kam sowohl in 30 Städten in England als auch in 30 Städten in Wales vor, was keinen signifikanten Unterschied darstellt. 
			Summiert man allerdings die Vorkommen und relativen Häufigkeiten pro Stadt auf ergibt sich daraus Tabelle \ref{tab:WalesVerw1}.

			\begin{table}[h]
			\centering
			\caption{"'Wales"'}
			\label{tab:WalesVerw1}
			\begin{tabular}{|l|l|l|}
			\hline
			Adm1 & abs. Häufigkeit & rel. Häufigkeit in \% \\ \hline \hline
			Wales                   & 298 & 86,1 \\ \hline
			England                 & 38  & 11,0 \\ \hline
			National Capital Region & 1   & 0,3  \\ \hline
			Stockholm               & 1   & 0,3  \\ \hline
			... & ... & ... \\ \hline
			\end{tabular}
			\end{table}  

			Die relative Häufigkeit von 86,1\% weißt nun deutlich darauf hin, dass der Referenzwert einen geografischen Bezug zu Wales hat.
			Mit diesem Vorgehen, kann der geografische Bezug eines Referenzwertes auf jeder der geografischen Hierarchieebenen untersucht werden.

			Analog können die Werte für die Verwaltungseinheit zweiter Ordnung und dem Land berechnet werden.
			Dieses Vorgehen ermöglicht es einen geografischen Bezug eines Referenzwertes auf jeder der geografischen Hierarchieebenen zu prüfen.

		\textit{Fazit}

			Mit der absoluten Häufigkeit besteht ein erster Hinweis darauf ob ein Referenzwert geografischen Bezug hat oder nicht.
			Die alleinige Betrachtung der absoluten Häufigkeit lässt aber die Verteilung der Werte auf den jeweiligen geografischen Hierarchieebenen außer betracht.
			Mit den berechneten relativen Häufigkeiten können die Referenzwerte zusätzlich auf ihre Verteilung untersucht werden.
			Die relativen Häufigkeiten können nach dem Einlernen berechnet und in der Georeferenz-Basis gespeichert werden. 
			Während der Geolokalisierung kann dies absolute und relative Häufigkeit genutzt werden um die geografischen Indikatoren zu bestimmen. 

		\subsection{Hochziehen und summieren ------ letzte drei abschnitte hier rein mergen ---------}

		\subsection{Anpassung der Schwellwerte -------- letzte drei abschnitte hier rein mergen ---------}

		\subsection{Wahl der Schwellwerte zur Justierung der Genauigkeit und der Trefferquote}

			Die Wahl der beiden Schwellwerte ist abhängig von den Anforderungen.
			Dabei ist die gewünschte geografische Hierarchie der zurückgegeben Georeferenz ein Faktor.
			Und die gewünschte Genauigkeit und Trefferquote.
			
			\paragraph{Gewünschte Hierarchieebene der Georeferenz}

				Umso größer die betrachtete geografische Region ist, umso weniger Möglichkeiten zur Einteilung gibt es.
				Auf Städteebene gibt es 23322 verschiedene geografische Regionen. 
				Für jede Stadt mit mehr als 15000 Einwohnern existiert dabei eine Region.
				Diese Städte verteilen sich auf 234 verschiedene Länder.
				Dieselbe Menge an Referenzwerten, verteilt sich auf Länderebene also auf weniger geografische Regionen. 
				Dadurch werden die Werte der relativen Häufigkeit und der absoluten Häufigkeit insgesamt größer.
				Relativ zueinander werden allerdings nach wie vor Referenzwerte mit geografischem Bezug größer sein als Referenzwerte ohne geografischen Bezug.

				Es müssen also für jeder geografsiche Hierarchieebene geeignete Schwellwerte $s_{rel}$ und $s_{abs}$ gefunden werden.

			\paragraph{Genauigkeit und Trefferquote} 

				Der zweite Faktor ist die gewünschte Trefferquote und die Genauigkeit.

				Umso niedriger der Schwellwert $s_{rel}$ ist, umso größer wird die Wahrscheinlichkeit Referenzwerte zu wählen die keinen geografischen Bezug haben.
				Daraus resultieren mehr fehlerhafte Zuordnungen einer Georeferenz.
				Wodurch die Genauigkeit schlechter wird.
				Dadurch können allerdings mehr Georeferenzen zugeordnet werden, wordurch die Trefferquote verbessert wird.
				Umso höher der Schwellwert $s_{rel}$ gewählt wird umso mehr Referenzwerte mit geografischem Bezug werden verworfen.
				Die Wahrscheinlichkeit, dass die gewählten Referenzwerte tatsächlich geografischen Bezug haben ist allerdings höher.
				Dadurch können weniger Georeferenzen zugeordnet werden.
				Somit sinkt die Trefferquote.
				Allerdings sind die zugewiesenen Georeferenzen sicherer womit die Genauigkeit steigt.

				Der Schwellwert $s_{abs}$ vermeidet, dass Refrenzwerte gewählt werden die eine hohe realtive Häufigkeit aufweisen aber aufgrund ihrer geringen Vorkommen nicht relevant sind.
				Die Auswirkungen der Wahl des Schwellwertes $s_{abs}$ verhalten sich Analog zum Schwellwert $s_{rel}$.

			\paragraph{Fazit}

				Die Wahl der Schwellwerte hängt zum einen von der Hierarchieebene und zum anderen von den Anforderungen an die Genauigkeit und die Trefferquote ab.
				In Bezug auf die geografischen Hierarchieebenen sind lediglich separate Schwellwerte für jede geografischen Hierarchieebenen zu bestimmen, da die relativen und absoluten Häufigkeiten sich insgesamt verändern.
				Bezüglich der Genauigkeit und der Trefferquote ist ein Kompromiss zwischen den beiden Werten einzugehen. 
				Die Verbesserung der Trefferquote geht mit einer Verschlechterung der Genauigkeit einher und umgekehrt.
				Es kann also entweder ein Kompromiss gefunden werden der ein Optimum für beide Werte darstellt. 
				Oder einer der Werte wird optimiert.  

	\subsection{Analyse SCHRITTE ALT, AUFSPLITTEN UND OBEN REIN MERGEN EINMAL IN ABS AUFLÖSEN EINMAL RELATIVE AUFLÖSEN}

			Die Analyse beinhaltet zwei Schritte. 
			Zuerst müssen diejenigen Referenzwerte gewählt werden, welche am wahrscheinlichsten einen geografischen Bezug haben.
			Dabei wird jeder Referenzwert separat betrachtet. 
			In einem nächsten Schritt wird derjenige Referenzwert gewählt, der unter den verbliebenen am wahrscheinlichsten die geografische Position des Nutzers beschreibt. 

			\subsubsection{Auswahl der Referenzwerte mit geografischem Bezug}

				Es können pro Referenzwert zunächst mehrere Datensätze vorliegen.
				Aus diesen sollen diejenigen gewählt werden, welche am wahrscheinlichsten einen geografischen Bezug aufweisen.
				Dazu werden sowohl die absoluten als auch die relativen Häufigkeiten genutzt. 
				Für jeden Referenzwert wird derjenige Datensatz gewählt, der die größte relative Häufigkeit $h_{rel}$ über einem Schwellwert $s_{rel}$ aufweist.
				Mit diesem Schwellwert lässt sich bestimmen wie verteilt der Referenzwert auftreten kann.
				Zusätzlich wird geprüft ob die absolute Häufigkeit ebenfalls über einem Schwellwert $s_{abs}$ liegt.
				Mit diesem Schwellwert lässt sich bestimmen wie häufig der Referenzwert an einer geografischen Position oder in einer geografischen Region auftreten muss. 

			\subsubsection{Bestimmung der wahrscheinlichsten Georeferenz} 

				Nun liegt wiederum eine Menge an Datensätzen vor.
				Jeder Referenzwert, und damit auch jeder potenzielle geografische Indikator, taucht nur noch ein mal auf. 

				Aus den verbliebenen Datensätzen soll nun die Georeferenz gewählt werden. 
				Dazu werden die relativen Häufigkeiten verglichen.
				Es wird der Datensatz mit der höchsten relativen Häufigkeit gewählt.
				Die Georeferenz dieses Datensatzes wird dann dem Twitter-Nutzer zugewiesen. 
				Damit wird der Referenzwert ausgewählt der die größte relative Häufigkeit aller untersuchten Referenzwerte aufweist. 

				Die Referenzwerte stellen NGramme dar, wie in der Vorverarbeitung in Unterkapitel \ref{subsec:VorverarbeitungStandortZeitzone} erläutert wird.
				Es werden hier also insbesondere auch Uni-, Bi- und Trigramme miteinander verglichen.
				Darauf soll nun eingegangen werden.

				\paragraph{Vergleich der relativen Häufigkeiten zu Uni- Bi- und Trigrammen}

					Jedes Element eines Bi- oder Trigrammes kann potenziell einen geografischen Bezug haben. 
					Umso mehr Elemente ein NGramm beinhaltet umso spezieller kann die Beschreibung des geografischen Objekts sein.
					Deshalb können NGramme mit einem höheren Grad ein Objekt genauer beschreiben als NGramme mit einem niedrigeren Grad.

					Allerdings können NGramme mit einem höheren Grad auch eine schlechtere Beschreibung darstellen. 
					Beispielsweise wenn das zusätzliche Element keinen geografsichen Bezug hat.

					Bei NGrammen mit einem Grad größer zwei können also zwei Fälle unterschieden werden.

					\begin{enumerate}
						\item Die Kombination aus den Elementen des NGrammes beschreibt einen Ort genauer
						\item Die Kombination aus den Elementen des NGrammes beschreibt einen Ort nicht genauer
					\end{enumerate}

					\subparagraph{Fall1} 

						Ein Beispiel für den ersten Fall ist der Nutzer-Standort "'york"' mit der Nutzer-Zeitzone "'eastern+time+us+canada"'. 
						Durch die Vorverarbeitung werden folgende potenzielle geografische Indikatoren erzeugt.
						\begin{enumerate}		
							\item york
							\item york \textit{eastern+time+us+canada}
						\end{enumerate}		

						Eine Stadt Namens York existiert sowohl in Grossbritannien als auch in den USA.
						Fragt man nun die beiden Referenzwerte in der Georeferenz-Basis ab erhält man folgende Werte:

							\begin{table}[h]
								\centering
									\caption{"'york"'}
									\label{tab:york}
									\begin{tabular}{|l|l|l|l|}
									\hline
									Referenzwert 				& Stadt  	& abs. Häufigkeit & rel. Häufigkeit in \% \\ \hline \hline
									York          				& York (GB) & 97              & 48,3       \\ \hline
									york eastern+time+us+canada & York (US) & 12              & 63,2        \\ \hline
									\end{tabular}
							\end{table}

							Die realtive Häufigkeit für york in Kombination mit der Zeitzone ist höher. 
							Die Zeitzone gibt zusätzliche Auskunft darüber welches York gemeint ist. 
							Die Kombination ist spezieller, kommt deshalb seltener vor und potenziell eher dort wo sie zutrifft. 
							In diesem Fall in York in den USA. 

							In den meisten Fällen beschreibt einer der beiden Indikatoren eine größere geografische Region wie beispielsweise einen Bundesstaat der USA.
							Wird ein weiterer Wert, beispielsweise ein Städtename hinzugenommen, wird die Angabe des Ortes genauer. 
							Die Wahrscheinlichkeit, dass diese Kombination ausserhalb des Ortes auftritt wird geringer. 

					\subparagraph{Fall 2}

						Hier können wiederum 2 Fälle unterschieden werden.

						\begin{enumerate}
							\item Beide Elemente beziehen sich auf unterschiedliche geografische Objekte
							\item Nur ein Element hat geografischen Bezug das andere nicht 
						\end{enumerate}

						Wenn zu einem Referenzwert mit geografischem Bezug ein Element hinzugefügt wird, welches keinen geografischen Bezug hat, beschreibt dies den Ort nicht genauer.
						Es ist zu erwarten, dass die Kombination der Elemente sehr selten vorkommt oder sehr verteilt ist. 
						Ist die Kombination verteilter, so ist der relative Wert geringer als der des einzelnen Referenzwertes mit geografischem Bezug.
						Ist die Kombination seltener kann der Referenzwert bereits durch den Schwellwert $s_{abs}$ aussortiert werden.		

						Wenn zu einem Referenzwert mit geografischem Bezug ein Element hinzugefügt wird, welches zwar geografischen Bezug hat, aber dieses sich auf ein anderes geografisches Objekt bezieht ist dasselbe Verhalten zu erwarten.

	\newpage
	
	
			
		



	