%!TEX root = ../document.tex
\chapter{Entwicklung einer Methode zur Ortsbestimmung von Social Media Daten in Abwesenheit geografischer Koordinaten oder anderer konkreter Ortsangaben} 

\section{Indikatoren zur Ortsbestimmung}

	\todo{u.U. "'unmittelbar und mittelbar geografische Indikatoren"' "'Entwurf"' -> "'Grundlagen und Stand der Technik verschieben"'} 
	\subsection{unmittelbar geografische Indikatoren}
		\begin{enumerate}
			\item Mögliche Alternativen
			\item Begründung warum Userlocation und Timezone
			\item Beispiele und Auswertungen (manuell getaggter Datensatz)
			\item \cite{Hecht:2011:TJB:1978942.1978976} 
		\end{enumerate}

	\subsection{mittelbar geografische Indikatoren}
		\begin{enumerate}
			\item bspsw. Hashtags, Inhaltsanalysen ohne spezielle geografische Hinweise, 
		\end{enumerate}

	\todo{Wie detailliert hier auf Framework eingehen? Präprozessor-Konzept zur universellen Vorverarbeitung, oder eher in Implementierung}
	\subsection{Vorverarbeitung der Indikatoren (Präprozessor-Konzept)}
		\begin{enumerate}
			\item geonames matching (geonames tree) für geografische Namen bestehend aus mehreren Wörtern
			\item Eliminierung von Sonderzeichen
			\item Tokenizing
			\item Ngram Erzeugung allgemein
			\item \todo{Was bringt die Zeitzone als zusätzlicher Indikator? Verbesserun messen} Zeitzone als "'schärfenden Indikator für doppeldeutige Namen"'
		\end{enumerate}

	\subsection{Encoding}
		Problematik unterschiedlicher Sprachen, 
		url-encoding sinnvoll als Vorbereitung auf Webservice. 

\section{Geolocation Mapping}

	\subsection{nearest neighbour mapping}
		\begin{enumerate}
			\item \todo{Welches Fehlermaß kann für das mapping angewandt werden? auf Städteebene gut möglich mit geografischen Distanzen, admin2,amdin1, Land schlecht möglich mit Distanzen} Wie genau kann gemappt werden? Fehler Durchschnitt. 
			\item Mapping auf cities 1000/1000/15000 mit Daten zu durchschnitll. Abstand
			\item Hier ist noch Verbesserungspotenzial -> wenn Mapping Distanz zu weit entfernt -> verwerfen! 
		\end{enumerate} 

\section{Verknüpfung von Indikatoren und geografischen Lokationen zur wiedergewinnung des erlernten Wissens}
	
	\subsection{Generierung eines Wissendatensatzes}

	\subsection{Verknüpfung mit Geodaten}

	\subsection{Auflösen auf Administartionsebenen, Länder}

\section{Lokalisieren von Tweets ohne konkrete geografische Daten}
	
	\subsection{Ablauf der Lokalisierung}
	
	\subsection{Lokalisierungssicherheit durch Ausnutzung der geografischen Hierarchiebeziehungen}





