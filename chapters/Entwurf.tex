%!TEX root = ../document.tex
\chapter{Lösungsansatz} \label{chp:Loesungsansatz}
In diesem Kapitel wird ein Verfahren zur Lokalisierung von Twitter-Nutzern vorgestellt.
Die Fragestellungen aus Kapitel \ref{sec:fragestellung} werden, unter Berücksichtigung der Anforderungen aus Kapitel \ref{sec:Anforderungen}, beantwortet.  

Zunächst soll ein Überblick über die Funktion und den Ablauf des Verfahrens gegeben werden ohne detailliert auf die einzelnen Verfahrensschritte einzugehen.
Danach wird das Verfahen von Grundauf betrachtet und die einzelnen Verfahrensschritte eingehender erläutert. 

	\section{Überblick} 

	Das erarbeitete Verfahren soll es ermöglichen Twitter-Nutzern, deren geografische Position unbekannt ist, eine Georeferenz zuzuordnen.
	Dabei sollen als Eingabe für die Georeferenzierung lediglich die Nutzer-Zeitzone und der Nutzer-Standort aus dem Profil eines Twitter-Nutzers verwendet werden.
	Als Ergebnis soll eine Georeferenz mit einem Konfidenzwert zurückgeliefert werden. 
	Dabei hat der Anwender die Möglichkeit, sowohl die Genauigkeit bezüglich der geografischen Position, als auch einen Schwellwert für die gewünschte Konfidenz der Georeferenz anzugeben.
	In Abbildung \ref{A1} ist der generelle Ablauf der Georeferenzierung dargestellt. 
	\todo{Schema Darstellung Eingabe->Ausgabe  Sketchbook A1: Unterschrift Die Eingabe besteht aus dem Nutzer-Standort sowie der Nutzer-Zeitzone. 
	Als Rahmenbedingungen wird die gewünschte Hierarchie-Ebene sowie der Schwellwert für die Konfidenz angeben. 
	Als Ausgabe erhält man eine Georeferenz und einen Konfidenzwert, der angibt wie sicher das Ergebnis ist.} 
	\label{A1} 

	Die Georeferenzierung nach dem Schema aus Abbildung \ref{A1} setzt voraus, dass aus den eingegebenen Indikatoren eine Georeferenz abgeleitet werden kann.
	Da es sich beim Nutzer-Standort und der Nutzer-Zeitzone um unmittelbare geografische Indikatoren handelt, lässt sich ein geografischer Bezug aus diesen Indikatoren ableiten.      
	Die Idee besteht nun darin aus einer umfangreichen Tweet-Datensammlung die Zuordnung der eingegebenen Indikatoren zu geografischen Positionen zu lernen und diese in einer Datenbasis zu speichern.
	Diese Datenbasis soll im folgenden als Georeferenz-Basis bezeichnet werden.
	Damit dies realsiert werden kann, muss die Tweet-Datensammlung pro Datensatz den Nutzer-Standort, die Nutzer-Zeitzone und eine Georeferenz in Form von Längen- und Breitengrad beinhalten.  \footnote{Siehe Kapitel \ref{chp:Einleitung} genutzte Daten}

	
	Der Wert des Nutzer-Standortes ist nicht objektiv und unzuverlässig, dies macht eine Analyse und eine entsprechende Vorverarbeitung dieses Indikators nötig.
	Zur Erstellung der Georeferenz-Basis werden deshalb zunächst einige Vorverarbeitung der Indikatoren durchgeführt.
	Durch die Vorverarbeitung wird der eingegebene Nutzer-Standort genauer analysiert, zusätzliche Informationen extrahiert und in eine einheitliche Form gebracht.
	Danach wird mit Hilfe des Längen- und Breitengrades und einer Geografie-Datenbank jedem Tweet eine konkrete Instanz der untersten Ebene der geografischen Hierarchie zugeordnet.
	Legt man die in Abschnitt \ref{chp:Grundlagen} vorgestellte geografische Hierarchie zugrunde, wird dadurch jedem Tweet eine Stadt zugeordnet.
	Durch diese Zuordnung werden jedem Tweet implizit auch die höheren Ebenen der geografischen Hierarchie zugeordnet, denn die verwendete Geografie-Datenbank kann die geografische Hierarchie abbilden.   	 
	Dieser Schritt markiert den Übergang von einer kontinuierlichen Darstellung der Georeferenz, durch Längen- und Breitengrad, in eine diskrete Darstellung , durch eine Stadt. 
	Die so verarbeiteten Datensätze werden nun einer simplen statistischen Auswertung unterzogen, dessen Resultat die Georeferenz-Basis darstellt.
	Die Georeferenz-Basis beinhaltet nun die verarbeiteten Indikatoren und die zuvor zugeordnete geografische Position, in Form einer Stadt, sowie die Häufigkeit in der die Kombination Indikator/Stadt ermittelt wurde.
	Der Ablauf zur Erstellung der Georeferenz-Basis wird in Abbildung \ref{A2} dargestellt.

	Bei der Georeferenzierung werden die Indikatoren zunächst derselben Vorverarbeitung wie beim einlernen unterzogen.
	Die Vorverarbeitung ist somit ein generischer Schritt des Verfahrens, da er sowohl vor dem erzeugen der Georeferenz-Basis als auch vor der eigentlichen Georeferenzierung durchgeführt wird.
	Nach der Vorverarbeitung wird in der Georeferenz-Basis nach diesen vorverarbeitenden Indikatoren gesucht.
	Das Ergebnis dieser Suche wird unter Berücksichtigung des Konfidenzschwellwertes und der geografischen Hierarchie ausgewertet und die zugeordnete geografische Position wird ausgegeben.  
	Der Ablauf der Georeferenzierung wird in Abbildung \ref{A3} dargestellt. 

	\todo{Ablaufplan A1, A2, A3} 

	In den folgenden Kapiteln soll nun genauer auf die einzelnen Teile des Verfahrens eingegangen werden. 
	Das Verfahren wird dabei in drei Teilen behandelt, die Vorverarbeitung, das Einlernen und die Georeferenzierung.

	Zunächst soll der generische Teil des Verfahrens, die Vorverarbeitung der Indikatoren, erläutert werden.
	Dabei werden die Indikatoren genauer betrachtet um die nötigen Schritte der Vorverarbeitung zu identifizieren. \todo{eventuell nicht Analyse, könnte schwirig werden}  
	Die einzelnen Schritte der Vorverarbeitung sollen daraufhin genauer betrachtet werden.

	Beim Einlernen wird zunächst das Verfahren zur Zuordnung eines Tweets zu einer Stadt erläutert bevor die Indikatoren einem weiteren Verarbeitungsschritt unterzogen werden. 
	Danach wird die Auswertung der durch die Vorverarbeitung und die Zuordnung der Tweets entstandenen Daten erläutert.
	Damit ist das Einlernen des Verfahrens, und die damit verbundene Erstellung der Georeferenz-Basis, abgeschlossen.

	Zuletzt soll auf die eigentliche Georeferenzierung eingegangen werden.
	Die Vorverarbeitung der Indikatoren wird in diesem Teil nur angeschnitten, da diese bereits im ersten Teil genau erläutert wird. 
	Nachdem die Indikatoren vorverarbeitet wurden, wird in der Georeferenz-Basis nach korrespondierenden Indikatoren gesucht. 
	Die Ergebnisse dieser Suche werden unter Berücksichtigung des Konfidenzschwellwertes und der gewünschten georafischen Genauigkeit, in Form der geografischen Hierarchieebene, ausgewertet.
	Das Ergebnis dieser Auswertung ist die gewünschte Georeferenz. 


	\section{Vorverarbeitung}
	Zunächst soll der Nutzer-Standort als geografischer Indikator genauer betrachtet werden. 
	Insbesondere die Tatsache, dass der Wert des Nutzer-Standortes nicht objektiv und unzuverlässig ist muss beachtet werden. 
	Die Vorverarbeitung des Nutzer-Standortes wird in mehreren Schritten durchgeführt, welche einzeln erläutert werden.
	Nach den einzelnen Schritten zur Vorverarbeitung des Nutzer-Standortes wird die Nutzer-Zeitzone betrachtet.
	Welche zwar objektiv, aber auch zuverlässig ist. 
	Dabei wird der Nutzen, diese als zusätzlichen Indikator hinzuzufügen, herausgestellt und an einem Beispiel belegt.  

		\subsection{Der Nutzer-Standort als geografischer Indikator} 

		Der Nutzer-Standort ist ein unmittelbarer geografischer Indikator, es wird vom Benutzer eine geografische Angabe, nämlich der Standort des Nutzers abgefragt. 
		Es kann also zunächst davon ausgegangen werden, dass der Nutzer-Standort eine geografische Angabe darstellt.
		Allerdings ist der Wert des Nutzer-Standortes nicht objektiv.
		Diese Eigenschaft resultiert daraus, dass der Nutzer-Standort von jedem Nutzer frei eingegeben werden kann und ohne weitere Verarbeitung durch Twitter gespeichert wird. 
		Aus diesem Umstand kann auch direkt hergeleitet werden, dass der Nutzer-Standort unzuverlässig ist.
		Damit ist bei der Auswertung mit Unsicherheiten bezüglich des Wertes zu rechnen. 
		
		Des weiteren soll das Verfahren unabhängig der verwendeten Sprache und des verwendeten Alphabets durchgeführt werden.

		Es kann also davon ausgegangen werden, dass eine geografische Angabe durch den Nutzer gemacht wird, aber man muss aufgrund der nicht Objektivität und der Unzuverlässigkeit davon ausgehen, dass die Angaben nicht korrekt, nicht überprüfbar, oder überhaupt keine geografische Angabe darstellen. 

		Dies bedeutet zum einen, dass die Werte die im Nutzer-Standort angegeben werden nicht einheitlich sind, und das die eingegebenen Werte nicht zwingend einen geografischen Ort bezeichnen. 

		Auch ist der Nutzer-Standort unzuverlässig, womit
		



		Des weiteren soll nach den Anforderungen in Abschnitt \ref{sec:Anforderungen} die Auswertung unabhängig von der verwendeten Sprache und den verwendeten Schriftzeichen sein. 


		

		In Tabelle \ref{Tabelle mit Besipielen für NICHT-Orte} sind einige Werte aufgeführt die als Nutzer-Standort angegeben wurden, aber keine geografische Position oder geografische Region bezeichnen. 
		Diese Werte sind natürlcih zur Georeferenzierung unbrauchbar. 


		Der Nutzer-Standort ist ein unmittelbarer geografischer Indikator.  
		Als Nutzer-Standort kann der Twitter-Nutzer eine beliebige Zeichenfolge eingeben. 
		Es handelt sich beim Nutzer-Standort deshalb um einen ungesicherten geografischen Indikator, es ist deshalb damit zu rechnen, dass unter Umständen keine geografische Position angegeben ist und andererseits keine einheitliche Angabe bezüglich des selben Standorts erwartet werden kann.
		Beispielsweise beschreiben die Zeichenketten "'Karlsruhe, Deutschland"' und "'Baden-Württemberg, Karlsruhe"' den selben Ort.
		Noch deutlicher wird dieser Umstand, wenn man alternative Namen oder umgangssprachliche Namen für Städte betrachtet. 
		Mit "'The Big Apple"' und "'New York, USA"' oder mit "'Motown"' und "'Detroit, MI"' sind dieselben Orte gemeint.   
		Auch die Genauigkeit bezüglich der geografischen Position ist nicht zuverlässig vorhersagbar, sehr konkrete geografische Positionen, wie die Angabe einer Stadt oder eines Stadtteils, oder aber eine geografische Region wie beispielsweise ein Land oder ein Kontinent, sind möglich. 

		Die Nutzer-Zeitzone stellt dagegen einen gesicherten, unmittelbaren geografischen Indikator dar.
		Bei der Nutzer-Zeitzone kann aus einer Liste möglicher Werte gewählt werden, womit keine Ungenauigkeiten bezüglich der Eingabe besteht und eine definierte Zeichenkette erwartete werden kann, deren geografische Region klar definiert ist. 
		Die Nutzer-Zeitzone beschreibt allerdings in jedem Fall eine größere geografische Region, die nicht immer mit den konventionellen Ländergrenzen korrespondiert und somit eine Bestimmung der geografischen Position nahezu unmöglich macht.
		\todo{Irgendwo auf den Umstand eingehen, dass Timezone nicht angegeben werden wird und dann der Standard gewählt wird der us central pacific time ist? .......} 

		Bei beiden Indikatoren besteht natürlich die Möglichkeit der Falscheingabe durch den Benutzer. Dieser Umstand wird jedoch durch die Analyse der Daten ausgemerzt. \todo{Umschreiben und woanders darauf eingehen!} 

		\todo{Zeitzone als zusätzlichen geografischen Indikator einführen Unsicherheiten ausräumen Beispiel suchen}

		Der Nutzer-Standort wird einer mehrstufigen Vorverarbeitung unterzogen.
		Die Vorverarbeitung des Nutzer-Standortes erfüllt mehrere Aufgaben. 

		Zunächst soll sichergestellt werden, dass keine unnötigen Sonder- oder Satzzeichen in die weitere Verarbeitung einfliessen. 
		Im Nutzer-Standort finden sich oft Sonder- oder Satzzeichen, die für die Bestimmung des Standortes von keinem oder sehr geringem Nutzen sind, diese sollen in der weiteren Verarbeitung nicht berücksichtigt werden.
	  	Des weiteren sollen Toponyme welche aus zwei oder mehr Wörtern bestehen \todo{Toponym definieren} erkannt, und als zusammengehörig markiert werden.
	  	Die erkannten Toponyme werden nicht zur Georeferenzierung genutzt, sondern ausschließlich um die Wörter zusammenfassen zu können und in weiteren Verarbeitungsschritten als zusammengehörige Einheit zu verarbeiten.
	  	Dies bringt den Vorteil, dass in den nächsten Verarbeitungsschritten weniger Wörter verarbeitet werden müssen.
	  	Die so verarbeiteten Worte werden zuletzt in Tokens aufgeteilt und alphanumerisch sortiert. 
	  	Dieser Schritt dient zur Vorverarbeitung für den nächsten Verarbeitugsschritt.

		  	\subsubsection{Eliminierung von Satz- und Sonderzeichen}
		  	Oft werden im Nutzer-Standort Sonder- und Satzzeichen verwendet. 
			Einige Nutzer haben ausschließlich Sonder- und Satzzeichen als Nutzer-Standort angegeben, in diesen Fällen wird davon ausgegangen, dass damit kein geografischer Ort gemeint ist. 
			In den x Millionen verfügbaren Tweets befinden sich y Nutzer welche einen Nutzer-Standort angegeben haben der nur aus Sonderzeichen besteht. 

			Viel öfter werden Sonder- und Satzzeichen als reine Dekoration verwendet oder um Emotionen auszudrücken. 
			Beispiele hierfür sind "'I $\heartsuit$ New York"' oder "'\dagger\textasciitilde Los Angeles\textasciitilde\dagger"'.
			Diese Zeichen geben keine weitere Auskunft über den geografischen Ort an dem sich der Nutzer befindet. 
			Auch Satzzeichen wie ein Punkt oder ein Fragezeichen bieten keinen Mehrwert zur Bestimmung der geografischen Position. 



		  	\subsubsection{Kleinschreibung}

		  	\subsubsection{Tokenisierung}

		  	\subsubsection{Identifizierung von Toponymen innerhalb des Nutzer-Standortes}
		  		Kann bereits in diesem Schritt erkannt werden, dass es sich bei zwei oder mehreren aufeinanderfolgenden Token um ein Toponym handelt werden die Tokens zusammengefasst. 
		  		Dadurch wird die weitere Verarbeitung effizienrter bezüglich der Anzahl der zu verarbeitenden Token. 
		  		Zum Beispiel werden die Token [New] [York] [City] zusammengefasst zu dem Token [New York City]  
		  		Aber insbesondere soll hier nicht das Toponym analysiert und direkt eine Georeferenz hinzugefügt werden. 
		  		Die Identifizierung von toponymen dient lediglich dazu die folgenden Verarbeitungsschritte effizienter zu machen. 
		  	

		  	\subsubsection{alphanumerische Sortierung der Tokens}

		  	\subsubsection{URL Encoding}
		  		\subsection{Encoding}
				Problematik unterschiedlicher Sprachen, 
				url-encoding sinnvoll als Vorbereitung auf Webservice. 

		  	

		\subsection{Die Zeitzone als geografischer Indikator}     

	\section{Einlernen der Georeferenz-Basis}

		\subsection{Zuordnung der georeferenzierten Tweets zu geografischen Entitäten} 

		\subsection{Erzeugung von N-Grammen}

				\paragraph{einbauen als Beispiel für Ngramisierung} 
				Beispiel wenn Karlsruhe Deutschland und Berlin Deutschland und Abstatt Deutschland dann kann durch die geografische Hierarchie und die NGramme Wissen über den Term Deutschland gezogen werden da auf Länderebene Deutschland immer dem gleichen geografischen Ding zugeordnet wird 

				Vorteil Toponym Identifizierung Verarbeitungsschritte:  insbesondere bei der Erzeugung von N-Grammen im Schritt 

		  		\paragraph{Erzeugung von N-Grammen}
			  	Aus den alphanumerisch sortierten Tokens der Vorverarbeitung werden in diesem Schritt N-Gramme erzeugt.
			  	Genauer werden N-Gramme der Ordnung 1, 2 und 3 erzeugt.
			  	Diese werden auch als Mono-, Bi- und Tri-Gramme bezeichnet.  

			  	Jedes dieser so entstandenen N-Gramme wird die Zeitzone angehängt.  
				Das Ergebnis dieser Phase ist eine Datenbasis welche eine Reihe von Zeichenketten aus der Nutzer-Standort und der Nutzer-Zeitzone generiert haben. 
				Jeder dieser Zeicheneketten wird die Häufigkeit ihres Vorkommens und die nächste Stadt mit über 15.000 Einwohnern zugeordnet. 

				
		\subsection{Matching auf N-Gramme und geografische Entität}

			\subsubsection{Counting} 


	\begin{enumerate}
		\item Datenstruktur
		\item Matching
		\item Counting  
	\end{enumerate}

	\section{Georeferenzierung}   
	In der zweiten Phase kann mit Hilfe der Datenbasis die eigentliche Georeferenzierung von Twitter-Nutzern, mit unbekannter geografischer Position, durchgeführt werden. 
	Es wird dem Anwender ermöglicht die Genauigkeit über die geografischen Hierarchien anzugeben und einen Konfidenzschwellwert zu bestimmen.

	Bei der Georeferenzierung kann bestimmt werden wie genau die geografische Position oder die geografische Region bestimmt werden soll. 
	Dabei kann zwischen den bereits in Kapitel \ref{par: geografische Hierarchie} erläuterten Hierarchieebenen gewählt werden.
	Dies hat den Vorteil, dass die Genauigkeit der Georeferenzierung den jeweiligen Anforderungen angepasst werden kann.
	Des weiteren kann eine Konfidenz angegeben werden, desto höher die Konfidenzschwelle gewählt wird desto sicherer ist die Georeferenzierung. 
	
	


\section{Geolocation Mapping}

	\subsection{nearest neighbour mapping}
		\begin{enumerate}
			\item \todo{Welches Fehlermaß kann für das mapping angewandt werden? auf Städteebene gut möglich mit geografischen Distanzen, admin2,amdin1, Land schlecht möglich mit Distanzen} Wie genau kann gemappt werden? Fehler Durchschnitt. 
			\item Mapping auf cities 1000/1000/15000 mit Daten zu durchschnitll. Abstand
			\item Hier ist noch Verbesserungspotenzial -> wenn Mapping Distanz zu weit entfernt -> verwerfen! 
		\end{enumerate} 

\section{Verknüpfung von Indikatoren und geografischen Lokationen zur wiedergewinnung des erlernten Wissens}
	
	\subsection{Generierung eines Wissendatensatzes}

	\subsection{Verknüpfung mit Geodaten}

	\subsection{Auflösen auf Administartionsebenen, Länder}

\section{Lokalisieren von Tweets ohne konkrete geografische Daten}
	
	\subsection{Ablauf der Lokalisierung}
	
	\subsection{Lokalisierungssicherheit durch Ausnutzung der geografischen Hierarchiebeziehungen}


\section*{einbauen!!!}
\subsection{Geografische Grundbegriffe und Geografiedaten}

		\subsubsection{Geografische Grundbegriffe}

		\subsubsection{Geonames.org} \footnote{eventuell erst in Implemetierung darauf eingehen} 
			

		\subsubsection{}	

	\subsection{???} 
		\subsubsection{N-Gramme}
			\begin{enumerate}
				\item NGramme allgemein, Verwendung, Beispiele. 
				\item \todo{NGramme -> Nochmal genau prüfen, Zusammenhang zu Markov Modell und NGram Statistik herausstellen} Zusammenhang zwischen Länge/Grad eines N-Grammes und Wahrscheinlichkeiten. -> mathematische Herleitung?!
			\end{enumerate}




\section{Einbauen in dieses Kapitel}

Die folgenden Vorverarbeitungsschritte werden durchgeführt.

	\begin{enumerate}
		\item Eliminierung von Sonderzeichen
		\item ausschließlich Kleinschreibung
		\item geografische Zuordnung von Teilworten \todo{Bessere Umschreibung finden für Teilworte, es ist eig. eine Menge von Worten aus Nutzer-Standort} 	
		\item Tokenisierung
		\item html Encoding
		\item alphanumerische Sortierung 
	\end{enumerate}

	Diese Vorverarbeitung dient der Bereiningung und Vereinheitlichung des Nutzer-Standorts.  
	Die gegografische Zuordnung dient der  

	In den einzelnen Stufen der Vorverarbeitung werden Sonderzeichen entfernt, alle Zeichen in Kleinbuchstaben umgewandelt, in der Wortfolge werden nach geografischen Begriffen gesucht die zwei oder mehr Worte beinhalten, die Zeichen werden url-encoded und die einzlenen Wörter werden in Tokens zerlegt welche alphanumerisch sortiert werden.
	Im nächsten Schritt werden aus den Tokens Uni-Gramme, Bi-Gramme und Tri-Gramme erstellt. 



	\paragraph{In Bewertung einbauen} 
	Legt man zugrunde, dass der Nutzer-Standort als ungesicherter, unmittelbarer geografischer Indikator angesehen werden kann, besteht eine Möglichkeit der Georeferenzierung darin, eine sogenannte Geocoding-API zu nutzen.
	Die Nutzer-Zeitzone wird in diesem Fall nicht als Indikator herangezogen, da diese von den Geocoding-API's nicht als zusätzliche Information verarbeitet wird. 
	Eine Reihe von bekannten Firmen bietet eine API zur Georeferenzierung an. 
	Die bekanntesten sind Google, Yahoo, Microsoft, Map Quest und Cloud Made. 
	Teilweise sind die Anfragen, welche an die Geocoding-API's gesendet werden können, in ihrer Anzahl begrenzt.
	Auch die Antwortzeiten der Geocoding-APIs begrenzen die Anzahl möglicher Anfragen pro Zeiteinheit. 
	In Tabelle \ref{eine Tabelle mit den Georef API anbietern} ist eine Auflistung der Anbieter mit den jeweiligen Begrenzungen dargestellt.
	\todo{Tabelle mit Georef Api Anbietern.}
	Eine detaillierte Analyse der Antwortzeiten wurde im Zuge dieser Arbeit nicht durchgeführt. \todo{Link in footnote einfügen} \footnote{für eine Analyse Vergleiche} 


	
	Dieses vorgehen wird in einigen Arbeiten angewendet um den Nutzer-Standort zu bestimmen. 
	Dabei wird, mit einer simplen Vorverarbeitung des Nutzer-Standortes, direkt in einer Geografie-Datenbank nach der eingegebenen Zeichenfolge gesucht. 
