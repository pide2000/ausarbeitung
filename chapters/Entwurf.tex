%!TEX root = ../document.tex
\chapter{Lösungsansatz} \label{chp:Loesungsansatz}
In diesem Kapitel wird ein Verfahren zur Lokalisierung von Twitter-Nutzern vorgestellt.
Die Fragestellungen aus Kapitel \ref{sec:fragestellung} werden, unter Berücksichtigung der Anforderungen aus Kapitel \ref{sec:Anforderungen}, beantwortet.  

Zunächst soll ein Überblick über die Funktion und den Ablauf des Verfahrens gegeben werden ohne detailliert auf die einzelnen Verfahrensschritte einzugehen.
Danach wird das Verfahen von Grundauf betrachtet und die einzelnen Verfahrensschritte eingehender erläutert. 

	\section{Überblick} \label{sec:ueberblick} 

	Das erarbeitete Verfahren soll es ermöglichen Twitter-Nutzern, deren geografische Position unbekannt ist, eine Georeferenz zuzuordnen.
	Dabei sollen als Eingabe für die Georeferenzierung lediglich die Nutzer-Zeitzone und der Nutzer-Standort aus dem Profil eines Twitter-Nutzers verwendet werden.
	Als Ergebnis soll eine Georeferenz mit einem Konfidenzwert zurückgeliefert werden. 
	Dabei hat der Anwender die Möglichkeit, sowohl die Genauigkeit bezüglich der geografischen Position, als auch einen Schwellwert für die gewünschte Konfidenz der Georeferenz anzugeben.
	In Abbildung \ref{A1} ist der generelle Ablauf der Georeferenzierung dargestellt. 
	\todo{Schema Darstellung Eingabe->Ausgabe  Sketchbook A1: Unterschrift Die Eingabe besteht aus dem Nutzer-Standort sowie der Nutzer-Zeitzone. 
	Als Rahmenbedingungen wird die gewünschte Hierarchie-Ebene sowie der Schwellwert für die Konfidenz angeben. 
	Als Ausgabe erhält man eine Georeferenz und einen Konfidenzwert, der angibt wie sicher das Ergebnis ist.} 
	\label{A1} 

	Die Georeferenzierung nach dem Schema aus Abbildung \ref{A1} setzt voraus, dass aus den eingegebenen Indikatoren eine Georeferenz abgeleitet werden kann.
	Da es sich beim Nutzer-Standort und der Nutzer-Zeitzone um unmittelbare geografische Indikatoren handelt, lässt sich ein geografischer Bezug aus diesen Indikatoren ableiten.      
	Die Idee besteht nun darin aus einer umfangreichen Tweet-Datensammlung die Zuordnung der eingegebenen Indikatoren zu geografischen Positionen zu lernen und diese in einer Datenbasis zu speichern.
	Diese Datenbasis soll im folgenden als Georeferenz-Basis bezeichnet werden.
	Damit dies realsiert werden kann, muss die Tweet-Datensammlung pro Datensatz den Nutzer-Standort, die Nutzer-Zeitzone und eine Georeferenz in Form von Längen- und Breitengrad beinhalten.  \footnote{Siehe Kapitel \ref{chp:Einleitung} genutzte Daten}
	
	Der Wert des Nutzer-Standortes ist nicht objektiv, nicht zuverlässig und nicht gesichert.
	Dies macht eine Analyse und eine entsprechende Vorverarbeitung dieses Indikators nötig.
	Zur Erstellung der Georeferenz-Basis werden deshalb zunächst einige Schritte zur Vorverarbeitung der Indikatoren durchgeführt.
	Durch die Vorverarbeitung werden die eingegebenen Indikatoren genauer analysiert, zusätzliche Informationen extrahiert und in eine einheitliche Form gebracht.
	Die Indikatoren werden durch die Vorverarbeitung derart verändert, dass aus einem Indikator-Paar mehrere Datensätze enstehen können, welche Informationen aus den ursprünglichen Indikatoren enthalten.
	Danach wird mit Hilfe des Längen- und Breitengrades jedem Tweet eine konkrete Instanz der untersten Ebene der geografischen Hierarchie zugeordnet.
	Legt man die in Abschnitt \ref{chp:Grundlagen} vorgestellte geografische Hierarchie zugrunde, wird dadurch jedem Tweet eine Stadt zugeordnet.
	Durch diese Zuordnung werden jedem Tweet implizit auch die höheren Ebenen der geografischen Hierarchie zugeordnet.   	 
	Dieser Schritt markiert den Übergang von einer kontinuierlichen Darstellung der Georeferenz, durch Längen- und Breitengrad, in eine diskrete Darstellung, durch eine Stadt. 
	Die so verarbeiteten Datensätze werden nun einer simplen statistischen Auswertung unterzogen indem ermittelt wird wie häufig einer Stadt ein bestimmter Indikator zugeordnet wurde.
	Die Georeferenz-Basis beinhaltet nun die verarbeiteten Indikatoren und die zuvor zugeordnete geografische Position, in Form einer Stadt, sowie die Häufigkeit in der die Kombination Indikator/Stadt ermittelt wurde.
	Der Ablauf zur Erstellung der Georeferenz-Basis wird in Abbildung \ref{A2} dargestellt.

	Bei der Georeferenzierung werden die Indikatoren zunächst derselben Vorverarbeitung wie beim einlernen unterzogen.
	Die Vorverarbeitungsschritte sind somit generisch, da sie sowohl vor dem erzeugen der Georeferenz-Basis als auch vor der eigentlichen Georeferenzierung durchgeführt werden.
	Nach der Vorverarbeitung wird in der Georeferenz-Basis nach diesen vorverarbeitenden Indikatoren gesucht.
	Das Ergebnis dieser Suche wird unter Berücksichtigung des Konfidenzschwellwertes und der geografischen Hierarchie ausgewertet und die zugeordnete geografische Position wird ausgegeben.  
	Der Ablauf der Georeferenzierung wird in Abbildung \ref{A3} dargestellt. 

	\todo{Ablaufplan A1, A2, A3} 

	In den folgenden Kapiteln soll nun genauer auf die einzelnen Teile des Verfahrens eingegangen werden. 
	Das Verfahren wird dabei in zwei Teilen behandelt.

	Im ersten Teil wird das Einlernen der Georeferenz-Basis behandelt.
	Im Zuge des Einlernens der Georeferenz-Basis sollen auch die generischen Vorverarbeitungsschritte erläutert werden.

	Die Vorverarbeitung soll den Nutzer-Standort in eine Form bringen die weiter verarbeitet werden kann, und aus der sich zusätzlich Informationen ableiten lassen.
	Die Art der Vorverarbeitung hängt mit der verwendeten geografischen Hierarchie und der späteren Auswertung zusammen.
	Deshalb werden die Vorverarbeitungsschritte einzeln und in Bezug auf die verwendete geografische Hierarchie und das Auswertungs-Verfahren zur Erstellung der Georeferenz-Basis betrachtet.
	Danach soll der gesamte Ablauf des Einlernens noch einmal dargestellt und kurz erklärt werden um einen Gesamtüberblick zu liefern.
	
	Zuletzt soll auf die Georeferenzierung eingegangen werden.
	Die Vorverarbeitungsschritte der Indikatoren wird in diesem Teil nur angeschnitten, da diese bereits im ersten Teil genau erläutert werden. 
	Nachdem die Indikatoren vorverarbeitet wurden, wird in der Georeferenz-Basis nach korrespondierenden Indikatoren gesucht. 
	Die Ergebnisse dieser Suche werden unter Berücksichtigung des Konfidenzschwellwertes und der gewünschten georafischen Genauigkeit, in Form der geografischen Hierarchieebene, ausgewertet.
	Das Ergebnis dieser Auswertung ist die gewünschte Georeferenz. 

	\section{Einlernen der Georeferenz-Basis}
	Zunächst soll der Nutzer-Standort als geografischer Indikator genauer betrachtet werden, dabei werden Probleme identfiziert die durch die Eigenschaften des Nutzer-Standortes entstehen.
	Es wird herausgearbeitet warum ein naiver Ansatz mit einer Abfrage an ein Ortsverzeichnis das Problem der Georeferenzierung nicht befriedigend lösen kann.  
	Danach werden die einzelnen Vorverarbeitungsschritte erläutert und insbesondere der Zusammenhang zwischen den Vorverarbeitungsschritten und der geografischen Hierarchie und der Auswertung der Ergebnisse eingegangen.   

			\subsection{Der Nutzer-Standort als geografischer Indikator}

			Zunächst soll der Nutzer-Standort als geografischer Indikator eingehender betrachtet werden. 
			In Abschnitt \ref{chp:Grundlagen} wurde der Nutzer Standort bezüglich seiner Eigenschaften untersucht. 
			Demnach kann der Nutzer-Standort als unmittelbarer geografischer Indikator angesehen werden, da als Eingabe eine geografische Position gefordert wird.  
			In zahlreichen anderen Arbeiten wurde deshalb versucht den Nutzer-Standort \todo{Referenzen auf frühere Ansätze mit Gazetteer Abfrage} durch eine Anfrage an ein Ortsverzeichis auf eine konkrete geografische Position aufzulösen.

			Dieser naive Ansatz ignoriert allerdings die Eigenschaften des Wertes der im Nutzer-Standort gespeichert ist und die folgenden Probleme verursacht.

			Nach Abschnitt \ref{chp:Grundlagen} ist der Wert des Nutzer-Standortes nicht objektiv, nicht zuverlässig und nicht gesichert. 
			Im folgenden werden die einzelnen Eigenschaften in Bezug zum Nutzer-Standort betrachtet und die enstehenden Probleme benannt.
			
			\paragraph{Nicht zuverlässig} 
				
				Diese Eigenschaft sagt aus, dass der eingegebene Wert nicht zwingend ein Toponym darstellt.
				Vom Nutzer kann jeder beliebige Wert eingegeben werden, unabhängig davon ob dieser ein Toponym ist oder nicht.
				Beim naiven Ansatz, der Abfrage an ein Ortsverzeichnis wird allerdings implizit angenommen das es sich beim eingegebenen Wert um ein Toponym handelt.
				Dadurch kann es zu Fehlern bei der Auflösung kommen.
				Dadurch das der Wert nicht zuverlässig ist wird die Eigenschaft, dass der Nutzer-Standort ein unmittelbarer geografischer Indikator ist, in gewisser Weise relativiert.
				Denn theoretisch könnte jeder Nutzer einen Wert eingeben, der kein Toponym darstellt, womit die Eigenschaft des unmittelbaren geografischen Indikators hinfällig wäre.
				Es muss allerdings beachtet werden, dass der Nutzer-Standort von einem Großteil der Nutzer zu seinem vorgesehenen Zweck genutzt wird.
				Nach Hecht et al in \cite{Hecht2011} wird in \todo{xyz Prozent} der Fälle der Nutzer-Standort verwendet um ein Toponym anzugeben. \footnote{Hecht et al untersuchen allerdings nur Nutzer-Standorte die in englischer Sprache eingegeben wurden.}
				Im Zuge der vorliegenden Arbeit wurde eine manuelle Untersuchung an xyz Tausend Twitter-Profilen vorgenommen bei denen xyz Prozent der Nutzer ein Toponym im Nutzer-Standort gespeichert hatten.   

				\todo{kleines Beispiel mit Hilfe der Geocoding Api erstellen} 

				Problem 1: Der Nutzer-Standort ist nicht zwingend ein Toponym.

			\paragraph{Nicht gesichert} 

				Dadurch besteht keine Garantie, dass der eingegebene Wert, sollte er ein Toponym darstellen, dem tatsächlichen Standort des Nutzers entspricht. 
				Dies kann durch verschiedene Umstände der Fall sein.
				Zum einen kann es sich um eine bewusste Fehleingabe des Nutzers handeln, hier wird eine korrekte Georeferenzierung schwierig uzusetzen sein.

				Zum anderen kann dies aufgrund des Toponyms und dessen Eigenschaften selbst vorkommen.

				Es gibt beispielsweise zahlreiche Städte-Namen, die in mehreren Ländern verwendet werden.
				Ein gutes Beispiel hierfür sind US Städte. 
				Da die USA ein Einwanderungsland ist übernahmen viele Einwanderer bei der Gründung neuer Städte die Namen aus der alten Heimat. 
				So finden sich in den USA zahlreiche Städte deren Namen exakt den deutschen Städtenamen entsprechen. 
				Es gibt in den USA 30 Städte mit dem Namen Hamburg, 40 mit dem Namen Hannover, 39 mit dem Namen Berlin und 25 mit dem Namen Frankfurt um nur einige beispiele zu nennen.
				Es gibt auch Orte die in einem Land mehrfach vorkommen.
				Innerhalb Deutschlands gibt es beispielsweise 12 Gemeinden mit dem Namen Hausen und zahlreiche Stadtteile die diesen Namen tragen.  
				Gibt ein Nutzer den Standort Hamburg ohne genauere Bezichnung an, kann nicht mit Sicherheit gesagt werden welches HAmburg er meint. 

				Problem 2.1: Der Nutzer-Standort entspricht nicht zwingend dem tatsächlichen Standort des Twitter-Nutzers aufgrund von bewussten Fehleingaben des Nutzers.
				Problem 2.2: Der Nutzer-Standort entspricht nicht zwingend dem tatsächlichen Standort des Twitter-Nutzers aufgrund von Doppel- oder Mehrdeutigkeiten des Toponyms.

				\todo{Hier noch überlegen ob Nutzer der sich temporär woanders befindet auch rein müssen.} 

			\paragraph{Nicht Objektiv} 

				Dies bedeutet, dass dasselbe geografische Objekt von unterschiedlichen Nutzern mit unterschiedlichen, gültigen Toponymen bezeichnet werden kann.
				Für den naiven Ansatz stellt dies kein Problem dar solange sämtliche verwendeten Toponyme im Ortsverzeichnis hinterlegt sind.  
				Dieser Umstand muss allerdings beachtet werden, da die umgangssprachlichen Toponyme oft nicht den offiziellen entsprechen. 
				Als Beispiel kann hier Deutschland oder USA gewählt werden. 
				Jedem dürfte bekannt sein, das mit diesen Bezeichnungen zum einen die '"Bundesrepublik Deutschland'" und zum anderen die "'Vereinigten Staaten von Amerika"' gemeint sind. 
				Dieser Umstand muss aleerdings beachtet werden.
				Ein weiteres Problem, welches hier offensichtlich wird, ist die Sprache in welcher der Nutzer-Standort angegeben ist. 
				USA sollte deshalb korrekterweise als "'United States of America"' angegeben werden.
				Für den naiven Ansatz, mit Hilfe eines Ortsverzeichnisses, ist es also notwendig Toponyme in der jeweiligen Landessprache und zusätzlich einige gängige, alternative, umgangssprachliche Toponyme auflösen zu können.  

				Problem 3: Es können mehrere unterschiedliche Toponyme für dasselbe geografische Objekt existieren.  


			\paragraph{Nicht Objektivität in Kombination mit nicht-Zuverlässigkeit des Wertes} 

				Die nicht-Zuverlässigkeit des Wertes besagt, dass der eingegebene Wert kein Toponym darstellen muss.
				Insbesondere kann aber keine Aussage darüber gemacht werden ob es sich beim Nutzer-Standort um ein Toponym handelt oder nicht. 
				Wie vorher bereits erwähnt, können aufgrund der nicht-Objektivität zum selben geografischen Objekt mehrere Toponyme existieren.
				Es kann daraus abgeleitet werden, dass immer ein weiteres Toponym existieren kann.
				In Kombination mit der nicht-Zuverlässigkeit kann angenommen werden, dass dieses zusätzliche Toponym nicht als solches bekannt ist und deshalb nicht als Toponym identifiziert werden kann.

				Es kann deshalb nicht entschieden werden ob der eingegebene Wert tatsächlich ein Toponym darstellt oder nicht. 
				
				Dies kann der Fall sein wenn ein Ort eine Bezeichnung besitzt die nicht allgemein bekannt ist und somit in keiner Datenbank hinterlegt sein kann. 
				Beispielsweise werden in Wikipedia für die Stadt Detroit, im US-Bundesstaat Michigan, folgende Spitznamen angegeben: The Motor City, Motown, Hockeytown, Rock City und The D. 
				Die ersten zwei dürften weltweit einen Gewissen Bekanntheitsgrad haben wohingegen Hockeytown, Rock City und The D weniger bekannt sein dürften.  
				Es existieren Datenbanken mit solchen Städte-Spitznamen, aber für diese kann keine Vollständigkeit garantiert werden. 
				
				Auch besteht die Möglichkeit, dass Spitznamen für Städte existieren die kaum über die Grenzen eines Landes oder sogar der Stadt selbst bekannt sind. 
				Eine weitere Fehlerquelle sind netzwerkinterne Schreibweisen die sich etabliert haben und ausschließlich innerhalb eines Netzwerks verwendet werden.

				Neben diesen Beispielen sind aber auch Toponyme denkbar die einen bestimmten Landstrich oder eine landschaftliche Besonderheit beschreiben.
				Beispielsweise beschreibt '"An der Förde'" eine geografische Region an der Ostsee, meist ist damit die Kieler Förde und dessen Umgebung gemeint. 
				Diese geografischen Bezeichnungen sind unter Umständen nur in speziellen Datenbanken zu finden und können so nur schlecht aufgelöst werden. 

				Problem 4: Toponyme können in ihrer Verwendung gänzlich unbekannt sein und somit nicht als solche erkannt werden.

			\paragraph{Weitere Probleme}

				Die vorgannanten Eingenschaften resultieren daraus, dass der Nutzer-Standort frei eingegeben werden kann und keinerlei Kontrolle oder Verarbeitung seitens Twitter unterliegt. 
				Neben den oben genannten Problemen können dadurch weitere Probleme entstehen.

				Durch die freie Eingabe können Nutzer auch ausschliesslich Sonder- und Satzzeichen eingeben womit kein bestimmtes geografisches Objekt referenziert wird. 
				Des weiteren können Sonder- und Satzzeichen als Trenner benutzt werden zwischen zwei oder mehr Worten welche ein Toponym bilden.
				
				Problem 5: Satz- und Sonderzeichen.

				Nach den Anforderungen aus Kapitel \ref{sec:Anforderungen} soll die Georeferenzierung unabhängig der verwendeten Sprache und des verwendeten Alphabets funktionieren.

				Problem 6: Sprache- und verwendetes Alphabet.

			\paragraph{Fazit} 

				Durch eine einfache Abfrage auf ein Ortsverzeichnis können die vorgenannten Probleme nicht vermieden werden. 

				Durch folgende Massnahmen kann versucht werden einige Probleme zu vermeiden. 

				Problem 1 und Problem 3 : 
				Durch ein möglichst umfangreiches Ortsverzeichnis oder die Verwendung mehrerer Ortsverzeichnisse kann versucht werden diese Probleme zu lösen.  
				Problem 2.1 : Kann nicht gelöst werden ohne weiteres Wissen über den Nutzer zu erhalten.
				Problem 2.2 : Kann unter Umständen gelöst werden durch die Einbeziehung weiterer Felder oder durch genauere Analyse des Wertes. Beispielsweise könnten zunächst alle Wörter im Wert einzeln untersucht werden und im Ortsverzeichnis abgefragt werden. Danach werden alle Paare von Wörtern abgefragt und so weiter. Werden mehrere Treffer zurückgeliefert wird überprüft ob diese eine Hierarchische Beziehung zueinander aufweisen. Ist dies der Fall kann entschieden werden welches Toponym gewählt wird. \todo{Beispiel Ablauf Hamburg, Deutschland}    
				Problem 5 : Kann durch die Eliminierung von Satz- und Sonderzeichen im Wert des Nutzer-Standortes gelöst werden. 
				Problem 6 : kann durch die Verwendung von Ortsverzeichnissen in verschiedenen Sprachen gelöst werden.

				Die Abfrage auf ein Ortsverzeichnis kann nur diejenigen Orte auflösen, welche hinterlegt sind. 
				Diese Tatsache betrifft in gewissem Maße die Probleme 1, 2.2, 3, 5 und 6 insbesondere aber kann Problem 4 nicht gelöst werden.

				Die Idee besteht deshalb darin den Nutzer-Standort so zu behandeln als ob keine Information direkt aus ihm entnommen werden kann.
				Dies bedeutet, der Wert des Nutzer-Standortes wird nicht inhaltlich nach Toponymen untersucht, sondern die verwendeten Toponyme und ihre geografische Zuordnung sollen aus einer Sammlung von Tweets gelernt werden.
				Es soll also aus den angegebenen Werten für den Nutzer-Standort eine Zuordnung zu geografischen Positionen gelernt werden. 
				Dadurch lässt sich eine Art spezifisches Ortsverzeichnis, die Georeferenz-Basis, für die untersuchte Domäne erstellen.
				Im vorliegenden Fall für den Nutzer-Standort in Twitter.
				
				Im folgenden werden die einzelnen Schritte zur Erstellung der Georeferenz-Basis betrachtet und die Probleme angesprochen welche dadurch gelöst werden können.


			\subsection{Die Georeferenz-Basis} 

				Die Georeferenz-Basis kann als Tabellenstruktur dargestellt werden. 
				Jeder Eintrag der Tabelle beinhaltet mehrere Felder und wird als Datensatz bezeichnet.

				Da während der Georeferenzierung die Indikatoren nachgeschlagen werden sollen muss ein Feld vorhanden sein welches die verarbeiteten Indikatoren beinhaltet.
				Dieses Feld wird als Indikatorfeld bezeichnet. 

				Des weiteren muss ein Feld zur Georeferenzierung enthalten sein, um dem Indikator-Wert ein geografisches Objekt zuweisen zu können. 
				Dieses Feld wird als Georeferenzfeld bezeichnet. 

				Des weiteren soll ein Feld vorhanden sein, welches angibt wie häufig die Kombination Indikator zu Georeferenz vorgekommen ist. 
				  

				Zunächst findet eine Vorverarbeitung der Indikatoren statt. 
				Durch die Vorverarbeitung können mehrere Werte für einen Tweet und somit für ein Indikator-Paar enstehen. 
				Das bedeuete die Vorverarbeitung erzeugt pro Tweet meherere Datensätze in der Georeferenz-Basis.

				Wie bereits im Abschnitt \ref{sec:ueberblick} erwähnt soll die Georeferenz-Basis zum Nachschlagen der Indikatoren dienen.  

				Die Idee besteht darin eine Datenbasis aufzubauen welche zum Nachschlagen der Indikatoren genutzt werden kann um diesen eine Georeferenz zuzuordnen.
				In dieser Hinsicht unterscheidet sich die hier vorgestellte Lösung nicht von der Georeferenzierung mit Hilfe eines Ortsverzeichnisses.
				In beiden Fällen werden die Werte der Indikatoren in einem Verzeichnis nachgeschlagen. 
				
				Die Georeferenz-Basis muss also potentielle Werte für die Indikatoren beinhalten.
				Im Gegensatz zum Ortsverzeichnis werden diese Werte jedoch mit Hilfe einer Tweet-Sammlung eingelernt.
				Diese Werte sollen als Indikator-Wert bezeichnet werden.

				Des weiteren muss eine Georeferenz hinterlegt sein welche den Indikator-Werten ein geografisches Objekt zuordnet.

				
				Die Indikator-Werte werden aus den Indikatoren in der Tweet-Sammlung generiert.
				Dabei können aus einem Tweet und dessen Indikatoren mehrere Einträge für die Indikator-Werte erzeigt werden. 
				Diese  

				Das Feld in dem diese Werte zum nachschlagen hinterlegt sind wird Indikator-Wert genannt. 
				Die Indikator-Werte werden durch die Vorverarbeitung erzeugt.   
				Es ist also möglich, dass Werte mehrmals vorkommen. 
				Diesem Umstand wird Rechnung getragen indem ein Feld eingeführt wird welches das Vorkommen der jeweiligen Indikatoren zählt. 
				Dieses Feld wird als Indikator-Anzahl bezeichnet.

				Zu jedem Indikator-Wert wird die Anzahl 

				Des weiteren werden die Indikatoren einer Vorverarbeitung unterzogen um mehr Informationen aus ihnen extrahieren zu können. 

				Die hinterlegte Georeferenz entspricht immer einer Stadt, einer Administartionsebene zweiter Ordnung, einer Administartionsebene erster Ordnung und einem Land. 
				Um dies aus zu erreichen muss jeder Tweet auf eine Stadt abgebildet werden.


		  	\subsubsection{Eliminierung von Satz- und Sonderzeichen}

			  	Oft werden im Nutzer-Standort Sonder- und Satzzeichen verwendet. 
				Einige Nutzer haben ausschließlich Sonder- und Satzzeichen als Nutzer-Standort angegeben, andere benutzen diese um Emotionen auszudrücken oder als Dekoration. 
				Beispiele hierfür sind "'I $\heartsuit$ New York"' oder "'$\dagger\textasciitilde$ Los Angeles$\textasciitilde\dagger$"'.
				Es konnte kein Hinweis darauf gefunden werden, dass Sonder- und Satzzeichen zusätzliche Informationen zum angegebenen Nutzer-Standort liefern. 
				In den x Millionen verfügbaren Tweets befinden sich y Nutzer welche einen Nutzer-Standort angegeben haben der ausschließlich aus Sonder- oder Satzzeichen besteht. 
				\todo{Auswertung von Sonder Satzzeichen Datensätzen in UL} 
				Es wurden lediglich einige Beispiele gefunden in denen Satzzeichen ein Bestandteil des Standortes bilden.
				Oft werden Kommas zur hierachischen Trennung zwischen den Toponymen verwendet.
				"'Karlsruhe, Deutschland"', "'San Diego CA, USA"' oder "'Mühlhausen, Thüringen"' sind Beispiele hierfür.
				Da die Verwendung allerdings nicht einheitlich erfolgt ist es schwierig daraus eine tatsächliche Hierarchie Beziehung abzuleiten.

				Auch als Bestandteil eines Toponyms können Sattzzeichen verwendet werden.
				Ein Beispielhierfür ist "'Paris, 3. Arrondissement"'.
				Entfernt man die Satzzeichen, liefern diese Werte allerdings immernoch ausreichende Informationen über den Standort.  
				In diesem Vorverarbeitungsschritt werden alle Sonder- und Satzzeichen entfernt. 
				Dadurch wird gewährleistet, dass keine unnötigen Zeichen in den folgenden Schritten verarbeitet werden müssen. 

			\subsubsection{Tokenisierung}

				Das Verfahren macht es nötig die Wörter im Nutzer-Standort einzeln zu behandeln. 
				Deshalb besteht der zweite Schritt des Verfahrens darin, den Nutzer-Standort zu zerlegen und jedes darin vorkommende Wort als einzelnen Token zu betrachten. 
				Dabei wird die Reihenfolge der Wörter zunächst nicht verändert.

				Um Tokens zu Kennzeichnen werden eckige Klammern verwendet.  
				Mit [New] ist dann der Token New gemeint. 
				"'New York City"' würde nach der Tokenisierung als [New][York][City] dargestellt werden. 

				\todo{Ablaufplan Tokenizing} 

			\subsubsection{N-Gramme erstellen}

				Die Kernidee besteht darin den Nutzer-Standort in N-Gramme zu zerlegen. 


		  	\subsubsection{Identifizierung von Toponymen innerhalb des Nutzer-Standortes}
		  		
		  		In diesem Schritt wird nun die Eigenschaft des unmittelbaren geografischen Indikators ausgenutzt.
		  		Die Idee besteht darin, Toponyme die aus zwei oder mehr Tokens bestehen zusammenfassen zu können und in den folgenden Schritten als zusammengehörig zu behandeln.

		  		Als Beispiel kann hier der Nutzer-Standort [New][York][City][United][States][of][America] betrachtet werden. \footnote{Dies ist ein fiktives Extrembeispiel für einen Nutzer-Standort nach Eliminierung der Sonder- und Satzzeichen und nach der Tokenisierung}
		  		Nach einer Abfrage auf ein Ortsverzeichnis können [New][York][City] und [United][States][of][America] zusammengeführt werden, woraus sich folgende Tokens ergeben:
		  		[New York City][United States of America]   

		  		Dieser Schritt vereinfacht die künftige Verarbeitung, da im Beispiel statt 6 Tokens nur noch 2 betrachtet werden müssen, welche überdies als Toponym erkannt wurden.
		  		Dadurch werden in diesem Schritt zwar zusätzliche Informationen aus dem Nutzer-Standort gezogen, allerdings wird keine konkrete Georeferenz dem Nutzer-Standort zugeordnet.
		  		In diesem Schritt soll der Indikator nicht auf ein geografische Objekt aufgelöst werden. 
		  		Die Abfrage dient lediglich dazu, zusammengehörige Token zusammenzuführen und diese in den folgenden Schritten als ein einzelner Token zu behandeln. 

		  		\todo{Ablaufplan Toponyme identifizieren} 



		  	\subsubsection{Alphanumerische Sortierung}

		  		In diesem Schritt werden die Token alphanumerisch sortiert. 
		  		Der Vorteil ist, dass die Reihenfolge in der die Token angegeben wurden keine Rolle mehr spielt. 
		  		Ist in einem Nutzer-Standort das Land und eine Stadt genannt, spielt es keine Rolle ob das Land oder die Stadt zuerst genannt wird. 

		  		Als Beispiel soll diesmal die Stadt San Diego im Budnesstaat Kalifornien (CA) in den Vereinigten Staaten von Amerika (USA) dienen. 
		  		Zwie Nutzer geben ihre Standorte folgendermaßen an:

		  		\begin{enumerate}
		  			\begin[Nutzer 1]  	[San Diego][California][USA]
		  			\begin[Nutzer 2] 	[USA][California][San Diego] 
		  		\end{enumerate}

		  		Direkt kann keine Übereinstimmung festgestellt werden. 
		  		Es müsste kontrolliert werden ob alle Token übereinstimmen.
		  		Sortiert man die Tokens vorher, kann die Übereinstimmung direkt erkannt werden.
		  		In beiden Fällen würde die Sortierung [California][San Diego][USA] ergeben.
		  		Es werden also nicht zwei Datensätze mit denselben Token angelegt, sondern die beiden Eingaben werden gleich behandelt.




		\todo{Zeitzone als zusätzlichen geografischen Indikator einführen Unsicherheiten ausräumen Beispiel suchen}

		  	\subsubsection{alphanumerische Sortierung der Tokens}

		  	\subsubsection{URL Encoding}
		  		\subsection{Encoding}
				Problematik unterschiedlicher Sprachen, 
				url-encoding sinnvoll als Vorbereitung auf Webservice. 

		  	

		\subsection{Die Zeitzone als geografischer Indikator}     

	\section{Einlernen der Georeferenz-Basis}

		\subsection{Zuordnung der georeferenzierten Tweets zu geografischen Entitäten} 

		\subsection{Erzeugung von N-Grammen}

				\paragraph{einbauen als Beispiel für Ngramisierung} 
				Beispiel wenn Karlsruhe Deutschland und Berlin Deutschland und Abstatt Deutschland dann kann durch die geografische Hierarchie und die NGramme Wissen über den Term Deutschland gezogen werden da auf Länderebene Deutschland immer dem gleichen geografischen Ding zugeordnet wird 

				Vorteil Toponym Identifizierung Verarbeitungsschritte:  insbesondere bei der Erzeugung von N-Grammen im Schritt 

		  		\paragraph{Erzeugung von N-Grammen}
			  	Aus den alphanumerisch sortierten Tokens der Vorverarbeitung werden in diesem Schritt N-Gramme erzeugt.
			  	Genauer werden N-Gramme der Ordnung 1, 2 und 3 erzeugt.
			  	Diese werden auch als Mono-, Bi- und Tri-Gramme bezeichnet.  

			  	Jedes dieser so entstandenen N-Gramme wird die Zeitzone angehängt.  
				Das Ergebnis dieser Phase ist eine Datenbasis welche eine Reihe von Zeichenketten aus der Nutzer-Standort und der Nutzer-Zeitzone generiert haben. 
				Jeder dieser Zeicheneketten wird die Häufigkeit ihres Vorkommens und die nächste Stadt mit über 15.000 Einwohnern zugeordnet. 

				
		\subsection{Matching auf N-Gramme und geografische Entität}

			\subsubsection{Counting} 


	\begin{enumerate}
		\item Datenstruktur
		\item Matching
		\item Counting  
	\end{enumerate}

	\section{Georeferenzierung}   
	In der zweiten Phase kann mit Hilfe der Datenbasis die eigentliche Georeferenzierung von Twitter-Nutzern, mit unbekannter geografischer Position, durchgeführt werden. 
	Es wird dem Anwender ermöglicht die Genauigkeit über die geografischen Hierarchien anzugeben und einen Konfidenzschwellwert zu bestimmen.

	Bei der Georeferenzierung kann bestimmt werden wie genau die geografische Position oder die geografische Region bestimmt werden soll. 
	Dabei kann zwischen den bereits in Kapitel \ref{par: geografische Hierarchie} erläuterten Hierarchieebenen gewählt werden.
	Dies hat den Vorteil, dass die Genauigkeit der Georeferenzierung den jeweiligen Anforderungen angepasst werden kann.
	Des weiteren kann eine Konfidenz angegeben werden, desto höher die Konfidenzschwelle gewählt wird desto sicherer ist die Georeferenzierung. 
	
	


\section{Geolocation Mapping}

	\subsection{nearest neighbour mapping}
		\begin{enumerate}
			\item \todo{Welches Fehlermaß kann für das mapping angewandt werden? auf Städteebene gut möglich mit geografischen Distanzen, admin2,amdin1, Land schlecht möglich mit Distanzen} Wie genau kann gemappt werden? Fehler Durchschnitt. 
			\item Mapping auf cities 1000/1000/15000 mit Daten zu durchschnitll. Abstand
			\item Hier ist noch Verbesserungspotenzial -> wenn Mapping Distanz zu weit entfernt -> verwerfen! 
		\end{enumerate} 

\section{Verknüpfung von Indikatoren und geografischen Lokationen zur wiedergewinnung des erlernten Wissens}
	
	\subsection{Generierung eines Wissendatensatzes}

	\subsection{Verknüpfung mit Geodaten}

	\subsection{Auflösen auf Administartionsebenen, Länder}

\section{Lokalisieren von Tweets ohne konkrete geografische Daten}
	
	\subsection{Ablauf der Lokalisierung}
	
	\subsection{Lokalisierungssicherheit durch Ausnutzung der geografischen Hierarchiebeziehungen}


\section*{einbauen!!!}
\subsection{Geografische Grundbegriffe und Geografiedaten}

		\subsubsection{Geografische Grundbegriffe}

		\subsubsection{Geonames.org} \footnote{eventuell erst in Implemetierung darauf eingehen} 
			

		\subsubsection{}	

	\subsection{???} 
		\subsubsection{N-Gramme}
			\begin{enumerate}
				\item NGramme allgemein, Verwendung, Beispiele. 
				\item \todo{NGramme -> Nochmal genau prüfen, Zusammenhang zu Markov Modell und NGram Statistik herausstellen} Zusammenhang zwischen Länge/Grad eines N-Grammes und Wahrscheinlichkeiten. -> mathematische Herleitung?!
			\end{enumerate}




\section{Einbauen in dieses Kapitel}

Die folgenden Vorverarbeitungsschritte werden durchgeführt.

	\begin{enumerate}
		\item Eliminierung von Sonderzeichen
		\item ausschließlich Kleinschreibung
		\item geografische Zuordnung von Teilworten \todo{Bessere Umschreibung finden für Teilworte, es ist eig. eine Menge von Worten aus Nutzer-Standort} 	
		\item Tokenisierung
		\item html Encoding
		\item alphanumerische Sortierung 
	\end{enumerate}

	Diese Vorverarbeitung dient der Bereiningung und Vereinheitlichung des Nutzer-Standorts.  
	Die gegografische Zuordnung dient der  

	In den einzelnen Stufen der Vorverarbeitung werden Sonderzeichen entfernt, alle Zeichen in Kleinbuchstaben umgewandelt, in der Wortfolge werden nach geografischen Begriffen gesucht die zwei oder mehr Worte beinhalten, die Zeichen werden url-encoded und die einzlenen Wörter werden in Tokens zerlegt welche alphanumerisch sortiert werden.
	Im nächsten Schritt werden aus den Tokens Uni-Gramme, Bi-Gramme und Tri-Gramme erstellt. 



	\paragraph{In Bewertung einbauen} 
	Geocoding nach Ortsverzeichnis umbauen
	Legt man zugrunde, dass der Nutzer-Standort als ungesicherter, unmittelbarer geografischer Indikator angesehen werden kann, besteht eine Möglichkeit der Georeferenzierung darin, eine sogenannte Geocoding-API zu nutzen.
	Die Nutzer-Zeitzone wird in diesem Fall nicht als Indikator herangezogen, da diese von den Geocoding-API's nicht als zusätzliche Information verarbeitet wird. 
	Eine Reihe von bekannten Firmen bietet eine API zur Georeferenzierung an. 
	Die bekanntesten sind Google, Yahoo, Microsoft, Map Quest und Cloud Made. 
	Teilweise sind die Anfragen, welche an die Geocoding-API's gesendet werden können, in ihrer Anzahl begrenzt.
	Auch die Antwortzeiten der Geocoding-APIs begrenzen die Anzahl möglicher Anfragen pro Zeiteinheit. 
	In Tabelle \ref{eine Tabelle mit den Georef API anbietern} ist eine Auflistung der Anbieter mit den jeweiligen Begrenzungen dargestellt.
	\todo{Tabelle mit Georef Api Anbietern.}
	Eine detaillierte Analyse der Antwortzeiten wurde im Zuge dieser Arbeit nicht durchgeführt. \todo{Link in footnote einfügen} \footnote{für eine Analyse Vergleiche} 


	
	Dieses vorgehen wird in einigen Arbeiten angewendet um den Nutzer-Standort zu bestimmen. 
	Dabei wird, mit einer simplen Vorverarbeitung des Nutzer-Standortes, direkt in einer Geografie-Datenbank nach der eingegebenen Zeichenfolge gesucht. 
