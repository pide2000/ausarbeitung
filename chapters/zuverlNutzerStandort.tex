\subsubsection{Zuverlässigkeit des Nutzer-Standortes}

				Es soll die Frage beantwortet werden ob der Nutzer-Standort mit dem tatsächlichen Aufenthaltsort des Nutzers übereinstimmt.
				Diese Frage ist schwierig und nicht mit absoluter Sicherheit zu beantworten.

				Vergleicht man den aufgelösten Nutzer-Standort mit den geografischen Koordinaten in einem Tweet können folgende Problem auftreten:

				Betrachtet man den Nutzer-Standort, so ist es schwierig diesen korrekt auf eine Georeferenz aufzulösen. 
				Zudem ist nicht klar ob der Nutzer-Standort tatsächlich dem

				Es müssen hierzu einige Annahmen getroffen werden.


				Um diese Frage beantworten zu können sollen die geografischen Koordinaten eines Tweets mit den 

				Zunächst ist der Nutzer-Standort als solcher unsicher. 
				Selbst wenn er richtig aufgelöst werden würde, kann der Nutzer sich zum Zeitpunkt des Absendens eines Tweets an einer anderen Position befunden haben.
				Befindet sich der Nutzer auf Reisen oder arbeitet er in einer anderen Stadt, stimmen die geografischen Koordinaten des Tweets nicht mit den Koordinaten des angegebenen Ortes überein.

				Der Nutzer kann aber auch absichtlich einen falschen Standort angeben oder bei einem Umzug schlicht vergessen haben den Nutzer-Standort anzupassen.

				Zuletzt kann die manuell bestimmte Position fehlerhaft sein.
				Bei Doppeldeutigkeiten oder ähnlichem ist es schwierig zu entscheiden welche Georeferenz gewählt werden soll.

				In der durchgeführten Untersuchung wurde auch die Entfernung der manuell bestimmten Stadt zu den tatsächlichen Tweet Koordinaten bestimmt.
				Auf Städteebene wurden 60\% der Tweets in weniger als 24 Kilometern Entfernung von der manuell ermittelten Stadt abgesetzt.
				25\% der Tweets sind aber mehr als 239 Kilometer von der manuell ermittelten Stadt entfernt. 
				Der Median liegt bei 11 Kilometern.
				In Tabelle \ref{tab:distancesP} sind weitere Informationen über die Distanzen zusammengetragen.

				\begin{table}[h]
				\centering
				\caption{Distanzen}
				\label{tab:distancesP}
				\begin{tabular}{|l|l|}
				\hline
				Wert 		 & Entfernung in km \\ \hline \hline
				Median       & 11,5  \\ \hline
				0,25 Quantil & 3,1   \\ \hline
				0,50 Quantil & 11,5  \\ \hline
				0,60 Quantil & 24,3  \\ \hline
				0,75 Quantil & 239,0 \\ \hline
				\end{tabular}
				\end{table}

				%6625.059088      11491.29743       24281.1066      106942.0694